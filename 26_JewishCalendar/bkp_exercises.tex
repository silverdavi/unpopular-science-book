\fullpageexercises{%
\textbf{Basic Exercises: Jewish Calendar Systems} \\[1em]
These exercises explore the mathematical and practical challenges of the Jewish calendar system, from basic molad calculations to complex geographical edge cases.

\vspace{1em}

\textbf{1. Molad Calculation and Accuracy} \\[0.5em]
The traditional Jewish lunar month is 29 days, 12 hours, 44 minutes, 3⅓ seconds. The modern astronomical value is 29.530589 days. \\
\emph{Tasks:} (a) Convert both values to decimal days. (b) Calculate the accumulated error after 1000 lunar months. (c) How many hours off would the molad be after 10 centuries?

\vspace{1em}

\textbf{2. Metonic Cycle Mathematics} \\[0.5em]
The Jewish calendar uses a 19-year cycle with 7 leap years to align lunar months with solar years. \\
\emph{Questions:} (a) Calculate exactly how many days are in 19 solar years vs. 235 lunar months using modern values. (b) How close is this approximation? (c) Why are exactly 7 leap years needed in 19 years?

\vspace{1em}

\textbf{3. Antarctic Shabbat Problem} \\[0.5em]
You're at a research station at 80° South latitude in December when the sun doesn't set for weeks. \\
\emph{Challenge:} Compare different rabbinical solutions: (a) Using the last observable sunset time; (b) Following your home timezone; (c) Using the nearest community's sunset. Which approach do you think is most reasonable and why?

\vspace{1em}

\textbf{4. International Date Line Dilemma} \\[0.5em]
A flight departs Los Angeles at 9 PM on Thursday, September 14 (just before Rosh Hashanah) and arrives in Tokyo at 11 PM on Friday, September 15 local time, with a 12-hour flight duration. \\
\emph{Questions:} (a) When should the passenger start observing the holiday? (b) Compare different approaches: following departure time, arrival time, or continuous counting. (c) What are the practical implications of each choice?

\vspace{1em}

\textbf{5. The Ben Meir Controversy} \\[0.5em]
In 922 CE, Ben Meir calculated the molad as occurring at "Day 2, Hour 9, Part 204" while Saadia Gaon calculated "Day 2, Hour 15, Part 589." \\
\emph{Analysis:} (a) How many hours difference is this? (b) Why would this 6-hour difference affect which day Rosh Hashanah falls on? (c) Research which calculation modern astronomy supports and explain why this historical accuracy matters.

\vspace{1em}

\textbf{6. Two-Day Holidays: History vs. Modern Practice} \\[0.5em]
Two-day holiday observance outside Israel originated from uncertainty about calendar transmission. Modern communication has eliminated this uncertainty. \\
\emph{Discussion:} (a) Why do most diaspora communities still observe two days? (b) What does this suggest about the relationship between practical necessity and religious tradition? (c) How might this practice evolve with improving global communication technology?
} 