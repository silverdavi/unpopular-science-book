\begin{technical}
{\Large\textbf{Jewish Calendar: Mathematical Formulation}}\\[0.7em]

\textbf{Lunar Month and Molad Calculation}\\[0.5em]
The Jewish calendar is fundamentally lunar, with months beginning at the molad (new moon). The average length of a lunar month is:
$$
\text{Month} = 29 \text{ days}, 12 \text{ hours}, 44 \text{ minutes}, 3\frac{1}{3} \text{ seconds} = 29.530594 \text{ days}.
$$
This value, established by tradition, differs slightly from the modern astronomical value of 29.530589 days.

The molad for any month can be calculated using:
$$
M_n = M_0 + n \cdot \Delta M,
$$
where $M_0$ is a reference molad, $n$ is the number of months since the reference, and $\Delta M$ is the mean lunation interval.

\textbf{Metonic Cycle and Intercalation}\\[0.5em]
To align the lunar calendar with solar seasons, the Jewish calendar uses a 19-year cycle with 7 leap years:
$$
19 \text{ solar years} \approx 235 \text{ lunar months}
$$
More precisely: $19 \times 365.2422 = 6939.6018$ days, while $235 \times 29.530594 = 6939.6893$ days.

Leap years occur in years 3, 6, 8, 11, 14, 17, and 19 of each 19-year cycle. In leap years, the month of Adar is doubled (Adar I and Adar II).

\textbf{Dechiyot (Postponement Rules)}\\[0.5em]
Rosh Hashanah cannot fall on Sunday, Wednesday, or Friday, to prevent conflicts with Sabbath observance and Yom Kippur. The postponement rules (dechiyot) are:

\begin{enumerate}
\item \textbf{Lo ADU:} If the molad falls on Sunday, Wednesday, or Friday
\item \textbf{Molad Zaken:} If the molad occurs after 18:00 (6 PM)
\item \textbf{GaTRaD:} In regular years, if molad falls on Tuesday after 9:204 AM
\item \textbf{BeTuTaKPaT:} In years following leap years, if molad falls on Monday after 15:589 AM
\end{enumerate}

\textbf{Polar Region Solutions}\\[0.5em]
For locations with extended polar day/night, several halachic approaches exist:

\textbf{Fixed Time Method:} Use the last observable sunset/sunrise times:
$$
t_{\text{Shabbat}} = t_{\text{last sunset}} + 7 \times 24 \text{ hours}
$$

\textbf{Proportional Method:} Calculate theoretical solar position:
$$
h = 15° \times (t - 12) - \lambda + E
$$
where $h$ is solar hour angle, $t$ is local time, $\lambda$ is longitude, and $E$ is equation of time.

\textbf{Home Community Method:} Follow origin timezone:
$$
t_{\text{local}} = t_{\text{home}} + \Delta T_{\text{zone}}
$$

\textbf{International Date Line Calculations}\\[0.5em]
When crossing the date line during holidays, different approaches apply:

\textbf{Civil Date Method:} Follow local civil calendar
\textbf{Continuous Count Method:} Maintain unbroken day count regardless of date line
\textbf{Halachic Day Method:} Calculate based on sunset-to-sunset cycles

For flight crossing date line:
$$
t_{\text{holiday}} = \begin{cases}
t_{\text{departure}} + \text{flight time} & \text{(continuous method)} \\
t_{\text{local arrival}} & \text{(civil method)} \\
\text{next local sunset} & \text{(halachic method)}
\end{cases}
$$

\textbf{Calendar Accuracy and Drift}\\[0.5em]
The traditional molad interval accumulates error over time. The current drift is approximately:
$$
\text{Error} = (29.530594 - 29.530589) \times N = 5 \times 10^{-6} \times N \text{ days}
$$
where $N$ is the number of months since epoch.

After 1000 years: Error ≈ 7.3 hours
After 2000 years: Error ≈ 14.6 hours

\textbf{Ben Meir Controversy Mathematics}\\[0.5em]
The 922 CE dispute centered on molad timing. Ben Meir calculated:
$$
\text{Molad Tishrei} = \text{Day 2, Hour 9, Part 204}
$$
Saadia Gaon calculated:
$$
\text{Molad Tishrei} = \text{Day 2, Hour 15, Part 589}
$$
The difference of approximately 6 hours led to different postponement rule applications, resulting in Rosh Hashanah dates two days apart.

\textbf{Modern Computational Verification}\\[0.5em]
Using modern astronomical data, the actual new moon for Tishrei 4683 (922 CE) occurred at approximately:
$$
\text{JD} = 2071260.125 \pm 0.001
$$
corresponding to Day 2, Hour 15, Part 589 in the traditional system, supporting Saadia Gaon's calculation.

\vspace{0.5em}
\textbf{References:}\\
Reingold, E. M., Dershowitz, N. (2018). \textit{Calendrical Calculations}, 4th ed. Cambridge University Press.\\
Stern, S. (2012). \textit{Calendars in Antiquity}. Oxford University Press.\\
Bushwick, N. (1989). The Jewish calendar system. In \textit{Mapping Time}. John Wiley \& Sons.\\
Resnikoff, L. (1943). Jewish calendar calculations. \textit{Scripta Mathematica}, 9:191-195.
\end{technical}
