\begin{technical}
{\Large\textbf{Calendar Mathematics}}\\[0.5em]

\textbf{Molad Calculation}\\[0.3em]
The traditional Jewish lunar month length:
\[
29d\,12h\,44m\,3\tfrac{1}{3}s = 29.530594\text{ days}
\]
Modern astronomical value: 29.530589 days. Error accumulates at:
\[
\Delta = 5 \times 10^{-6} \times N \text{ days}
\]
where $N$ is months elapsed. After 1000 years ($\approx$12,400 months): error $\approx$ 2.6 hours.

\textbf{Metonic Cycle}\\[0.3em]
19 solar years $\approx$ 235 lunar months:
\[
19 \times 365.2422 = 6939.602\text{ days}
\]
\[
235 \times 29.530594 = 6939.689\text{ days}
\]
Difference: 0.087 days per 19-year cycle. Leap years occur in years 3, 6, 8, 11, 14, 17, 19.

\textbf{Dechiyot (Postponements)}\\[0.3em]
Rosh Hashanah cannot fall on Sun, Wed, or Fri:
\begin{enumerate}[leftmargin=*,topsep=0pt,itemsep=0pt]
\item \textbf{Lo ADU}: Direct postponement
\item \textbf{Molad Zaken}: If molad $\geq$ 18:00
\item \textbf{GaTRaD}: Regular year, Tuesday $\geq$ 9h 204p
\item \textbf{BeTuTaKPaT}: After leap, Monday $\geq$ 15h 589p
\end{enumerate}

\textbf{Ben Meir Dispute (922 CE)}\\[0.3em]
Ben Meir: Molad threshold = 642 parts\\
Traditional: Molad threshold = 0 parts\\
For Tishrei 4683 (922 CE):\\
Ben Meir: Day 2, 9h 204p\\
Saadia: Day 2, 15h 589p\\
Modern calculation confirms Saadia.

\textbf{Polar Day Solutions}\\[0.3em]
\textit{Sun-circle method}: Each 24h circuit = 1 day\\
\textit{Origin timezone}: Follow departure location\\
\textit{Proportional}: Calculate theoretical solar angle:
\[
h = 15°(t-12) - \lambda + E
\]

\textbf{Date Line Positions}\\[0.3em]
International: 180° from Greenwich\\
Chazon Ish: 90° E of Jerusalem (125.2°E)\\
Practical: Follow local civil date\\

\textbf{Orbital Period}\\[0.3em]
ISS orbit: 90 minutes\\
Sunrises per day: 16\\
Solution: Follow launch site timezone

\textbf{Classical Source}\\[0.3em]
\textit{Halakhot Pesuqot} (Rav Yehudai Gaon, 8th century):

\begin{hebrew}
לעולם ראש חדש אדר סמוך לניסן הוא ערב הפסח,
והפסח הוא ערב העצרת,
והעצרת הוא ערב ראש השנה.
לא בד"ו פסח,
לא גה"ז עצרת,
לא אד"ו ראש השנה וסוכה,
לא אג"ו יום הכיפורים,
ולא זבד פורים.
\end{hebrew}

\textit{Translation:}\\
Always, the new moon of Adar close to Nisan is the eve of Passover; Passover precedes Shavuot; and Shavuot precedes Rosh Hashanah. Passover never occurs on days בּד"ו (Sun, Wed, Fri); Shavuot never on גּה"ז (Tue, Thu, Sat); Rosh Hashanah and Sukkot never on אד"ו (Sun, Wed, Fri); Yom Kippur never on אג"ו (Sun, Tue, Fri); and Purim never on זבד (Mon, Wed, Sat).

\vspace{0.5em}
\textbf{References:}\\
{\footnotesize
Feldman, W.M. (1931). \textit{Rabbinical Mathematics}.\\
Stern, S. (2019). \textit{The Jewish Calendar Controversy of 921/2}.\\
Lifschitz, Y. (1850). \textit{Tiferet Yisrael}, Berakhot 1.
}
\end{technical}