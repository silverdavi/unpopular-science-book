\begin{SideNotePage}{

    \textbf{Top (Cultural Calendars and Celestial Cycles):}  
    Different civilizations devised calendars based on solar, lunar, or lunisolar cycles. The outer diagrams represent astronomical alignments and calendrical intercalation schemes. The moon–Earth–sun layout highlights the tension between observational cycles and constructed systems. \par
  
    \textbf{Second (The Oven of Aknai and Rabbinic Authority):}  
    In the Talmudic dispute over the purity of an oven, Rabbi Eliezer insists on a minority opinion, backed by signs from nature and heaven—including a river flowing backward and a divine voice. But the sages reject all evidence, asserting that law is not decided by miracles once given to humans. \par
  
    \textbf{Third (Halakhic Calendar and Defiance of Astronomy):}  
    Even when astronomical calculations proved the new moon hadn’t occurred, Rabbi Yehoshua obeys the Sanhedrin’s ruling and appears before Rabban Gamliel on the declared Yom Kippur, carrying a stick and wallet. Another famous schism was between Rabbi Saadia Gaon and Rabbi Aaron ben Meir over Rosh Hashanah dates. \par
  
    \textbf{Bottom (Shabbat Beyond Earth):}  
    In deep space, solar day cycles may be minutes or years. How do human-defined timeframes like Shabbat apply when detached from Earth's diurnal rhythm? \par
  
}{26_JewishCalendar/26_ When Midnight Means Nothing.pdf}
\end{SideNotePage}
