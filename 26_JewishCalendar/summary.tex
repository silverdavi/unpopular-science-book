Jewish law requires sunset to mark each new day, but what happens at the North Pole where the sun circles the horizon for months? The Talmud never contemplated Jews praying in 24-hour daylight or astronauts experiencing 16 sunrises daily. Rabbi Lifschitz in 1850s Danzig addressed whalers hunting near the pole: count each 24-hour sun-circle as one day. The International Space Station poses harder questions — which of those 16 sunsets counts for Sabbath? These edge cases explore what happens when ancient texts meet unprecedented geography, religious authority must choose between literal application and communal coherence. From the Talmud's "stick and wallet" showdown over calendar authority to modern rabbis merging technology with religious practice, each generation discovers that time depends as much on human consensus as celestial mechanics.