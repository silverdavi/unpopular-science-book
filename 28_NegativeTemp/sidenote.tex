\begin{SideNotePage}{
  \textbf{Top (Maximum Entropy – Overcrowded Room):}  
  A metaphor for a negative temperature system. The room is so disordered — crammed with objects, decor, and chaos — that any change would make it more ordered. This represents a population-inverted state: entropy is at its maximum, and additional energy leads to increased order. Thermodynamically, such states must have negative temperature. \par

  \textbf{Middle (Positive Temperature – Typical Disorder):}  
  A moderately cluttered room. It reflects a normal positive temperature system: there’s room for more disorder, and adding energy generally increases entropy. This is the typical regime described by classical statistical mechanics. \par

  \textbf{Bottom (Low Temperature – Ordered State):}  
  A clean, sparse room. This is a low-entropy state where most particles (or objects) are in low-energy configurations. Adding energy would increase disorder. The system is clearly in the conventional positive-temperature regime, but near the low end. \par

}{28_NegativeTemp/28_ Too Hot or Too Cold.pdf}
\end{SideNotePage}
