\begin{historical}
\textbf{Scaling Time: Accuracy and Element Size from Antiquity to the Atomic Nucleus}

\vspace{1em}
\begin{tabular}{p{4.5cm} c c l}
\textbf{Era / Technology} & \textbf{Accuracy} & \textbf{Size (m)} & \textbf{Time Reference} \\ \hline
Sundials (1800 BCE) & $10^{-2}$ & $10^1$ & Solar shadow on gnomon \\
Ancient Water Clocks & $10^{-3}$ & $10^{-1}$ & Liquid level change \\
Verge Clocks (13th c.) & $10^{-4}$ & $10^{-1}$ & Crown wheel / verge foliot \\
Pendulum Clocks (1656) & $10^{-5}$ & $10^0$ & Pendulum arc length \\
Marine Chronometer (18th c.) & $10^{-6}$ & $10^{-2}$ & Balance spring oscillator \\
Quartz Oscillators (1930s) & $10^{-8}$ & $10^{-3}$ & Crystal thickness (MHz mode) \\
Ammonia Maser (1949) & $10^{-9}$ & $10^{-10}$ & NH$_3$ inversion barrier \\
Cesium Beam Standard (1955) & $10^{-10}$ & $10^{-10}$ & $^{133}$Cs hyperfine structure \\
Hydrogen Maser (1960s) & $10^{-13}$ & $10^{-10}$ & $^{1}$H hyperfine structure \\
Rubidium Vapor & $10^{-11}$ & $10^{-10}$ & $^{87}$Rb hyperfine structure \\
Cesium Fountain (1990s) & $10^{-15}$ & $10^{-10}$ & Interference of free atoms \\
Optical Lattice (2010s) & $10^{-18}$ & $10^{-10}$ & Atomic dipole transitions \\
Projected Thorium Nuclear & $10^{-20}$ & $10^{-14}$ & Intrinsic nuclear excitation \\
\end{tabular}

\vspace{1em}
As clock elements shrink from meters to femtometers, accuracy improves from one part in $10^2$ to $10^{20}$. Modern clocks no longer rely on motion, but on invariant transitions within atoms and nuclei.
\end{historical}
