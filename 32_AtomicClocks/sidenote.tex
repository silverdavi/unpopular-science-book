\begin{SideNotePage}{
  \textbf{Top (Mechanical and Early Timekeeping):}  
  The image begins with sundials — using cast shadows to track solar position. Mechanical clocks follow: weight-driven escapements, pendulums, and balance springs. These systems introduced regular, countable oscillations, enabling precision independent of sunlight. \par

  \textbf{Middle (Electronic and Atomic Clocks):}  
  The transition to electronic timing brought quartz crystal oscillators, tuning forks, and circuit-based digital clocks. Atomic clocks, like the cesium beam and rubidium standards, use microwave transitions in atoms to define the second with extreme precision. These devices revolutionized navigation, telecommunications, and metrology. \par

  \textbf{Bottom (Next-Generation Nuclear Clocks):}  
  The current frontier involves optical lattice and nuclear clocks—probing energy levels in nuclei rather than electrons. These offer unprecedented temporal stability and resolution. If perfected, they could detect minute variations in gravity, test general relativity, and redefine time itself. \par
}{32_AtomicClocks/32_ Timing is Everything.pdf}
\end{SideNotePage}
