\fullpageexercises{%

\textbf{Fusible Numbers: Exercises in Constructive Time} \\[0.5em]

A famous riddle is given two candles, each one burns for one hour, how can one measure 45 minutes? \par

Fusible numbers form a well-ordered subset of rationals constructed iteratively from zero. A number \( z \) is fusible if there exist previously constructed fusible numbers \( x \) and \( y \), with \( |x - y| < 1 \), such that $z = (x + y + 1)/2$.
The construction corresponds to lighting a unit-time fuse at both ends with delay. The resulting set exhibits extraordinary growth rates unprovable in Peano arithmetic.

\vspace{1em}
\textbf{1. The Fuse Construction} \\
For a unit fuse lit at time \(x\) on one end and time \(y\) on the other (with $|x-y| < 1$), prove it extinguishes at $z = (x + y + 1)/2$. Consider the burn dynamics when both flames are active.

\vspace{1em}
\textbf{2. Enumeration Below 2} \\
Determine all fusible numbers less than 2 by systematic application of the construction rule.

\vspace{1em}
\textbf{3. Dyadic Structure} \\
Prove that all fusible numbers have form \( a/2^k \) for integers \( a \geq 0 \) and \( k \geq 0 \).

\vspace{1em}
\textbf{4. The Margin Function} \\
Let \( a_n \) be the smallest fusible number exceeding \( n \). Prove that
\[
a_n = n + \frac{1}{2^{k(n)}}
\]
for some \( k(n) \in \mathbb{N} \). Compute \( k(0) \), \( k(1) \), \( k(2) \). The function \( k(n) \) grows so rapidly that its behavior for large \( n \) is unprovable in Peano arithmetic.

\vspace{1em}
\textbf{5. Well-Ordering} \\
Prove that the fusible numbers form a well-ordered subset of \(\mathbb{Q}^+\). What does this imply about infinite decreasing sequences?

\vspace{2em}
\hrule
\vspace{1em}
\textbf{Context} \\
Fusible numbers (Erickson, Xu) demonstrate how elementary constructions yield incomprehensibly fast growth. The margin values are:
\[
a_0 = \frac{1}{2}, \quad a_1 = 1 + \frac{1}{8}, \quad a_2 = 2 + \frac{1}{1024}
\]
The margin function's growth rate connects to ordinal \( \varepsilon_0 \) and surpasses Graham's number for modest inputs. (See the chapter about big numbers)
}
