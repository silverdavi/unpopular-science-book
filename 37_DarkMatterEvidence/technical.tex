\begin{technical}
{\Large\textbf{Gravitational Inference and the Distribution of Dark Matter}}\\[0.2em]

\noindent\textbf{Introduction}\\[0.25em]
Dark matter is not detected directly but inferred through its gravitational influence. This section formalizes three principal lines of evidence: the internal dynamics of galaxy clusters, the rotation curves of spiral galaxies, and gravitational lensing. Each constrains the total mass distribution and reveals a significant mismatch between luminous matter and total gravitational mass.

\noindent\textbf{Virial Mass in Galaxy Clusters (recognize from the chapter on osmosis?)}\\[0.25em]
Clusters of galaxies are treated as self-gravitating systems in equilibrium. Let a system of \( N \) particles with masses \( m_i \) and velocities \( \mathbf{v}_i \) have total kinetic and potential energy:
\[
K = \sum_{i=1}^N \tfrac{1}{2} m_i v_i^2, \qquad
U = - \sum_{i<j} \frac{G m_i m_j}{r_{ij}}.
\]
By the virial theorem:
\[
2\langle K \rangle + \langle U \rangle = 0.
\]
Approximating \( \langle v^2 \rangle = \sigma_v^2 \) and \( U \approx - \tfrac{G M^2}{R} \), we obtain:
\[
M_{\text{vir}} \approx \frac{5 \sigma_v^2 R}{G},
\]
where \( R \) is the effective radius of the system. For rich clusters like Coma, the mass inferred by this formula exceeds luminous mass (stars + gas) by over an order of magnitude.

\noindent\textbf{Rotation Curves and Halo Profiles}\\[0.25em]
In spiral galaxies, stars and gas orbit the galactic center under gravitational attraction. For circular orbits:
\[
\frac{v^2(r)}{r} = \frac{G M(r)}{r^2} \quad \Rightarrow \quad M(r) = \frac{v^2(r)\,r}{G}.
\]
Observations show that \( v(r) \) remains nearly constant beyond the optical radius, implying \( M(r) \propto r \), inconsistent with the radial profile of visible mass.

This necessitates a dark matter halo extending beyond the visible disk. Simulations and fits to data often use the Navarro–Frenk–White (NFW) profile:
\[
\rho(r) = \frac{\rho_0}{(r/r_s)(1 + r/r_s)^2},
\]
where \( \rho_0 \) is a characteristic density and \( r_s \) a scale radius. This profile produces approximately flat rotation curves at large \( r \) and matches the mass distributions required to stabilize galaxies against dispersal.

\noindent\textbf{Weak Gravitational Lensing}\\[0.25em]
Lensing measures projected surface mass. In the weak lensing regime, the convergence \( \kappa(\theta) \) is given by:
\[
\kappa(\theta) = \frac{\Sigma(\theta)}{\Sigma_{\text{crit}}}, \quad
\Sigma_{\text{crit}} = \frac{c^2}{4\pi G} \cdot \frac{D_s}{D_d D_{ds}},
\]
with angular diameter distances \( D_s \) (observer to source), \( D_d \) (observer to lens), and \( D_{ds} \) (lens to source). The deflection angle is sensitive to the integrated surface mass density \( \Sigma(\theta) \). Mapping \( \kappa \) via background galaxy distortions reconstructs the total projected mass distribution.

In systems like the Bullet Cluster, lensing peaks and X-ray emission peaks are spatially offset. This implies that the dominant mass component is not collisional (as hot gas is), but rather behaves as a collisionless fluid, consistent with dark matter expectations.

\noindent\textbf{Implications}\\[0.25em]
The gravitational field inferred from cluster dynamics, orbital motion in galaxies, and lensing geometry all point to a dominant non-luminous mass component. Its spatial distribution is extended, centrally concentrated, and required across all scales. These observations define dark matter phenomenologically: a gravitationally interacting, non-emissive mass component that clumps and seeds structure.

\vspace{0.5em}
\noindent\textbf{References:}\\[0.25em]
Zwicky, F. (1933). Die Rotverschiebung von extragalaktischen Nebeln. \textit{Helv. Phys. Acta}, \textbf{6}, 110--127.\\
Clowe, D., Gonzalez, A., Markevitch, M. (2006). Weak-Lensing Mass Reconstruction of the Interacting Cluster 1E0657-558. \textit{ApJ}, \textbf{648}, L109.
\end{technical}
