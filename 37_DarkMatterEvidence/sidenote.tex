\begin{SideNotePage}{
  \textbf{Top (Candidate Properties):}  
  Dark matter candidates are evaluated by properties like mass (from ultralight to massive), speed (cold vs. warm), interaction type (weak, gravitational only), and detectability (via lab or decay signatures). Each row indicates a possible property constraint. \par

  \textbf{Second Row (WIMPs and Axions):}  
  WIMPs (Weakly Interacting Massive Particles) are cold, massive, and weakly interacting, long favored due to supersymmetric theories. Axions are extremely light, produced via field misalignment, and may convert to photons in magnetic fields. \par

  \textbf{Third Row (Sterile Neutrinos and Primordial Black Holes):}  
  Sterile neutrinos mix with active ones but don’t interact via the weak force, allowing them to evade detection. Primordial black holes are relics from the early universe that behave as cold dark matter through pure gravitational influence. \par

  \textbf{Bottom Row (Kaluza-Klein and MOND):}  
  Kaluza-Klein particles arise from extra spatial dimensions; their quantized modes could serve as dark matter if stable. Modified Newtonian Dynamics (MOND) suggests gravity itself needs correction at low accelerations—removing the need for particle dark matter. \par

}{37_DarkMatterEvidence/37_ The Invisible Scaffold.pdf}
\end{SideNotePage}
