Dark matter's existence is inferred through multiple independent lines of evidence spanning different cosmic scales. Galaxy rotation curves remain flat far beyond visible matter, indicating extended gravitational influence. Galaxy clusters contain hot gas whose temperature and confinement require gravitational potentials deeper than visible matter can provide. Gravitational lensing reveals mass distributions exceeding luminous components, particularly in systems like the Bullet Cluster where dark and visible matter separate during collisions. Cosmic web structure formation requires gravitational scaffolding predating visible matter to develop within observed timeframes. The cosmic microwave background's fluctuation patterns indicate that ordinary matter comprises only 15\% of the total matter content needed to match observations, with the remainder consisting of non-baryonic material already present before photon-matter decoupling.
