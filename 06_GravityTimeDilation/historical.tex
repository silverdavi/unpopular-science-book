\begin{historical}
Isaac Newton's 1687 \textit{Principia} described gravity as a force acting at a distance between masses, accurately predicting the behavior of falling objects and planetary orbits. This framework dominated physics for more than two centuries. Yet as astronomical measurements grew more precise, small but persistent anomalies emerged — most notably the unexplained excess precession of Mercury's orbit, which Newtonian mechanics could not account for.

In 1915, Albert Einstein introduced general relativity, fundamentally reframing gravity not as a force but as the curvature of spacetime. This revolutionary insight came from Einstein's realization that the equivalence of gravitational and inertial mass was no mere coincidence. His theory predicted phenomena beyond Newton's reach: time would flow differently in gravitational fields, and even light would bend when passing massive bodies.

The name "general" relativity reflects the theory's ambitious scope — it applies to all observers, whether accelerating, rotating, or in gravitational fields, generalizing his 1905 special relativity which was limited to inertial frames.

Experimental confirmation came in 1919, with Arthur Eddington's expedition to observe a solar eclipse. They found starlight bending around the Sun precisely as Einstein predicted — instantly transforming the German physicist into a global celebrity. Later experiments provided increasingly precise validation: the 1959 Pound–Rebka experiment detected gravitational redshift using gamma rays in a Harvard tower; Gravity Probe A (1976) launched a hydrogen maser clock on a rocket to confirm time dilation with altitude; and by 1980, ground-based cesium clocks could measure these effects with exquisite precision.

By the late 20th century, relativistic effects had become engineering concerns — GPS satellites must continuously correct for both special and general relativistic time shifts to maintain meter-level positioning accuracy.

The ultimate confirmation came a century after Einstein's publication. In 2015, LIGO detected gravitational waves from two black holes spiraling together over a billion light-years away. This observation, followed by dozens more including neutron star collisions, validated Einstein's theory in extreme gravitational regimes.

General relativity remains one of physics' most tested theories, validated from subatomic to cosmological scales.
\end{historical}
