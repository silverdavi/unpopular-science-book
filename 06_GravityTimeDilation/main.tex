The equivalence principle interprets gravity in geometric terms. It identifies the universality of free fall as a property of spacetime rather than a force between masses.

The weak equivalence principle states that all objects follow identical free-fall trajectories when released from the same point — whether lead, feathers, or antimatter. Formulated by Galileo and tested for centuries, it shows gravitational acceleration is independent of mass, charge, or composition. This universality implies gravitational (how strongly gravity pulls on an object) and inertial mass (the object's resistance to acceleration) are proportional — a coincidence in Newtonian physics that inspired Einstein to reinterpret gravity as motion through curved spacetime.

The strong equivalence principle generalizes this: in any local region of spacetime, gravitational effects can be eliminated by choosing a freely falling reference frame. Within such a frame, spacetime curvature becomes negligible. The geometry flattens to first order, and all physical processes proceed as they would in the absence of gravitation. Mechanical systems evolve according to Newton's laws, electromagnetic fields obey Maxwell's equations in vacuum, and the motion of particles is governed by the inertial character of special relativity.

While spacetime may be globally curved, it admits neighborhoods indistinguishable from flat Minkowski space. No experiment in a small free-falling laboratory can detect gravity.

Einstein's elevator thought experiment illustrates this: in a free-falling elevator, released objects float weightlessly and light travels straight. No experiment inside can detect the external gravitational field. Mechanical pendulums, electromagnetic resonators, radioactive decay rates, and interference patterns all behave as in gravity-free space. Free fall and inertial motion are locally identical when tidal effects are negligible. Over larger regions, deviations accumulate and curvature becomes detectable.

The spacetime metric is the mathematical object that defines how distances and time intervals are measured. It specifies the separation between nearby events.

In flat spacetime, the metric is constant: \(ds^2 = -c^2 dt^2 + dx^2 + dy^2 + dz^2\). Time gets a negative coefficient (-c²) while space coordinates get positive coefficients, meaning the shortest distance between two points is when space is minimized and time is maximized. In curved spacetime where gravity is present, the metric tensor \(g_{\mu\nu}\) varies from point to point. This variation determines how clocks tick at different locations and how distances are measured — it is what we experience as gravity.

These variations manifest in observable ways. Clocks at different gravitational potentials accumulate time at different rates. Initially parallel free-falling trajectories converge or diverge. Tidal effects — variations in acceleration across extended objects — reveal that spacetime is not flat. When exchanging signals between different altitudes, identical clocks emit pulses at regular intervals but receivers measure changed intervals: upward light is redshifted, downward light is blueshifted, reflecting different time flow rates at each location.

A local inertial frame has no proper acceleration and follows special-relativistic laws — light travels straight and clocks tick uniformly. The metric matches flat spacetime at a point, with deviations appearing only at second order in displacement. But when comparing such frames at different locations, discrepancies emerge: clocks at different altitudes tick at different rates, transported vectors fail to align, and no single coordinate system makes the metric flat everywhere. The curvature is intrinsic, encoded in the metric's second derivatives.

Spacetime curvature reveals itself when comparing measurements across finite regions. Free-falling particles initially at rest maintain separation in flat spacetime but drift apart or together in curved geometry. This geodesic deviation reflects how curvature alters nearby inertial paths, with the rate determined by the local curvature tensor.

The metric captures spacetime geometry, determining clock rates, length measurements, and free motion. In flat spacetime it's constant; in curved spacetime it varies with position, producing gravitational phenomena.

A helpful analogy comes from curved surfaces. Imagine two travelers walking north from the equator along different lines of longitude. Their paths begin parallel and appear straight locally, yet they eventually converge at the pole. The meeting point is not due to any force between them but to the surface’s geometry. In spacetime, freely falling objects can similarly start with zero relative velocity and later converge or diverge, not because of an interaction, but because the spacetime metric changes from point to point. The curvature of the manifold governs how local inertial paths extend and how global structure emerges from local geometry.

The metric component \( g_{00} \) controls proper time for stationary observers. When \( g_{00} \) varies with location, the same coordinate time corresponds to different proper times — gravitational time dilation. While spatial curvature affects lensing and tides, the gradient of \( g_{00} \) determines local clock rates and free fall direction. Gravitational acceleration comes from the variation in time's passage.

A geodesic is the path of a freely moving object through spacetime — a straight line in special relativity, but in general relativity the trajectory that extremizes proper time. The metric defines these generalized straight paths. In a gravitational field, geodesics curve in space. Near Earth, for example, an object released from rest accelerates downward because proper time accumulates more slowly near the surface. The object curves toward regions where time flows more slowly. The acceleration we observe is the projection of this curved spacetime path into spatial coordinates.

Free-falling objects follow geodesics, moving in directions that maximize proper time. Orbits, falling, and light deflection all obey this rule. The metric's temporal and spatial components together determine free motion.

That clocks at different altitudes tick at different rates is a direct consequence of the metric, specifically of the spatial variation in \( g_{00} \). A fixed interval of coordinate time corresponds to more or less proper time depending on position. This difference has been confirmed using atomic clocks aboard airplanes, satellites, and mountaintops, and it is essential for GPS accuracy.

This same variation governs motion. A freely falling object always moves toward regions of slower time. Consider an apple hanging from a tree. While attached, it resists the geodesic assigned to its position. Once the stem breaks, it is no longer constrained. It enters inertial motion, and its path curves downward because proper time accumulates more slowly below.

As the apple descends, the gradient of proper time changes continuously. The trajectory adjusts to follow the direction where time advances least quickly. The resulting curvature in its path is what we perceive as acceleration. This acceleration is not caused by any external agent. It is not carried by particles, nor transmitted through a field in the usual sense. The motion arises from the metric itself. Spacetime tells the apple how to move by encoding how time flows from point to point.

Spatial curvature affects lensing and tides, but falling is temporal. Non-uniform time flow determines motion direction. Gravity is the effect of unequal clock rates in spacetime geometry.


\begin{commentary}[Falling Into Slower Time]

The realization that gravity arises from differences in the rate of time’s passage, rather than from any applied force, marks a sharp shift in understanding. This perspective is mathematically well-defined and supported by high-precision experiments, but it remains difficult to visualize in intuitive terms.

The rubber sheet analogy shows mass distorting a surface, with objects curving toward the indentation. While this conveys how mass alters geometry, it emphasizes spatial curvature. In general relativity, gravitational motion comes primarily from varying proper time — time flows more slowly deeper in gravitational fields. This temporal gradient, not spatial curvature, governs free fall.

This is, for me, a \textit{fantastic} observation. An apple falls from a tree not because it is pulled downward, but because time flows slightly faster at the top of the tree than at the bottom. Once released, the apple is no longer constrained. It follows a trajectory through spacetime that maximizes the amount of proper time experienced along the way. The curve of this path is determined by how the clock rate changes with altitude.

In everyday reasoning, we often think of objects as taking the shortest route through space. But in general relativity, freely falling objects follow the most direct path through spacetime as a whole, which is about minimizing spatial distance while maximizing proper time, until even the distinction between space and time becomes irrelevant (as in the vicinity of a black hole which we will discuss in another chapter).

\end{commentary}

