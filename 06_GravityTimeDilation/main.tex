
The weak equivalence principle states that all objects follow identical free-fall trajectories when released from the same point — whether lead, feathers, or antimatter. Formulated by Galileo and tested for centuries, it shows gravitational acceleration is independent of mass, charge, or composition. This universality implies gravitational (how strongly gravity pulls on an object) and inertial mass (the object's resistance to acceleration) are proportional — a coincidence in Newtonian physics that inspired Einstein to reinterpret gravity geometrically, as motion through curved spacetime.

The strong equivalence principle generalizes this: in any local region of spacetime, gravitational effects can be eliminated by choosing a freely falling reference frame. Within such a frame, spacetime curvature becomes negligible. The geometry flattens to first order, and all physical processes proceed as they would in the absence of gravitation. Mechanical systems evolve according to Newton's laws, electromagnetic fields obey Maxwell's equations in vacuum, and the motion of particles is governed by the inertial character of special relativity.

While spacetime may be globally curved, it admits neighborhoods indistinguishable from flat Minkowski space. No experiment in a small free-falling laboratory can detect gravity.

Einstein's elevator thought experiment illustrates this: in a free-falling elevator, released objects float weightlessly and light travels straight. No experiment inside can detect the external gravitational field. Mechanical pendulums, electromagnetic resonators, and radioactive decay rates, all behave as in gravity-free space. Free fall and inertial motion are locally identical when tidal effects are negligible — that is, when the variation in gravitational field strength across the laboratory is too small to measure. Over larger regions, these tidal effects become detectable as objects at different positions experience slightly different accelerations, causing initially parallel trajectories to converge or diverge.

The spacetime metric is the mathematical object that defines how distances and time intervals are measured. It quantifies the separation between nearby events.

In flat spacetime, the metric is constant: \(ds^2 = -c^2 dt^2 + dx^2 + dy^2 + dz^2\). Time gets a negative coefficient (-c²) while space coordinates get positive coefficients, meaning the shortest distance between two points is when space separation is minimized and time separation is maximized. In curved spacetime where gravity is present, the metric tensor \(g_{\mu\nu}\) varies from point to point. This variation determines how clocks tick at different locations and how distances are measured — it is what we experience as gravity.

These variations manifest in observable ways. Clocks at different gravitational potentials accumulate time at different rates. Initially parallel free-falling trajectories converge or diverge. Tidal effects — the differential forces that stretch objects toward massive bodies and compress them perpendicular to that direction — arise because gravitational field strength varies with position. On Earth, these variations cause ocean tides as the Moon pulls more strongly on the near side than the far side. In extreme cases near black holes, tidal forces can tear objects apart or spaghettify them. When exchanging signals between different altitudes, identical clocks emit pulses at regular intervals but receivers measure changed intervals: upward light is redshifted, downward light is blueshifted.

A local inertial frame has no proper acceleration and follows special-relativistic laws — light travels straight and clocks tick uniformly. The metric matches flat spacetime at a point, with deviations appearing only at second order in displacement. But when comparing such frames at different locations, clocks at different altitudes tick at different rates, transported vectors fail to align, and no single coordinate system makes the metric flat everywhere. The curvature is encoded in the metric's second derivatives.

A helpful analogy comes from curved surfaces. Imagine two travelers walking north from the equator along different lines of longitude. Their paths begin parallel and appear straight locally, yet they eventually converge at the pole. The meeting point is not due to any force between them but to the surface’s geometry. In spacetime, freely falling objects can similarly start with zero relative velocity and later converge or diverge, not because of an interaction, but because the spacetime metric changes from point to point.

The metric component \( g_{00} \) encodes how proper time flows for stationary observers. When \( g_{00} \) varies with position, identical coordinate intervals correspond to different amounts of proper time — this is gravitational time dilation. Atomic clocks aboard airplanes, satellites, and mountaintops have confirmed these predictions with precision essential for GPS accuracy. While spatial curvature produces phenomena like gravitational lensing and tidal forces, the gradient of \( g_{00} \) determines both the rate at which clocks tick and the direction objects fall.

This connection between time and motion is the essence of gravity. In general relativity, freely falling objects follow geodesics — paths that extremize proper time through spacetime. Near Earth's surface, these geodesics curve downward in space precisely because proper time accumulates more slowly at lower altitudes. The "force" we attribute to gravity is actually the spatial projection of motion along these curved spacetime paths.

Consider an apple hanging from a tree. While attached to the branch, it resists its natural geodesic. Once the stem breaks, the apple enters free fall — not pulled by any force, but following the path of maximum proper time. As it descends, the changing gradient of \( g_{00} \) continuously adjusts its trajectory toward regions where time passes more slowly. This curvature in the apple's spacetime path manifests as what we perceive as gravitational acceleration. The metric itself, encoding how time flows differently at each point in space, tells matter how to move.

Spatial curvature affects lensing and tides, but falling is temporal. Non-uniform time flow determines motion direction. Gravity is the effect of unequal clock rates in spacetime geometry.

\begin{commentary}[Falling Into Slower Time]

The realization that gravity arises from differences in the rate of time’s passage, rather than from any applied force, marks a sharp shift in our description of nature. This perspective is mathematically well-defined and supported by high-precision experiments, but it remains difficult to visualize and understand intuitively.

The rubber sheet analogy shows mass distorting a surface, with objects curving toward the indentation. While this conveys how mass alters geometry, it emphasizes spatial curvature. In general relativity, gravitational motion comes primarily from varying proper time — time flows more slowly deeper in gravitational fields. This temporal gradient, not spatial curvature, governs free fall.

This is, for me, a \textit{fantastic} observation. An apple falls from a tree not because it is pulled downward, but because time flows slightly faster at the top of the tree than at the bottom. Once released, the apple is no longer constrained. It follows a trajectory through spacetime that maximizes the amount of proper time experienced along the way. The curve of this path is determined by how the clock rate changes with altitude.

In everyday reasoning, we often think of objects as taking the shortest route through space. But in general relativity, freely falling objects follow the most direct path through spacetime as a whole, which is about minimizing spatial distance while maximizing proper time, until even the distinction between space and time becomes irrelevant (as in the vicinity of a black hole which we will discuss in another chapter).

\end{commentary}