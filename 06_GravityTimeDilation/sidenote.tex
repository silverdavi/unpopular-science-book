\begin{SideNotePage}{
  \textbf{Top (GR — Falling Toward Slow Time):} An apple follows a curved free-fall trajectory toward Earth. Along the path, it is shown accompanied by clocks, which tick progressively slower as they get closer to Earth. This is an insight relativity: free-falling objects move toward regions where proper time is maximized — in other words, where the time component of the metric, $g_{00}$, decreases. The apple is not being pulled by a force, but slides through spacetime toward slower time.

  \textbf{Middle (Misleading Space Bending):} Two “rubber sheet” grid diagrams are shown. The top depicts an exaggerated funnel-shaped distortion caused by a black mass — a common pop-science depiction of gravity. The bottom shows the Earth resting on a seemingly flat grid, emphasizing that for small masses like Earth, spatial curvature is negligible. What matters is the distortion of time, not space. Gravity in this regime arises almost exclusively from the $g_{00}$ gradient.

  \textbf{Bottom (Aristotle to Newton):} Left: Aristotle’s model shows apples falling back to Earth because they “belong” there — matter returning to its natural place. Right: Newton’s model depicts a single apple being pulled downward by an invisible gravitational force, with spring lines suggesting mutual attraction between masses. This is Newton’s universal law of gravitation — a force acting at a distance.
}{06_GravityTimeDilation/06_ The Apple Falls the Shortest Path from the Tree.pdf}
\end{SideNotePage}