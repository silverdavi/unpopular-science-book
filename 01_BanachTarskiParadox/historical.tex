\begin{historical}
In the second half of the 19th century, questions about the foundations of analysis led mathematicians to examine the basic assumptions underlying number, function, and space. Dedekind formalized the real numbers via cuts, while Weierstrass removed appeals to geometric intuition from calculus. At the same time, mathematicians encountered pathologies — such as functions continuous everywhere but differentiable nowhere, or nowhere-dense sets of full measure — that challenged classical notions of size and shape. These developments revealed that intuitive notions of length, area, and convergence required formal clarification, especially in the context of infinite processes.

In the late 19th century, Georg Cantor changed mathematics with his work on infinite sets, laying the groundwork for new perspectives on measure and cardinalities. In 1905, Giuseppe Vitali introduced the first example of a non-measurable set, suggesting that some subsets of \( \mathbb{R}^n \) cannot be assigned an intuitive notion of size. Building on this foundation, Felix Hausdorff presented a paradox in 1914, showing that a sphere could be decomposed in a way that hinted at even more surprising outcomes.

A decade later, in 1924, Stefan Banach and Alfred Tarski formulated it further in a concrete result, known as the Banach–Tarski paradox. They demonstrated that a solid ball in three-dimensional space could be split into a finite number of pieces and then reassembled, through rigid motions, into two full copies of the original in a process that requires the Axiom of Choice, introduced two decades prior and still a subject of research. Though it does not apply to physical objects, the Banach–Tarski paradox remains a powerful example of how set-theoretic assumptions can lead to unexpected and extraordinary results in geometry.
\end{historical}
