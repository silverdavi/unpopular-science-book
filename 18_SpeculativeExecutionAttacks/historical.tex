\begin{historical}
The core techniques behind speculative execution — pipelining, branch prediction, and out-of-order execution — originated as performance optimizations in the 1960s and matured across RISC and superscalar designs in the 1980s–1990s. The IBM System/360 Model 91 (1966) introduced dynamic scheduling. Tomasulo's algorithm and register renaming were foundational. By the 2000s, speculative execution had become ubiquitous in high-performance processors.

Meanwhile, side-channel attacks emerged independently in cryptography. Kocher (1996) demonstrated timing attacks on modular exponentiation. Cache timing attacks followed, with Bernstein (2005) showing cache-based AES key recovery. Yet until 2018, these methods were largely seen as requiring deliberate software flaws or special setups.

The Spectre and Meltdown disclosures reframed this landscape. They showed that speculation itself — once considered internal and safe — could be manipulated into violating memory isolation. The result was a universal class of vulnerabilities, rooted not in incorrect logic, but in correct optimization.
\end{historical}
