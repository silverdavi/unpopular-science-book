The Sun produces energy through nuclear fusion in its core. Gravitational compression generates densities exceeding $150\,\text{g}/\text{cm}^3$ and temperatures near $1.5 \times 10^7\,\text{K}$. At these conditions, hydrogen nuclei fuse into helium, releasing binding energy.

In the simplest view, fusion proceeds through direct collisions of hydrogen nuclei. At core temperatures of order $10^7\,\text{K}$, protons possess sufficient thermal kinetic energy to approach closely enough for the strong nuclear force to bind them. Through a sequence of interactions known as the proton–proton chain, four protons are ultimately transformed into a helium nucleus. 

Each fusion event in the Sun's core releases a small amount of energy: approximately $26.7\,\text{MeV}$ per helium nucleus formed. However, the Sun generates a total power output of roughly $3.8 \times 10^{26}\,\text{W}$, which requires converting mass to energy at an enormous rate. By the relation $E = mc^2$, this luminosity implies a mass loss of about $4.3 \times 10^9\,\text{kg}$ per second.

This mass loss manifests as outward radiation pressure. Within the core, energy liberated by fusion builds up pressure that counteracts gravitational collapse. The resulting hydrostatic equilibrium maintains the Sun's structure — every second, the immense weight of the Sun's outer layers is balanced by pressure generated from fusing approximately $6 \times 10^{11}$ kilograms of hydrogen into helium. The Sun's long-term stability emerges from this balance. Fusion sustains the outward force needed to resist the crushing pull of its own mass.

The energy generated in the core undergoes radiative diffusion. Photons produced during fusion scatter trillions up to trillions of times off electrons and nuclei as they migrate outward. In the outer layers, convective transport becomes dominant, with rising and sinking plasma transporting energy. After this migration, energy is finally emitted from the photosphere as sunlight, spanning a broad electromagnetic spectrum.

Conservation of energy, momentum, and electric charge ensures consistency in nuclear reactions. Quantum field theories of particle interactions also impose another conserved quantity: lepton number. Leptons — a class of particles including electrons, neutrinos, and their antiparticles — must be created or destroyed in such a way that the total lepton number remains unchanged.

The proton–proton chain, which powers the Sun, involves changes in particle types that require mechanisms beyond the electromagnetic and strong forces. In particular, the weak nuclear force is necessary to enable the conversion of protons into neutrons while preserving all conservation laws. The weak force enables the fusion of hydrogen into helium.

Here is the first step of the chain:
\[
\text{p} + \text{p} \;\to\; \text{d} + e^+ + \nu_e,
\]

where $\text{p}$ denotes a proton, $\text{d}$ a deuteron (a bound state of one proton and one neutron), $e^+$ a positron, and $\nu_e$ an electron neutrino. In this reaction, one proton transforms into a neutron through a weak interaction. To conserve electric charge, a positron — the antimatter counterpart of the electron — is emitted. To conserve lepton number, an electron neutrino is emitted simultaneously. 

In the lepton number accounting, electrons and neutrinos are assigned a lepton number of $+1$, while positrons and antineutrinos carry a lepton number of $-1$. Before the reaction, the system has zero net lepton number; after the reaction, the positron ($-1$) and neutrino ($+1$) balance each other, maintaining overall neutrality. The emission of the neutrino is therefore a necessity for the reaction to be consistent with the symmetries of particle physics.

Although neutrinos possess extremely small mass and interact only via the weak force, they carry away a significant fraction of the reaction's energy and linear momentum. Unlike photons — which scatter thousands of times before reaching the solar surface — neutrinos traverse the Sun's dense interior with minimal interaction and escape into space almost immediately. Neutrinos produced in the Sun's core reach Earth in about 8 minutes, providing a direct and real-time probe of nuclear processes inside the Sun.

The detection of solar neutrinos has been crucial for confirming theoretical models of stellar energy generation. Measurements not only validate the dominance of the proton–proton chain but also reveal minor contributions from alternative fusion pathways, such as the carbon–nitrogen–oxygen (CNO) cycle in which carbon, nitrogen, and oxygen nuclei fuse to produce helium.

Quantum mechanics introduces behaviors absent in classical physics. One of these is tunneling: the ability of a particle to penetrate and traverse a potential barrier even when its total energy is insufficient to overcome it. 

Classically, a particle with energy less than the height of a potential barrier would be fully reflected, with zero probability of passage. In quantum mechanics, however, particles are described by continuous wavefunctions governed by the Schrödinger equation. Even in classically forbidden regions, the wavefunction persists, decaying exponentially rather than vanishing abruptly.

When a quantum particle encounters a barrier higher than its energy, its wavefunction inside the barrier takes the form of a decaying exponential. If the barrier has finite width, there exists a nonzero probability that the particle will appear beyond the barrier — a phenomenon known as quantum tunneling.

In the solar core, the fusion of protons faces a major obstacle: the Coulomb barrier arising from electrostatic repulsion when the protons are close enough to trigger the strong nuclear force. The potential energy associated with two protons at close approach is approximately $1\,\text{MeV}$, whereas the typical thermal kinetic energy at $1.5 \times 10^7\,\text{K}$ is about $1\,\text{keV}$. Classically, the probability of overcoming the barrier would be vanishingly small, and fusion would be effectively impossible.

Despite this, fusion proceeds because quantum tunneling allows protons to penetrate the Coulomb barrier with nonzero probability. Quantum mechanics enables fusion at energies far below the classical threshold. The proton wavefunctions extend into and through the classically forbidden region, resulting in occasional barrier penetration and subsequent nuclear fusion.

The probability of tunneling through the Coulomb barrier is quantified by the Gamow factor. This factor arises from solving the Schrödinger equation for two charged particles and introduces an exponential suppression depending on the product of the charges, the reduced mass of the system, and the relative kinetic energy. The tunneling probability scales as
\[
P \sim \exp\left( -b\sqrt{E} \right),
\]
where \( b \) is a constant depending on fundamental parameters such as the fine-structure constant and the masses and charges involved.

At stellar core temperatures, the Gamow factor dominates the fusion reaction rate. Although tunneling remains rare per collision, the immense number of protons ensures sufficient fusion events to sustain the Sun's energy output. The exponential sensitivity of tunneling probability to temperature creates a self-regulating system: if fusion falls below the rate needed to balance gravitational compression, contraction increases core temperature until equilibrium restores; if fusion runs too high, expansion cools the core and reduces the reaction rate. This feedback mechanism maintains stable stellar burning within a narrow band of core conditions.

This regulatory mechanism underlies the main sequence which is the phase during which hydrogen fusion occurs steadily in the core. A star remains on the main sequence while hydrogen supply sustains the equilibrium fusion rate. The phase lifetime depends on stellar mass, which sets both compression rate and required temperature. For the Sun, this balance produces stability lasting approximately $10^{10}$ years.

The Sun's luminosity remains constant through stable interaction between gravity, fusion kinetics, and quantum tunneling probabilities. These parameters determine the mass-to-energy conversion rate. The resulting energy supports overlying layers without expansion or collapse.

Solar neutrinos arise when the weak force converts a proton's up quark into a down quark during fusion. This process must conserve not only the lepton number we discussed above, but also fermion number: creating a neutron would violate conservation alone, so the reaction simultaneously produces a positron (fermion number $-1$) and an electron neutrino (fermion number $+1$).

The Sun produces approximately $2 \times 10^{38}$ neutrinos per second, carrying $2\%$ of fusion energy. With interaction cross-sections of $10^{-44}\,\text{cm}^2$, they pass through matter nearly unimpeded — while photons require thousands of years to diffuse through the Sun, neutrinos escape instantaneously, reaching Earth in 8 minutes.

Every detected neutrino was produced moments earlier in the solar core. Measuring their flux and energy spectrum tests stellar energy generation models with high precision.

When physicists first detected solar neutrinos in the 1960s, they encountered a puzzle. Raymond Davis Jr.'s Homestake experiment used 400,000 liters of perchloroethylene to capture neutrinos through the reaction:
\[
\nu_e + \text{Cl}^{37} \;\to\; e^- + \text{Ar}^{37}.
\]

The Homestake detector measured only about one-third of the neutrino flux predicted by standard solar models. This deficit, known as the solar neutrino problem, persisted for over three decades despite improved experiments and refinements to stellar theory.

The resolution came through discovering neutrino oscillations — neutrinos transform between different flavors as they propagate. The Standard Model lists three flavors: electron ($\nu_e$), muon ($\nu_\mu$), and tau ($\nu_\tau$) neutrinos. Solar fusion produces only electron neutrinos, but oscillations into other flavors during travel to Earth explain why early detectors registered a deficit.

The Sudbury Neutrino Observatory (SNO), 2 kilometers underground in Ontario, used heavy water to measure both total neutrino flux and electron neutrino flux. SNO's 2001 results confirmed the total flux matched predictions, but two-thirds of electron neutrinos had oscillated into other flavors en route to Earth.

Neutrino oscillations require nonzero mass. The original Standard Model assumed massless neutrinos, so oscillations constitute evidence for physics beyond it. Current measurements indicate neutrino masses are less than a few tenths of an electron volt — over a million times smaller than the electron mass.

This discovery resolved the solar neutrino problem and validated both solar fusion theory and quantum field theory. The neutrino flux matches predictions from nuclear burning models. Oscillations opened new physics avenues, including CP violation studies and implications for the universe's matter-antimatter asymmetry.

While neutrinos probe nuclear processes directly, helioseismology — the study of solar oscillations — maps conditions throughout the solar interior.

The Sun undergoes acoustic oscillations driven by outer-layer convection. These pressure waves propagate through the interior like seismic waves through Earth. Oscillation frequencies, amplitudes, and patterns depend on internal temperature, density, and composition profiles.

Solar oscillations appear as periodic Doppler shifts in photospheric absorption lines. GONG and SOHO monitor these oscillations continuously. Millions of distinct modes have been identified, each with characteristic radial and angular patterns.

Analyzing oscillation mode frequencies reveals the Sun's interior structure — a three-dimensional map of temperature, density, and rotation rate versus depth and latitude.

Helioseismology confirms stellar model predictions with high accuracy. Temperature profiles match theory within $0.1\%$ throughout most of the interior. The convective zone depth measures $0.287$ solar radii from the surface, matching theoretical predictions. It also validates density and temperature profiles used for neutrino predictions. The inferred central temperature of $(1.57 \pm 0.01) \times 10^7\,\text{K}$ confirms conditions for the observed proton-proton chain rate. This independent confirmation strengthens confidence in stellar evolution theory and solar nuclear processes.