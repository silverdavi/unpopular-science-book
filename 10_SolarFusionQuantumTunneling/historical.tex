\begin{historical}
In the late 19th century, Lord Kelvin and Hermann von Helmholtz proposed that gravitational contraction powered the Sun, but this mechanism accounted for only tens of millions of years — far shorter than the timescales implied by geological and biological evidence on Earth. In the early 20th century, Arthur Eddington rejected this view, positing that nuclear processes must fuel the Sun’s enduring luminosity.

In the 1920s, George Gamow introduced quantum mechanics into stellar models, showing that charged particles could penetrate electrostatic barriers via quantum tunneling. Around the same time, Robert Atkinson, Fritz Houtermans, and Ralph Fowler explored how fusion might occur at stellar temperatures, providing theoretical support for nuclear reactions in stars.

Hans Bethe’s 1939 work clarified these mechanisms, describing both the proton–proton chain and the carbon–nitrogen–oxygen (CNO) cycle. This established the theoretical basis for stellar fusion in different stellar environments. Later decades brought confirmation through solar neutrino detection and improved nuclear cross-section measurements. These developments cemented the view that the mechanism of quantum tunneling — initially a purely theoretical construct — directly powers the Sun and shapes the broader evolution of stars.
\end{historical}