\begin{technical}
{\Large\textbf{DNA Sequencing Technologies: Technical Formulation}}\\[0.7em]

\textbf{Sanger Chain Termination Method}\\[0.5em]
Sanger sequencing uses dideoxynucleotides (ddNTPs) that lack the 3'-OH group required for phosphodiester bond formation:
$$
\text{dNTP: } \text{Base}-\text{Sugar}(\text{3'-OH, 5'-PO}_4) \rightarrow \text{Extension possible}
$$
$$
\text{ddNTP: } \text{Base}-\text{Sugar}(\text{3'-H, 5'-PO}_4) \rightarrow \text{Termination}
$$

The probability of termination at position $n$ for ddNTP concentration ratio $r$ is:
$$
P_{\text{term}}(n) = \frac{r}{1+r}
$$
where $r = [\text{ddNTP}]/[\text{dNTP}]$. Optimal ratios typically range from 1:50 to 1:200.

\textbf{Next-Generation Sequencing Platforms}\\[0.5em]

\textbf{454 Pyrosequencing:} Detects pyrophosphate release during DNA synthesis:
$$
\text{dNTP} + \text{DNA}_n \xrightarrow{\text{polymerase}} \text{DNA}_{n+1} + \text{PPi}
$$
$$
\text{PPi} + \text{APS} \xrightarrow{\text{sulfurylase}} \text{ATP} + \text{SO}_4^{2-}
$$
$$
\text{ATP} + \text{luciferin} \xrightarrow{\text{luciferase}} \text{light} + \text{AMP} + \text{PPi}
$$

Light intensity is proportional to number of incorporated nucleotides: $I \propto n_{\text{bases}}$.

\textbf{Ion Torrent:} Measures pH changes from proton release:
$$
\text{dNTP incorporation} \rightarrow \text{H}^+ \text{ release} \rightarrow \Delta\text{pH} \rightarrow \Delta V_{\text{ISFET}}
$$
ISFET voltage response: $\Delta V = \frac{kT}{q} \ln(10) \times \Delta\text{pH}$ where $k$ is Boltzmann constant, $T$ is temperature, $q$ is elementary charge.

\textbf{Illumina Sequencing:} Uses reversible terminators with fluorescent labels:
$$
\text{Terminator} \xrightarrow{\text{incorporation}} \text{Imaging} \xrightarrow{\text{cleavage}} \text{Next cycle}
$$
Each cycle produces 4-color fluorescence image with signal-to-noise ratio typically >10:1.

\textbf{Third-Generation Sequencing}\\[0.5em]

\textbf{Pacific Biosciences (PacBio):} Single-molecule real-time sequencing in zero-mode waveguides (ZMWs):
$$
\text{ZMW diameter} \approx 100\text{nm} < \lambda_{\text{excitation}}/2n
$$
where $\lambda$ is excitation wavelength and $n$ is refractive index. This confines illumination to ~20 zeptoliters.

Pulse width analysis determines base identity:
$$
t_{\text{pulse}} = t_{\text{incorporation}} + t_{\text{diffusion}}
$$
Different bases show characteristic pulse duration distributions.

\textbf{Oxford Nanopore:} Measures current modulation as DNA translocates through protein pores:
$$
I(t) = f(\text{5-mer sequence at time } t)
$$
Current levels are sequence-dependent with ~20 pA differences between bases. Base calling uses hidden Markov models or recurrent neural networks.

\textbf{Genome Assembly Algorithms}\\[0.5em]

\textbf{De Bruijn Graph Construction:} For reads $R$ and k-mer length $k$:
$$
G = (V, E) \text{ where } V = \{k\text{-mers}\}, E = \{(u,v) : \text{suffix}_{k-1}(u) = \text{prefix}_{k-1}(v)\}
$$

\textbf{Eulerian Path Assembly:} Seek path visiting each edge exactly once:
$$
\text{Assembly} = \text{sequence corresponding to Eulerian path in } G
$$

For ideal case with perfect coverage and no repeats:
$$
|\text{Genome}| = |E| + k - 1
$$

\textbf{Coverage and Assembly Statistics}\\[0.5em]
Expected coverage follows Poisson distribution:
$$
P(\text{coverage} = c) = \frac{\lambda^c e^{-\lambda}}{c!}
$$
where $\lambda = \frac{N \times L}{G}$ (reads × read length / genome size).

Probability of zero coverage: $P(c = 0) = e^{-\lambda}$

For complete assembly with probability $p$: $\lambda \geq -\ln(1-p)$

N50 statistic: Length $L$ such that 50\% of assembly is in contigs ≥ $L$.

\textbf{Paired-End Constraints}\\[0.5em]
Insert size distribution typically normal: $\text{Insert} \sim \mathcal{N}(\mu, \sigma^2)$

For scaffolding, gap size estimation:
$$
\text{Gap} = \text{Insert size} - \text{Read}_1 - \text{Read}_2 - \text{Overlap}
$$

Mate-pair validation: $|\text{Observed distance} - \text{Expected}| < 3\sigma$

\textbf{Error Models}\\[0.5em]
Illumina error rate: $\varepsilon \approx 0.1\%$ per base
PacBio error rate: $\varepsilon \approx 10-15\%$ per base (mostly indels)
Nanopore error rate: $\varepsilon \approx 5-10\%$ per base

Quality scores: $Q = -10\log_{10}(P_{\text{error}})$
$$
P_{\text{error}} = 10^{-Q/10}
$$

\vspace{0.5em}
\textbf{References:}\\
Sanger, F., Nicklen, S., Coulson, A. R. (1977). DNA sequencing with chain-terminating inhibitors. \textit{Proc. Natl. Acad. Sci.}, 74(12):5463-5467.\\
Ronaghi, M. (2001). Pyrosequencing sheds light on DNA sequencing. \textit{Genome Res.}, 11(1):3-11.\\
Bentley, D. R., et al. (2008). Accurate whole human genome sequencing using reversible terminator chemistry. \textit{Nature}, 456(7218):53-59.\\
Eid, J., et al. (2009). Real-time DNA sequencing from single polymerase molecules. \textit{Science}, 323(5910):133-138.\\
Pevzner, P. A., Tang, H., Waterman, M. S. (2001). An Eulerian path approach to DNA fragment assembly. \textit{Proc. Natl. Acad. Sci.}, 98(17):9748-9753.
\end{technical}
