\begin{historical}
The quest to read DNA sequences began in the early 1970s with the development of methods to sequence RNA. Frederick Sanger and Alan Coulson first applied RNA sequencing techniques to DNA, but these early methods were laborious and could only read short stretches of sequence. The breakthrough came in 1977 when Sanger developed the chain-termination method, simultaneously with Allan Maxam and Walter Gilbert's chemical cleavage approach.

Sanger's method proved more practical and became the foundation for DNA sequencing over the next three decades. Early automation efforts in the 1980s introduced fluorescent labeling and capillary electrophoresis, replacing radioactive labels and polyacrylamide gels. Companies like Applied Biosystems commercialized automated sequencers that could process multiple samples simultaneously.

The Human Genome Project, launched in 1990, represented the first large-scale application of DNA sequencing technology. Led by figures like Craig Venter, Francis Collins, and John Sulston, the project aimed to sequence all 3.2 billion bases of human DNA. The effort required massive coordination between laboratories worldwide and drove significant improvements in sequencing technology and computational methods.

The project's approach relied on hierarchical shotgun sequencing: large DNA clones were mapped and ordered, then randomly fragmented and sequenced. Sophisticated algorithms were developed to assemble overlapping fragments into complete sequences. The draft human genome, completed in 2001, required over a decade of work and cost approximately \$3 billion.

The limitations of Sanger sequencing motivated the development of massively parallel methods. 454 Life Sciences, founded by Jonathan Rothberg in 2000, introduced the first commercially successful next-generation sequencing platform. Their pyrosequencing approach could generate millions of reads simultaneously, reducing costs and time requirements dramatically.

Illumina's acquisition of Solexa technology in 2007 established their dominance in the sequencing market. Their reversible terminator chemistry and bridge amplification method became the most widely used platform for genomic research. The combination of high throughput, relatively low cost, and good accuracy made Illumina sequencing the workhorse of modern genomics.

Ion Torrent, developed by Jonathan Rothberg after 454, introduced semiconductor-based sequencing that eliminated optical detection entirely. Their approach demonstrated how Moore's Law improvements in semiconductor manufacturing could be leveraged for DNA sequencing, suggesting a path toward continued cost reductions.

The development of single-molecule sequencing represented the next major advance. Pacific Biosciences, founded in 2004, spent over a decade developing their SMRT technology, overcoming significant technical challenges in detecting single fluorescent events. Their zero-mode waveguide innovation solved the fundamental problem of distinguishing incorporated nucleotides from the high background of unincorporated labeled nucleotides.

Oxford Nanopore took an entirely different approach, licensing technology originally developed by David Deamer and others in academic laboratories. Their protein nanopore platform represented a radical departure from all previous sequencing methods, requiring no amplification, labeling, or optical detection. The development of portable sequencers like MinION democratized access to long-read sequencing.

The computational challenges of genome assembly evolved alongside sequencing technology. Early algorithms like PHRAP and CAP3 were designed for Sanger reads. The emergence of short-read data necessitated new approaches, leading to the development of de Bruijn graph assemblers like Velvet, SOAPdenovo, and SPAdes. Long-read assembly algorithms like Canu and Flye have more recently addressed the challenges of high-error-rate data.

The broader impact of sequencing technology extends far beyond genomics. Applications in evolutionary biology, ecology, medicine, agriculture, and biotechnology have fundamentally changed these fields. The cost of sequencing has fallen faster than Moore's Law, dropping from roughly \$10 per base in 1990 to less than \$0.01 per million bases today.

Recent developments focus on improving accuracy, reducing costs further, and developing new applications. Advances in base-calling algorithms using machine learning have significantly improved nanopore accuracy. New methods for detecting DNA modifications, sequencing RNA directly, and analyzing chromatin structure continue to expand the capabilities of sequencing technology.

The field now faces new challenges as it moves toward routine whole-genome sequencing for clinical applications, population-scale studies, and real-time pathogen surveillance. The integration of sequencing with other omics technologies promises to provide comprehensive views of biological systems at unprecedented resolution.
\end{historical}
