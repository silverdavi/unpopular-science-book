\begin{SideNotePage}{
  \textbf{Top (Addition and Multiplication):} The exponential map translates addition into multiplication. Sequences of additions along the number line become products, illustrating how $\exp(a+b)=\exp(a)\cdot\exp(b)$. \par
  \textbf{Middle (Unit Circle Mapping):} On the complex plane, the exponential maps purely imaginary numbers $ia$ to points on the unit circle $e^{ia}\in U(1)$. Tangent vectors correspond to directions in the Lie algebra, while their exponential images wrap around the circle as group elements. \par
  \textbf{Bottom (Manifold and Tangent Plane):} In differential geometry, the exponential map extends to manifolds. A tangent vector at a point $p$ defines a geodesic, whose endpoint on the manifold is $\exp_p(v)$. Conversely, the logarithm map returns tangent vectors from manifold points. This generalization connects linear structure in tangent spaces to nonlinear geometry of manifolds. \par
}{21_ExponentialMapsLieTheory/21_ Exponentially Generalizable.pdf}
\end{SideNotePage}
