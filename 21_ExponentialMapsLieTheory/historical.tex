\begin{historical}
In 1614, John Napier introduced logarithms to simplify arithmetic, turning multiplication into addition. Though his tables were constructed geometrically and not in terms of a function \( e^x \), the underlying idea — of an operation whose inverse linearizes multiplication — was foundational. Over a century later, in 1748, Leonhard Euler formally introduced the exponential function, defining \( e^x \) through its power series and connecting it to the constant \( e \approx 2.71828 \). He also showed that this function uniquely solves the differential equation \( f' = f \) with \( f(0) = 1 \).

By the 19th century, the exponential function was generalized to complex analysis, where its series converges for all complex inputs, and to linear algebra via matrix exponentials. In parallel, Sophus Lie developed the theory of continuous transformation groups, now called Lie groups, and demonstrated how exponentiation links the tangent space at the identity (the Lie algebra) to the global group structure. In the early 20th century, Élie Cartan further extended these ideas into geometry and topology, embedding exponential maps into the study of curvature, connections, and geodesics. What began as a computational tool thus evolved into a central organizing principle of modern mathematics.
\end{historical}
