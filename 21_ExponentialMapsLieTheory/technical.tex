% Custom semantic coloring
\newcommand{\A}[1]{\textcolor{purple}{#1}}          % Elements from A
\newcommand{\B}[1]{\textcolor{orange!80!black}{#1}} % Elements from B
\newcommand{\F}[1]{\textcolor{blue!60!black}{#1}}   % Scalars from F

\begin{technical}
{\Large\textbf{Two Equivalences of the Exponential Map}}\\[0.7em]

\noindent\textbf{1. Functional Characterization via Derivation Operator (the other way is much easier!)}\\[0.5em]

Let \( \A{A} \) be an additive group and \( \B{B} \) a unital algebra over a field \( \F{F} \). Let \( \mathrm{Func}(\A{A}, \B{B}) \) denote the algebra of functions from \( \A{A} \) to \( \B{B} \), equipped with pointwise operations inherited from \( \B{B} \). Suppose there exists a linear operator
\[
K: \mathrm{Func}(\A{A}, \B{B}) \to \mathrm{Func}(\A{A}, \B{B})
\]
satisfying for all \( g, h \in \mathrm{Func}(\A{A}, \B{B}) \), and all \( \F{c} \in \F{F} \):

\begin{itemize}
  \item[(i)] Linearity: \( K(g + h) = K(g) + K(h), \quad K(\F{c} \cdot g) = \F{c} \cdot K(g) \)
  \item[(ii)] Product Rule: \( K(g \cdot h) = K(g) \cdot h + g \cdot K(h) \)
  \item[(iii)] Kills Constants: \( K(c_{\B{b}}) = \B{0} \), where \( c_{\B{b}}(\A{x}) \equiv \B{b} \in \B{B} \)
\end{itemize}

Let \( f \in \mathrm{Func}(\A{A}, \B{B}) \) satisfy:

\begin{itemize}
  \item[(iv)] \( K(f) = \F{\lambda} \cdot f \)
  \item[(v)] \( f(\A{0}) = \B{1} \)
  \item[(vi)] For all \( \A{y} \in \A{A} \), define \( g_{\A{y}}(\A{x}) := f(\A{x} \A{+} \A{y}) \), and assume \( K(g_{\A{y}}) = \F{\lambda} \cdot g_{\A{y}} \)
  \item[(vii)] If \( K(g) = \F{\lambda} \cdot g \) and \( g(\A{0}) = \B{0} \), then \( g(\A{x}) = \B{0} \) for all \( \A{x} \)
\end{itemize}

Define the function:
\[
g(\A{x}) := f(\A{x} \A{+} \A{y}) - f(\A{x}) \B{\cdot} f(\A{y}).
\]
Then:
\begin{align*}
g(\A{0}) 
&= f(\A{y}) - f(\A{0}) \B{\cdot} f(\A{y}) \\
&= f(\A{y}) - \B{1} \B{\cdot} f(\A{y}) \\
&= \B{0}.
\end{align*}

Now compute \( K(g) \), using linearity and the product rule:
\begin{align*}
K(g)(\A{x}) 
&= K\big(f(\A{x} \A{+} \A{y})\big) 
   - K\big(f(\A{x}) \B{\cdot} f(\A{y})\big) \\[0.5em]
&= \F{\lambda} \cdot f(\A{x} \A{+} \A{y}) 
   \\& - \Big( 
       K(f(\A{x})) \B{\cdot} f(\A{y}) 
       + f(\A{x}) \B{\cdot} K(f(\A{y})) 
     \Big) \\[0.5em]
&= \F{\lambda} \cdot f(\A{x} \A{+} \A{y}) 
   \\&- \Big( 
       \F{\lambda} \cdot f(\A{x}) \B{\cdot} f(\A{y}) 
       + \F{\lambda} \cdot f(\A{x}) \B{\cdot} f(\A{y}) 
     \Big) \\[0.5em]
&= \F{\lambda} \cdot f(\A{x} \A{+} \A{y}) 
   - 2 \F{\lambda} \cdot f(\A{x}) \B{\cdot} f(\A{y}) \\[0.5em]
&= \F{\lambda} \cdot \left( 
       f(\A{x} \A{+} \A{y}) 
       - f(\A{x}) \B{\cdot} f(\A{y}) 
   \right) \\[0.5em]
&= \F{\lambda} \cdot g(\A{x}).
\end{align*}

So \( K(g) = \F{\lambda} \cdot g \), and since \( g(\A{0}) = \B{0} \), uniqueness (vii) implies \( g(\A{x}) = \B{0} \) for all \( \A{x} \). Hence:
\[
f(\A{x} \A{+} \A{y}) = f(\A{x}) \B{\cdot} f(\A{y}).
\]

This confirms that the exponential law emerges directly from the derivation structure, normalization, and shift invariance — independent of analytic assumptions.

\vspace{1em}

\noindent\textbf{2. Geometric Interpretation in Riemannian Manifolds}\\[0.5em]

Let \( M \) be a Riemannian manifold and \( p \in M \). The exponential map at \( p \) is defined as:
\[
\exp_p(v) := \gamma_v(1),
\]
where \( \gamma_v \) is the geodesic starting at \( p \) with velocity \( v \), solving:
\[
\frac{D}{dt} \dot{\gamma}_v(t) = 0, \quad 
\gamma_v(0) = p, \quad 
\dot{\gamma}_v(0) = v.
\]

In \( \mathbb{R}^n \), geodesics are straight lines:
\[
\gamma_v(t) = p + t v, \quad 
\Rightarrow \quad 
\exp_p(v) = p + v.
\]

In a Lie group \( G \subset \mathrm{GL}_n(\mathbb{R}) \), one has:
\[
\gamma_X(t) = \exp(tX), \quad 
\frac{d}{dt} \gamma(t) = X \cdot \gamma(t), \quad 
\gamma(0) = I.
\]

This mirrors the scalar equation:
\[
x'(t) = a \cdot x(t), \quad 
x(0) = 1, \quad 
\Rightarrow \quad 
x(t) = e^{at}.
\]

The exponential map lifts linear generators — derivations or tangent vectors — to integrated flows across groups or manifolds, justifying its name in both analytic and geometric settings.

\vspace{0.5em}
\noindent\textbf{References:}\\
Lang, S. (2001). \textit{Fundamentals of Differential Geometry}. Springer.\\
Kobayashi, S., \& Nomizu, K. (1963). \textit{Foundations of Differential Geometry, Vol. I}. Wiley.
\end{technical}
