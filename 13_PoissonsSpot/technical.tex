\begin{technical}
{\Large\textbf{Mathematical Analysis of the Arago Spot}}\\[0.7em]

We derive the Arago spot from the wave theory of light using scalar diffraction theory. This analysis formalizes the intuitive explanation through the Huygens–Fresnel principle and the Helmholtz equation, culminating in a quantitative prediction for the on-axis intensity behind an opaque disk.

\noindent\textbf{Scalar Wave Equation and Helmholtz Formulation}\\[0.5em]
Light propagation in free space is governed by the scalar wave equation:
\[
\nabla^2 \psi - \frac{1}{c^2} \frac{\partial^2 \psi}{\partial t^2} = 0.
\]
Assuming monochromatic fields, we write $\psi(\mathbf{r},t) = \Re\{\Psi(\mathbf{r})\,e^{-i\omega t}\}$, which leads to the Helmholtz equation:
\[
\nabla^2 \Psi + k^2 \Psi = 0, \quad k = \frac{2\pi}{\lambda}.
\]
In diffraction problems, $\Psi(\mathbf{r})$ represents the complex field amplitude. The solution behind an obstacle is obtained by integrating contributions of secondary wavelets from all unobstructed regions, according to the Huygens–Fresnel principle.

\noindent\textbf{On-Axis Diffraction Integral and Constructive Interference}\\[0.5em]
Consider a plane wave incident on a circular opaque disk of radius $a$. The screen is located distance $b$ behind the disk. At the center point $P_1$ on the optical axis, all wavelets diffracted around the disk edge follow paths of length
\[
\sqrt{a^2 + b^2},
\]
so the path difference relative to the direct path is
\[
\Delta = \sqrt{a^2 + b^2} - b \approx \frac{a^2}{2b} \ll \lambda.
\]
This small $\Delta$ implies all contributing wavelets arrive in phase at $P_1$, yielding constructive interference.

To evaluate the on-axis field, we apply the Kirchhoff–Fresnel integral in cylindrical coordinates, integrating over the region $r \ge a$ (outside the shadow):
\begin{align}
U(P_1) &= -\frac{i}{\lambda} \frac{A\,e^{ik(g+b)}}{gb} \cdot \\
&\quad 2\pi \int_{a}^{\infty} \exp\left[ i\,\frac{k}{2} \left(\frac{1}{g} + \frac{1}{b}\right) r^2 \right] r\,dr, 
\end{align}
where $A$ is the incident amplitude and $g$ is the distance to the source. In the limit $g \to \infty$ (plane wave), and assuming axial symmetry and uniform phase across the disk edge, this simplifies to:
\[
U(P_1) = A_0 \cdot \frac{b}{\sqrt{b^2 + a^2}} \cdot \exp\left[ ik\sqrt{b^2 + a^2} \right],
\]
where $A_0 = \lim (A e^{ikg} / g)$ is the incident amplitude in the far field.

The resulting on-axis intensity is:
\[
I(0) = |U(P_1)|^2 = |A_0|^2 \cdot \frac{b^2}{b^2 + a^2}.
\]
In the limit $b \gg a$, this reduces to $I(0) \approx I_{\text{inc}}$, indicating that the central shadow point receives nearly the full beam intensity — a direct prediction of the Arago spot.

\vspace{0.5em}
\noindent\textbf{References}\\
Born, M. and Wolf, E. (1999). \textit{Principles of Optics}, 7th ed. Cambridge University Press.\\
Feynman, R. P., Leighton, R. B., \& Sands, M. (1965). \textit{The Feynman Lectures on Physics}, Vol. III.
\end{technical}
