\begin{SideNotePage}{
  \textbf{Model Comparison:} \par Three theoretical models are applied to two classic experiments: the double slit and the photoelectric effect. The goal is to highlight where predictions agree with the observed.

  \vspace{1.5em}
  \textbf{Top (Corpuscular):} \par Double-slit: no interference, \( I(x) \sim e^{-(x - x_1)^2} + e^{-(x - x_2)^2} \). \par Photoelectric: emission occurs only if photon energy exceeds the threshold, \( R \sim \Theta(\omega - \phi) \).

  \vspace{1.5em}
  \textbf{Middle (Wave):} \par Double-slit: interference arises from wave superposition, \( I(x) \sim \cos^2(x) \). \par Photoelectric: continuous energy absorption incorrectly predicts emission at all \( \omega \), \( R \sim 1 \).

  \vspace{1.5em}
  \textbf{Bottom (QED):} \par Double-slit: coherent paths interfere, \( I(x) \sim |\Psi_1 + \Psi_2|^2 \sim \cos^2(x) \). \par Photoelectric: decoherent transition amplitudes yield \( R \sim |\mathcal{M}|^2 \sim \Theta(\omega - \phi) \). The same path integral formalism applies, but coherence conditions differ.

  \vspace{1.5em}
  Green frames highlight where predictions match experimental outcomes.
}{13_PoissonsSpot/13_ Right on Spot.pdf}
\end{SideNotePage}