\begin{historical}
At the start of the 19th century, France’s scientific institutions were dominated by \textbf{Isaac Newton}’s \emph{corpuscular theory} of light. Though British, his mechanical worldview had become the orthodox framework for nature in France, treated with near-religious devotion. His theories were seen as mathematically perfect expressions of divine order, where challenging his optical framework risked accusations of scientific heresy. This dominance persisted through institutional inertia within the \emph{Académie des Sciences}, where senior mathematicians like \textbf{Siméon Denis Poisson} and the recently deceased \textbf{Joseph-Louis Lagrange} had built their legacies on Newtonian principles.

In contrast, \textbf{Christiaan Huygens}’ earlier \emph{wave theory}, though developed in Paris in the late 17th century, had fallen out of favor. His principle — that every point on a wavefront acts as a source of secondary wavelets — was largely viewed as a heuristic, lacking the mechanical precision demanded by the Newtonian establishment.

The political context mattered. Post-Napoleonic Wars, French science underwent institutional consolidation. State-sponsored prizes regulated scientific boundaries as much as they rewarded discovery. The Académie’s \emph{Grand Prix} competitions reinforced orthodoxy while offering recognition. When the 1818 diffraction prize was announced, it tested allegiance as much as it sought an explanation.

Into this charged environment entered \textbf{Augustin-Jean Fresnel}, a provincial engineer without formal academic standing. He submitted a comprehensive wave theory treating interference and diffraction as fundamental, not anomalous. Fresnel extended Huygens’ principle with integrals and phase relations to predict intensity patterns. For the Academy establishment, this represented an unwelcome challenge to Newtonian dogma that had defined scientific legitimacy for generations.

Poisson, serving on the jury, represented the old guard — a committed Newtonian who viewed mathematical elegance as the ultimate arbiter of truth. \textbf{Dominique-François Arago}, also on the committee, occupied a more ambiguous position. Though not yet fully aligned with the wave camp, Arago had corresponded with Fresnel and was known for his openness to alternative frameworks. The committee thus represented an ideological fault line within French science: Newtonian orthodoxy versus a nascent wave revival driven by empirical evidence and mathematical rigor. What followed would be a confrontation not only of theories but of institutional momentum and scientific methodology.
\end{historical}
