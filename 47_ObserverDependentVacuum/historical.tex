\begin{historical}
Shortly after Einstein’s 1905 introduction of special relativity, Hermann Minkowski’s geometric formulation of spacetime (1908) hinted that motion might alter physical measurements. In the 1920s and 1930s, when Paul Dirac and others established quantum field theory, scientists began to realize that “empty space” could look different to observers in different states of motion.

In the 1970s, Stephen Fulling (1973) and William Unruh (1976) demonstrated that an accelerating observer in flat spacetime perceives a thermal bath of particles, while an inertial observer sees none. In 1974, Stephen Hawking extended these insights to black holes, showing that they emit faint thermal radiation — a result now known as Hawking radiation. These milestones underscored that an observer’s trajectory and gravitational context shape the notion of particle content.

Later work clarified how coordinate choices and boundary conditions influence which states appear particle-free. By the early 1980s, researchers like N. D. Birrell and P. C. W. Davies had explored these phenomena in curved spacetime, deepening our understanding of black hole physics and cosmic expansion. Thus, the idea that particle counts depend on the observer’s frame emerged as a central concept in modern theoretical physics.
\end{historical}