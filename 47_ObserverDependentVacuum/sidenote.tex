\begin{SideNotePage}{
  \textbf{Top (Event Horizon and Causal Disconnection):}  
  At the boundary of a black hole, the event horizon marks the limit of causal contact. From outside, an infalling astronaut appears frozen near the edge; from their own frame, they cross smoothly. Different observers slice spacetime differently—no paradox, just relativity of coordinate systems. \par

  \textbf{Middle (The Andromeda Paradox and Relativity of Simultaneity):}  
  Two observers at the same point on Earth—one stationary, one walking—will disagree on what is happening right now in distant galaxies. A moon event (like an egg breaking) can be “now” for the stationary, and “not yet” for the runner. Relativity of simultaneity makes “present” frame-dependent. \par

  \textbf{Bottom (Quantum Fields and Observer-Dependent Particles):}  
  What counts as a particle depends on the observer. Each pink sphere represents a local quantum field mode, vibrating against the vacuum. But acceleration or curvature shifts the vacuum state, so an observer may see particles where another sees none—like Unruh radiation or Hawking emission. \par

}{47_ObserverDependentVacuum/47_ Optimist_s Take on Vacuum.pdf}
\end{SideNotePage}
