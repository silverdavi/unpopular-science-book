\begin{SideNotePage}{
  \textbf{Top (Spacetime Geometry):} The geometry near a black hole distorts spacetime so severely that, from a coordinate perspective, the black hole itself lies in the future light cone of any nearby event. For an infalling observer, continuing forward in time inevitably means moving closer to the singularity — tomorrow \textit{is} the black hole in spacetime terms. \par
  \textbf{Middle (Observer Perspectives):} Two perspectives on crossing the horizon. From the distant observer’s viewpoint (left), the astronaut appears to freeze at the horizon, increasingly redshifted and dimmed, never quite crossing. From the astronaut’s own frame (right), there’s no discontinuity — no freezing, no slowdown — only smooth, uninterrupted free fall through the horizon in finite proper time. \par
  \textbf{Bottom (Penrose Diagram):} The full Penrose diagram for the maximally extended Schwarzschild solution, showing all causal regions: the external universe, black hole interior, white hole region, and a second asymptotically flat universe. The diagram captures the global causal structure: horizons, infinities and singularities.
}{23_BlackHoleTimeDilationRedshift/23_ A Place at the End of Time.pdf}
\end{SideNotePage}
