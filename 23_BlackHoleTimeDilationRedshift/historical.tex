\begin{historical}
Karl Schwarzschild derived the first exact solution to Einstein's field equations in 1916, giving rise to what is now known as the Schwarzschild metric. This solution described the spacetime geometry outside a static, spherically symmetric mass, and revealed an intriguing radius where the metric becomes singular — a mathematical curiosity at the time.

In 1939, Robert Oppenheimer and Hartland Snyder explored the gravitational collapse of massive stars, showing that under general relativity, such collapse could lead to the formation of a region from which no signals escape: the conceptual precursor to what we now call a black hole.

The term “black hole” was popularized by John Wheeler in the 1960s, highlighting that these regions are not conventional objects, but causal domains shaped by the warping of spacetime. In 1963, Roy Kerr discovered an exact solution for rotating black holes. The Kerr metric demonstrated that black holes can possess angular momentum, vastly enriching the theory’s physical relevance. Unlike Schwarzschild’s static solution, the Kerr geometry features an ergosphere, frame dragging, and a more rich causal structure — including an inner and outer horizon.

Throughout the 20th century, indirect astrophysical evidence for black holes mounted, from X-ray binaries to quasars and galactic nuclei. By the 2010s, observations of gravitational waves and the first black hole shadow image (captured by the Event Horizon Telescope in 2019) solidified their status as real astrophysical objects.

Despite this progress, black holes continue to raise deep questions. At their core lies the unresolved issue of singularities — regions where classical spacetime is undefined — and the challenge of unifying general relativity with quantum theory remains a central frontier in physics.
\end{historical}