\begin{historical}
Interest in how Gaussian measures behave under geometric constraints emerged in the mid-twentieth century, particularly in multivariate statistics and convex geometry. By the 1950s, researchers studying elliptical distributions began formulating conjectures about the probability content of intersections between symmetric convex regions.

The modern form of the Gaussian Correlation Inequality (GCI) was solidified in the 1970s through work by Das Gupta, Olkin, Pitt, and others, who framed it in terms of standard Gaussian measures over \(\mathbb{R}^n\). They asked whether Gaussian probability favors overlap: specifically, whether the measure of the intersection of two symmetric convex sets is always at least as large as the product of their individual measures. The conjecture attracted attention because it combined natural geometric symmetry with the most analytically tractable probability distribution.

Over the following decades, progress was made in restricted settings. The inequality was proven for two-dimensional cases, for coordinate-aligned boxes, and for ellipsoids. The partial results relied on tools from real analysis, measure theory, and convex optimization. The general case resisted all attempts, despite appearing elementary in formulation.

In 2014, a breakthrough came from Thomas Royen, a retired statistician with a background in pharmaceutical applications. Royen published a short paper that resolved the inequality in full generality. His approach was elementary in the technical sense: it used standard tools, required no heavy machinery, and invoked only modest linear algebra and probability. Nonetheless, it connected several overlooked identities in a way that previous attempts had not. Although initially unnoticed, Royen's proof was soon verified and reformulated in expository papers by Latała, Matlak, and others, and has since been accepted as the definitive solution to the GCI.
\end{historical}
