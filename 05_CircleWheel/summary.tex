The terms "wheel" and "cycle" (but not circle!) derive from Proto-Indo-European *kʷékʷlos despite their phonetic dissimilarity in modern languages. Regular sound shifts transformed this root differently in Germanic and Hellenic branches through documented phonological processes. These linguistic patterns preserve evidence of Bronze Age terminology and illustrate consistent patterns of language change. Comparative methods identify these transformations through sound correspondences across Indo-European languages.