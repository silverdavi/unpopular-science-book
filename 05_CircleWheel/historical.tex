\begin{historical}
    The systematic study of language families emerged in the eighteenth and nineteenth centuries, when philologists began identifying regular correspondences between phonemes, grammatical structures, and syntactic patterns across geographically distant languages. Sir William Jones's 1786 observation that Sanskrit, Greek, and Latin exhibited similarities unlikely to arise by chance laid the foundation for reconstructing their common ancestor: \textbf{Proto-Indo-European (PIE)}, a prehistoric language hypothesized to have been spoken around 3000–4000~BCE in the Pontic–Caspian steppe.
    Franz Bopp advanced the field by developing the \textit{comparative method}, a procedure for recovering unattested forms through systematic analysis of sound correspondences and inflectional morphology. August Schleicher introduced genealogical tree diagrams to represent linguistic divergence — a model that remains standard today. These methods enabled reconstruction of PIE roots with high consistency, revealing shared grammatical principles across its descendants.
    The PIE daughter branches — Indo-Iranian, Hellenic, Italic, Celtic, Germanic, Balto-Slavic, and Anatolian, among others — developed distinct phonologies while preserving identifiable ancestral features. For example, the PIE voiced aspirated stop \piefont{*bʰ} appears as \emph{bh} in Sanskrit, \emph{f} in Latin, and \emph{b} in English. Such rules apply across lexicon and morphology, enabling broad reconstruction. Grimm's Law captured the phonetic shifts distinguishing Proto-Germanic from other Indo-European languages, accounting for correspondences like Latin \emph{pater}, Greek \emph{patēr}, Sanskrit \emph{pitṛ́}, and English "father."
    Though no written PIE record survives, its form emerges from consistent patterns in attested ancient languages including Hittite, Old Church Slavonic, and Old Persian. Core vocabulary — kinship terms, natural elements, agriculture, and tools — resists borrowing and anchors the comparative framework.
    PIE reconstruction also illuminates semantic evolution. Many roots generate both concrete and abstract derivatives across daughter languages. Motion, time, and cyclical processes often yield terms spanning physical action, ritual practice, and philosophical speculation.
\end{historical}