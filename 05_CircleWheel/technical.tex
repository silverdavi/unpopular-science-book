\begin{technical}
{\Large\textbf{The Phonological Evolution of Labiovelars in the Indo-European Descendants of \piefont{*kʷékʷlos}}}\\[0.7em]

Proto-Indo-European (PIE) contained labiovelar stops (\piefont{*kʷ}, \piefont{*gʷ}, \piefont{*gʷʰ}) with simultaneous velar closure and labialization, contrasting with plain velars (\piefont{*k}, \piefont{*g}, \piefont{*gʰ}) and palatalized velars (\piefont{*ḱ}, \piefont{*ǵ}, \piefont{*ǵʰ}). The PIE root \piefont{*kʷékʷlos}, a reduplicated form of \piefont{*kʷel-} ("to turn"), underwent systematic shifts across branches. \textbf{In Greek},
By Mycenaean Greek (c. 1400 BCE, attested in Linear B), Greek had lost labiovelars in most environments, replacing them with plain velars. Thus, \piefont{*kʷ} became \piefont{*k}:
\[
\textgreek{κύκλος} \text{ (\emph{kyklos}) < PIE } \piefont{*kʷékʷlos}
\]
This process, known as de-labialization, removed lip rounding. The same shift appears in \textgreek{πέντε} (\emph{pente}, "five") < PIE \piefont{*pénkʷe}.
Some Greek dialects retained traces of labiovelars, but standard evolution eliminated them. \textbf{In Sanskrit}, labiovelars merged with palatals before front vowels, so \piefont{*kʷ} became \textsanskrit{च} (\emph{c}, [t͡ʃ]):
\[
\textsanskrit{चक्र} \text{ (\emph{chakra}) < PIE } \piefont{*kʷékʷlos}
\]
This is part of a broader Indo-Iranian shift where labiovelars fronted or merged with palatals. \textbf{In Latin}, the reflex of \piefont{*kʷ} depended on the following vowel:
- Before front vowels (\piefont{*e, *i}): \piefont{*kʷ} remained a labiovelar, later spelled \emph{qu}.
- Before back vowels (\piefont{*o, *u}): \piefont{*kʷ} lost its labialization and became \emph{c}.
- Before \piefont{*a}: Variation occurred, but de-labialization was common.

For example, in Latin \emph{colere} ("to cultivate"), the labiovelar \piefont{*kʷ} was retained before a back vowel: PIE \piefont{*kʷel-} → Latin \emph{colere}.
In contrast, \emph{circulus} ("circle") derives from PIE \piefont{*sker-}, not a labiovelar. \textbf{In Proto-Germanic}, Grimm's Law altered the stop system: \piefont{*kʷ} → \piefont{*hw}. Thus, \piefont{*kʷékʷlos} became \piefont{*hwehwlą} (Proto-Germanic), which evolved into Old English \emph{hwēol}, Middle English \emph{whele}, and Modern English:
\[
\emph{wheel}
\] \textbf{Summary}: Greek de-labialized (\piefont{*kʷ} > \piefont{k} → \textgreek{κύκλος}), Sanskrit palatalized (\piefont{*kʷ} > \piefont{c} → \textsanskrit{चक्र}), Latin retained or de-labialized conditionally (\emph{colere, circulus}), and Germanic fricativized via Grimm's Law (\piefont{*kʷ} > \piefont{hw} → \emph{wheel}).

\noindent
These transformations illustrate how a single PIE labiovelar stop produced diverse reflexes across Indo-European languages, shaping words that remain etymologically linked despite significant phonetic divergence.

\vspace{0.5em}
\noindent\textbf{References}\\
Fortson, B. (2010). \emph{Indo-European Language and Culture: An Introduction}. Wiley-Blackwell.\\
Ringe, D. (2006). \emph{From Proto-Indo-European to Proto-Germanic}. Oxford University Press.\\
Online Etymology Dictionary: \url{https://www.etymonline.com/}\\
\end{technical}
