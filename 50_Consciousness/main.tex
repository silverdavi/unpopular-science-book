General anesthesia abolishes subjectivity itself. Other drugs alter perception, mood, or pain. Anesthetics suspend the condition for all perception and mood. A standard intravenous dose of propofol — two milligrams per kilogram — eliminates awareness in less than a minute. The transition is not gradual. One moment the subject tracks voices and surroundings; the next moment there is no report, no continuity of thought, and no subsequent memory. The effect is reliable, reversible, and indispensable to surgical practice. Yet it remains unexplained.

Different drugs converge on this endpoint through divergent and sometimes contradictory mechanisms. Propofol potentiates $\gamma$-aminobutyric acid type A (GABA\textsubscript{A}) receptors, amplifying inhibitory currents and reducing excitability across the cortex. Isoflurane, sevoflurane, and other volatile anesthetics bind to potassium and sodium channels, producing generalized dampening of neuronal firing. Nitrous oxide and xenon inhibit $N$-methyl-D-aspartate (NMDA) receptors, reducing excitatory drive. Ketamine blocks NMDA receptors yet increases cortical activity globally, producing electroencephalographic patterns closer to wakefulness than sleep while still abolishing awareness. Distinct molecular actions — some silencing neurons, some exciting them — terminate consciousness with equal reliability. No unified account reconciles how these divergent perturbations converge on the same disappearance.

The search for such an account once looked promising. At the turn of the twentieth century, Hans Meyer and Charles Ernest Overton noted a striking correlation: anesthetic potency scales with lipid solubility. The Meyer–Overton rule suggested that anesthetics dissolved into neuronal membranes, altering their physical properties. For decades this correlation dominated, reinforced by its simplicity. Yet the correlation broke. Non-immobilizers — molecules with high lipid solubility — fail to anesthetize. Others, poorly soluble in fat, work effectively. The membrane theory could not account for exceptions.

Attention shifted to receptors. Different anesthetic classes bind to distinct proteins: GABA\textsubscript{A}, NMDA, and two-pore domain potassium channels among prime candidates. Yet receptor theories also encounter anomalies. No single target is necessary. Mice engineered with GABA\textsubscript{A} subunits resistant to volatile anesthetics still lose consciousness when exposed. No single target is sufficient: receptor agonists or antagonists with precise effects on candidate pathways often fail to produce general anesthesia. What remains is a map of partial correlates, not a law specifying why awareness vanishes.

Network hypotheses move up a level. Thalamic "switch-off" models propose that sensory relay and intralaminar nuclei disengage cortical broadcasting. Alternatives hold that long-range cortico-cortical integration degrades: effective connectivity fragments, ignition-like reverberation collapses, and fronto-parietal synchrony decouples. Empirically, anesthetic depth tracks changes in spectral power, complexity, and coherence. But counterexamples persist. Ketamine increases cortical activity and high-frequency power yet abolishes consciousness. Dexmedetomidine reduces thalamic throughput yet permits vivid dreams.

The opposite extreme clarifies the boundary. Fatal familial insomnia, a prion disease, destroys neurons in the thalamus, especially in the anteroventral and mediodorsal nuclei. These nuclei regulate sleep architecture. As they degenerate, the subject loses the ability to enter non-rapid eye movement sleep. Ordinary fatigue accumulates, but sleep never arrives. Patients remain in escalating wakefulness until death, often within a year of symptom onset. Other neurodegenerative diseases erase memory, language, or motor control. This one erases the possibility of unconsciousness. Consciousness persists compulsively until the body collapses under uninterrupted wakefulness.

Anesthesia and prion disease bracket the same mystery. Milligrams of a synthetic molecule suspend awareness entirely. Widespread neuronal loss fails to interrupt it. Consciousness is too easy to subtract and, simultaneously, impossible to eliminate. The paradox indicates that manipulations reach only the conditions under which consciousness manifests. They do not specify what consciousness is. Practitioners can toggle the switch without knowing what is being switched.

Measuring consciousness remains harder than turning it off. Clinical scales rely on responsiveness; neurophysiology adds proxies: cross-regional EEG coherence, perturbational complexity from TMS-evoked responses, and theoretical constructs like Integrated Information Theory's $\Phi$. Each stumbles. Some unresponsive patients process speech. High $\Phi$ can be assigned to systems with no plausible subjectivity. EEG signatures of wakefulness can appear under amnestic sedation. Competing theories — Global Workspace, Integrated Information, Recurrent Processing — disagree on what makes a state conscious, and experiments often adjudicate proxies rather than experience itself.

The working picture is pragmatic: multiple molecular routes converge on a few network-level motifs — reduced ignition, impaired integration, altered thalamocortical gating — sufficient to block access to a reportable workspace. That picture explains much of practice and little of essence.

The gap between control and understanding demands a different frame entirely. Consciousness is singular. Treating it as a parameter vector to be fit by a support-vector machine or a deep network condescends to the phenomenon. A classifier extracts invariants and separates classes. Consciousness is first-personal presence and deliberative control. No change of basis, no margin optimization, no loss function turns one into the other. The distinction is categorical.

Any research program that seeks to locate the origin of consciousness in physical mechanisms presupposes the very phenomenon it attempts to explain. The attempt uses consciousness to investigate consciousness. You deploy attention, select among hypotheses, compare results, and conclude. Each of those acts exercises the thing under study. The circularity constitutes an epistemological wall. Every explanation collapses back into the standpoint it aimed to eliminate.

Free will and physics appear incompatible. If physics is a complete description — deterministic or stochastic, local or quantum, simulated or fundamental — then every decision reduces to a trajectory in state space. Free will becomes an illusion, a narrative that complex systems tell themselves about their own deterministic unfolding. But if free will exists, then physics is either incomplete or itself simulated. The standoff seems symmetric: pick your side.

The symmetry is false. Free will is the lived fact. Physics is the constructed model. If physics denies free will, physics has misclassified its own status. Constructing, testing, and revising physical theories requires a subject that directs thought, selects among candidate explanations, and exercises judgment. To declare that subject an illusion saws off the branch on which the declaration sits. Illusions presuppose a subject that misperceives. If the subject is deleted, the word "illusion" loses reference. The sentence "free will is an illusion" requires a subject that can contrast seeming with being. That requirement reinstates free will.

Consciousness is the exercise of will on one's own stream of thought. Hold, release, redirect, compare, adopt, reject. Deliberate selection among candidate continuations. The stream is the ordered sequence of contents available for such selection. The subject is the locus at which selection is enacted. Physics may constrain, enable, or block. It does not exhaust.

Rank commitments by revision cost — how much of the apparatus of knowing must be discarded if the commitment fails.

\medskip

\textit{I know the sky is blue.} If tomorrow I learn it is an optical illusion — scattering, refraction, atmospheric tricks — fine. Mildly interesting. Nothing essential breaks.

\textit{I know there is gravity.} If someone pulls the plug and reveals the simulation, forces redraw, mass no longer bends spacetime — I am stunned for days. Ontology shifts. But then I rebuild the catalog of causes and move on. The capacity to model persists.

\textit{I know there are atoms and electrons.} If the simulation ends and someone shows me a different physics underneath, mind blown. Weeks to recover. But recovery is possible. I can still compare, infer, and correct.

\textit{I know $2+3=5$.} If someone demonstrates that arithmetic itself is wrong — that I had a cognitive shortcut, and really $2+3=11$ — the machinery of thought disassembles. Counting, comparison, consistency all rest on that foundation. Without it, reasoning collapses.

\textit{I know I have free will.} I know I exist as the thing that directs its own thoughts. If this turns out to be false — in the strong sense that there is no subject, no selection, no directedness — then there is no "I" left to register the failure. Incompatible with the standpoint from which acceptability is judged.

\medskip

The highest commitment dominates. Every statement, inference, or model presupposes a subject that can assert, doubt, compare, and revise. That presupposition is the content of the highest tier. Lower tiers describe states of affairs in the world. The highest tier secures the existence of the knower to whom the world appears. In any conflict, the knower wins. Without the knower, conflict is unintelligible.

Write the revision cost as $C(\cdot)$. Then:
\[
C(\text{appearances}) \ll C(\text{physics}) \ll C(\text{mathematics}) \ll C(\text{agency}).
\]
The last inequality is decisive. If agency conflicts with physics, agency prevails. Agency is the condition for there being importance at all.

Neural correlates, receptor binding, thalamic gating, and network fragmentation describe \textit{when} consciousness appears or vanishes and \textit{how} physiology couples to report. That scope is exact and valuable. \textit{What it is to be} the subject for whom appearance and vanishing matter lies elsewhere. "When does awareness switch off?" asks about timing and mechanism. "What is it to direct one's own thought?" asks about the standpoint that makes timing intelligible. Neuroscience answers the first. Philosophy addresses the second. Conflating them produces the reduction error: mistaking access conditions for the subject to whom access matters.

Research that maps brain states to behavioral outputs achieves correlation. Intervention studies that disrupt nodes and track changes achieve mechanism. Both are genuine progress. Constitution — the precondition without which correlation and mechanism cannot be stated — remains distinct. Consciousness sits at the constitutional level. Adding more parameters or finer imaging cannot bridge the gap. The gap is categorical.

Anesthesia deletes awareness in seconds. Fatal familial insomnia prevents its deletion for months. Both manipulate conditions. Neither touches essence. We can flip the switch without knowing what is being switched. The circuit diagram remains incomplete because the subject that the diagram aims to explain is also the subject examining the diagram. That reflexive loop cannot be broken by adding variables. It can only be acknowledged. Consciousness occupies the apex of the revision hierarchy, where it cannot be dissolved into the mechanisms it observes. Attempting dissolution commits category error at the highest cost.

