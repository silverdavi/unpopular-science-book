Milligrams of propofol erase consciousness in seconds. Fatal familial insomnia prevents its cessation for months until death. While we can reliably toggle awareness through divergent molecular routes — some that silence neurons, some that excite them — yet no unified mechanism explains why subjectivity vanishes. Consciousness cannot be reduced to neural correlates or fit by classifiers. Any attempt to locate its origin in physical mechanisms presupposes the very phenomenon under study. Free will and physics appear incompatible, but the standoff is asymmetric: agency is the lived fact that makes physics construction possible. Consciousness occupies the apex of a revision hierarchy where, in any conflict with lower-level descriptions, the knower must prevail.