\begin{technical}
    {\Large\textbf{Agency as Axiomatic Ground}}\\[0.5em]

    \textbf{Epistemic Formalism}\\[0.3em]
    Let $S$ be a knowing subject, $\mathcal{P}$ the set of propositions, and $K_S \subseteq \mathcal{P}$ the commitment set of propositions $S$ holds true. For $p, q \in K_S$, write $p \vdash q$ if $q$ logically follows from $p$.
    
    Define \textit{revision cost}:
    \[
    C(p) := |\{q \in K_S \mid p \vdash q\}|
    \]
    This induces partial order $(K_S, \preceq)$ where $p \preceq q \iff C(p) \leq C(q)$.

    \textbf{Hierarchy with Revision Costs}\\[0.3em]
    \begin{itemize}[leftmargin=*,topsep=0pt,itemsep=1pt]
        \item $p_1$: "Sky is blue"\\
        \textit{If false}: Mildly interesting. Nothing breaks.
        \item $p_2$: "Not in The Matrix"\\
        \textit{If false}: Stunned. Rebuild ontology. Days to recover.
        \item $p_3$: "Gravity exists"\\
        \textit{If false}: Physics rebuilds. Weeks to recover.
        \item $p_4$: "$2+3=5$"\\
        \textit{If false}: Arithmetic collapses. Reasoning disassembles.
        \item $p_5$: "$P \vee \neg P$" (excluded middle)\\
        \textit{If false}: Logic fails. Cannot reason about contradictions.
        \item $A$: "I direct my thought"\\
        \textit{If false}: No subject remains to register the failure.
    \end{itemize}
    Strictly: $C(p_1) \ll C(p_2) \ll C(p_3) \ll C(p_4) \ll C(p_5) \ll C(A)$.

    \textbf{Agency as Maximal Element}\\[0.3em]
    \textit{Agency} ($A$): capacity to perform operations on $K_S$ (selecting, comparing, affirming, rejecting propositions). This is control over thought, not physical action.
    
    To revise $K_S$ by removing $A$ requires performing an operation on $K_S$, which presupposes $A$. Thus revision of $A$ is self-undermining:
    \[
    A \vdash p \quad \forall p \in K_S \quad \Rightarrow \quad C(A) = |K_S|
    \]
    Agency is the maximal element in $(K_S, \preceq)$.

    \textbf{Resolution: Physics vs. Agency}\\[0.3em]
    Let $P_{\text{phy}}$ = "Universe fully determined by physical law." If $P_{\text{phy}} \vdash \neg A$, we face contradiction: $\{P_{\text{phy}}, A, P_{\text{phy}} \to \neg A\}$.
    
    Comparing revision costs:
    \begin{itemize}[leftmargin=*,topsep=0pt,itemsep=1pt]
        \item $C(P_{\text{phy}})$: finite. Subject persists, rebuilds model.
        \item $C(A)$: total. No subject remains.
    \end{itemize}
    
    Since $C(P_{\text{phy}}) < C(A)$, agency dominates. Physical models invalidating agency are epistemologically incoherent.

    \columnbreak

    \textbf{Category Error Formalized}\\[0.3em]
    Let $\mathcal{C}_{\text{phys}}$ be category of physical processes with morphisms as causal relations. Let $\mathcal{C}_{\text{agent}}$ be category of first-person agency with morphisms as deliberative acts.
    
    No structure-preserving functor $F: \mathcal{C}_{\text{phys}} \to \mathcal{C}_{\text{agent}}$ exists satisfying:
    \[
    F(\text{correlation}) = \text{constitution}
    \]
    Classifiers extract invariants; subjects exert control. These are distinct logical types. Classification $\not\cong$ first-personal presence.

    \textbf{Philosophical Grounding}\\[0.3em]
    \textit{Descartes' Cogito ergo sum} (1641): The act of doubting presupposes the existence of a doubter. Even radical skepticism cannot eliminate the thinking subject. This establishes the subject as the foundation of knowledge, not a conclusion derived from it.

    \textit{Kant's Transcendental Apperception} (1781): The unity of consciousness is not empirically observed but is the logical precondition for any structured experience. The "I think" must accompany all representations. Without a unified subject, no comparison, judgment, or synthesis of data is possible.

    \textit{Modern Parallel}: These establish that agency is axiomatic, not a theorem derived from lower-level descriptions. The subject is the condition for there being theorems, descriptions, and derivations at all.

    \textbf{Implications}\\[0.3em]
    \begin{itemize}[leftmargin=*,topsep=0pt,itemsep=2pt]
        \item Neural correlates map \textit{when} consciousness appears, not \textit{what} it constitutes.
        \item Illusions require a subject to be deceived. Denying the subject makes "illusion" incoherent.
        \item Physical models constrain manifestation conditions but cannot invalidate the standpoint from which models are constructed and evaluated.
    \end{itemize}

    \vspace{0.5em}
    \textbf{References:}\\
    {\footnotesize
    Descartes, R. (1641). \textit{Meditations on First Philosophy}.\\
    Kant, I. (1781/1787). \textit{Critique of Pure Reason}.\\
    }
\end{technical}