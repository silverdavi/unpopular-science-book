\begin{SideNotePage}{
  \textbf{Top (Thermal Motion):} Thermal motion in equilibrium. Even with no applied voltage, conduction electrons within the wire exhibit rapid random motion (thermal velocities on the order of $10^5$ to $10^6$ m/s), but with no net directional flow — resulting in zero macroscopic current. \par
  \textbf{Middle (Net Drift):} Net drift under an applied electric field. Applying a voltage creates an electric field along the wire, introducing a slight statistical bias in electron velocities. This produces a slow net drift (typically fractions of a millimeter per second), superimposed on the much faster thermal motion. Despite the minuscule drift speed, the circuit responds almost instantly to changes in voltage, as the electromagnetic field propagates at near light speed. \par
  \textbf{Bottom (Energy Flow):} Macroscopic current and surrounding fields. The collective electron drift constitutes a measurable current, which, according to Ampère’s Law, is accompanied by a magnetic field encircling the wire (as shown by the green loops and the right-hand rule). Together, the electric field driving the current and the magnetic field generated by it produce a nonzero Poynting vector $\vec{S} = \vec{E} \times \vec{H}$ oriented along the wire. This indicates that energy flow occurs through the electromagnetic field in the space surrounding the conductor — not within the conductor itself. The drift of electrons ensures charge continuity but plays no role in determining the speed of energy delivery.
}{04_EMFieldsEnergyFlow/04_ Think Outside the Wire.pdf}
\end{SideNotePage}
