\begin{historical}  
Early investigations linking electricity and magnetism began with Hans Christian Ørsted’s 1820 observation that a current-carrying wire deflected a nearby compass needle, demonstrating that electric currents generate magnetic effects. Shortly thereafter, André-Marie Ampère quantified these interactions, paving the way for a unified framework. Michael Faraday’s idea of lines of force emphasized that fields permeate space and mediate electrical phenomena.  

In the 1860s, James Clerk Maxwell brought together these concepts, formulating a concise set of equations that governs how changing electric and magnetic fields propagate as electromagnetic waves. This discovery challenged earlier assumptions that electrical energy was confined to wires alone. John Henry Poynting then introduced the Poynting vector in 1884, clarifying how electromagnetic energy flows through the space surrounding conductors.  

At the same time, Oliver Heaviside simplified Maxwell’s equations into the modern vector calculus form, making them more accessible for engineers and physicists. His insights led to a better understanding of power transmission, highlighting that energy is not carried by the motion of electrons in a wire but by the surrounding electromagnetic fields. This change in perspective would later prove critical in the development of radio transmission, telecommunication systems, and waveguide theory.  

Despite these breakthroughs, the older "current as fluid in a pipe" analogy persisted in basic electrical education well into the 20th century. Only with the advent of high-frequency engineering and transmission line theory did the role of electromagnetic fields become widely acknowledged in practical applications. Today, the principles laid down by Ørsted, Ampère, Faraday, Maxwell, Poynting, and Heaviside form the foundation of modern electromagnetism, from power grids to fiber-optic communication.  
\end{historical}