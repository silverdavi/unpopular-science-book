\begin{technical}
{\Large\textbf{Carbon Fixation and Mass Accumulation in Trees}}\\
Trees accumulate mass through atmospheric $\mathrm{CO}_2$ fixation powered by sunlight. This section quantifies the chemical and energetic processes converting gaseous carbon into solid biomass.
\noindent\textbf{Light-Driven Reactions}\\
Photosystems I and II generate ATP and NADPH from light energy (680 nm photons ≈ 176 kJ/mol):
\begin{align}
2\,\mathrm{H}_2\mathrm{O} 
&+ 2\,\mathrm{NADP}^+ 
+ 3\,\mathrm{ADP} 
+ 3\,\mathrm{P}_i 
+ h\nu \nonumber \\
&\rightarrow 2\,\mathrm{NADPH} 
+ 3\,\mathrm{ATP} 
+ \mathrm{O}_2.
\end{align}
Quantum requirement: 8–10 photons per $\mathrm{CO}_2$ molecule fixed.
\noindent\textbf{Carbon Fixation and Biomass Synthesis}\\
In the Calvin–Benson cycle, carbon dioxide is enzymatically fixed into triose phosphates using the energy carriers from the light reactions. The overall reaction for one glucose unit is:
\begin{align}
6\,\mathrm{CO}_2 
&+ 18\,\mathrm{ATP} 
+ 12\,\mathrm{NADPH} \nonumber \\
&\rightarrow \mathrm{C}_6\mathrm{H}_{12}\mathrm{O}_6 
+ 18\,\mathrm{ADP} + 
&18\,\mathrm{P}_i 
+ 12\,\mathrm{NADP}^+.
\end{align}
Glucose is polymerized into cellulose by dehydration:
\begin{align}
n\,\mathrm{C}_6\mathrm{H}_{12}\mathrm{O}_6 
&\rightarrow (\mathrm{C}_6\mathrm{H}_{10}\mathrm{O}_5)_n 
+ n\,\mathrm{H}_2\mathrm{O}.
\end{align}
These polymers form the primary structure of wood (secondary xylem), alongside lignin and hemicellulose.
\noindent\textbf{Oxygen Source Identification via Isotope Labeling}\\
The $^{18}\mathrm{O}$ labeling experiments by Ruben and Kamen (1941) definitively established oxygen source separation:
\noindent\textit{Water source test:}
\begin{align}
\mathrm{CO}_2 + \mathrm{H}_2^{18}\mathrm{O} + h\nu 
&\rightarrow [\mathrm{CH}_2\mathrm{O}] + ^{18}\mathrm{O}_2
\end{align}
\textit{Result:} Heavy oxygen ($^{18}\mathrm{O}$) appeared exclusively in released $\mathrm{O}_2$, not in organic products.
\noindent\textit{$\mathrm{CO}_2$ source test:}
\begin{align}
\mathrm{C}^{18}\mathrm{O}_2 + \mathrm{H}_2\mathrm{O} + h\nu 
&\rightarrow [\mathrm{CH}_2^{18}\mathrm{O}] + \mathrm{O}_2
\end{align}
\textit{Result:} Heavy oxygen ($^{18}\mathrm{O}$) appeared in biomolecules, not in released $\mathrm{O}_2$.
\noindent\textbf{Energy Storage Density}\\
Wood represents highly concentrated solar energy storage:
\begin{itemize}[leftmargin=*]
  \item \textbf{Energy density:} 16–20 MJ/kg (dry wood)
  \item \textbf{Solar capture efficiency:} 1–3\% of incident radiation
  \item \textbf{Mature tree storage:} 50–100 GJ total (accumulated over decades)
  \item \textbf{Photon requirement:} ~8–10 photons per $\mathrm{CO}_2$ molecule fixed
\end{itemize}
This energy density approaches that of fossil fuels, demonstrating that photosynthesis creates a highly efficient biological battery.
\noindent\textbf{Quantitative Mass Accumulation}\\
For annual NPP of $10^4\,\mathrm{kg/ha}$ dry biomass (50\% carbon):
\begin{align}
\text{$\mathrm{CO}_2$ fixed} &= 18.4\,\mathrm{tonnes}\,\mathrm{CO}_2/\mathrm{ha}/\mathrm{year} \\
\text{Per tree (100/ha)} &= 184\,\mathrm{kg}\,\mathrm{CO}_2/\mathrm{year}
\end{align}
Over 50 years, each tree accumulates ~2.5 tonnes carbon, corresponding to ~5 tonnes total dry biomass — consistent with mature forest measurements.
\noindent\textbf{Elemental Mass Contribution}\\
Typical dry mass composition:
\begin{itemize}[leftmargin=*]
  \item Carbon: 45–50\% (from atmospheric $\mathrm{CO}_2$)
  \item Oxygen: 40–45\% (exclusively from $\mathrm{CO}_2$)
  \item Hydrogen: ~6\% (from water)
  \item Minerals: 1–5\% (from soil: N, P, K, Ca, etc.)
\end{itemize}

\noindent\textbf{References:}\\
Farquhar, G. D., von Caemmerer, S. (1980). \textit{Planta}, \textbf{149}, 78–90.\\
Taiz, L., Zeiger, E. (2010). \textit{Plant Physiology}.
\end{technical}
