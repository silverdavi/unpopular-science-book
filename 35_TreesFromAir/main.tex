A tree's material body, the wood, leaves, and branches it accumulates year by year, is not extracted from the ground in the way stones or metals are quarried. Its dry mass arises from elements that were once distributed in dilute form throughout the atmosphere and hydrosphere. The key components of this mass, carbon, oxygen, and hydrogen, enter through invisible flows: air, water, and sunlight. A tree is built from what passes through it.

Although visually and mechanically tied to the soil, a tree records processes that unfold mostly above ground. The mass that persists after all water is removed, the dry matter, is composed primarily of carbon atoms originally fixed from atmospheric \(\mathrm{CO}_2\). The atoms were drawn down through the stomata of leaves, diffused through mesophyll tissue, and incorporated into sugar molecules via light-powered biochemical cycles. The resulting solid form is the memory of many such molecular events.

The notion that trees "grow out of the earth" conflates anchorage with origin. The soil does provide essential ions and mechanical stability, but its contribution to the actual mass is minor. Most of what endures in a dried trunk, cellulose, lignin, hemicellulose, was once part of the air surrounding it. The verticality of a tree, its rise toward the sky, is materially made possible by the intake of that sky's gaseous contents.

The central process enabling this conversion is photosynthesis. It is not a singular reaction, but a layered sequence of energy transduction and molecular reconfiguration. The first phase occurs in the chloroplasts of leaf cells, where chlorophyll pigments absorb incoming photons. The photons elevate electrons to higher energy states, dislodging them from their atomic orbitals and initiating a cascade of electron transfers through the thylakoid membrane.

Oxygen-producing photosynthesis altered Earth's atmosphere and biosphere. When it first evolved in cyanobacteria around 2.5 billion years ago, it triggered the Great Oxygenation Event — a transformation that poisoned most existing anaerobic life but enabled the eventual emergence of complex organisms. The oxygen released by water-splitting is a byproduct that reshaped planetary chemistry. Every breath taken by an animal, every flame that burns, every rusting of iron depends on this ancient process continuing in plant chloroplasts. Trees are participants in a planetary-scale atmospheric engine that has operated continuously for billions of years.

The chain of transfers generates two critical energy carriers: ATP (adenosine triphosphate) and NADPH (nicotinamide adenine dinucleotide phosphate). The molecules store the electromagnetic energy harvested from light and shuttle it into the chemical domain. In the aqueous interior of the chloroplast, the stroma, the stored energy is used to convert inorganic carbon into organic intermediates.

The source of oxygen reveals molecular accounting. When water molecules are split in photosystem II, their oxygen atoms are released directly to the atmosphere as O₂ gas. Isotope labeling experiments using ¹⁸O-enriched water demonstrated that the heavy oxygen appeared in the released gas, not in the organic products. The oxygen atoms incorporated into cellulose and other biomolecules originate from CO₂. Every molecule of atmospheric oxygen released by plants represents a water molecule that was split to extract electrons, while the oxygen in wood records the atmospheric carbon that was fixed. The process literally separates air from water at the atomic level.

The fixation of carbon takes place in the Calvin–Benson cycle. Atmospheric \(\mathrm{CO}_2\) diffuses into leaf tissue and reacts with ribulose bisphosphate, a five-carbon sugar, under the catalytic action of the enzyme Rubisco. The resulting six-carbon intermediate is immediately split into three-carbon molecules, triose phosphates, that serve as building blocks for carbohydrates. The triose units are reassembled into glucose and other hexoses, which in turn feed biosynthetic pathways across the plant.

The energy accounting reveals efficiency patterns. Sunlight delivers approximately 1,000 watts per square meter on a clear day — but plants capture only 1-3\% of this energy in chemical bonds. What is captured becomes concentrated: each kilogram of dry wood stores about 16-20 megajoules of energy, roughly equivalent to the combustion energy of natural gas. A single mature tree may contain 50-100 gigajoules of stored solar energy, accumulated over decades of photosynthetic capture. Millions of individual photons contribute quantum energy to the construction of molecular architecture that can persist for centuries.

Once synthesized, the sugars are exported from the site of fixation. Through the phloem, a network of conductive tissues, they are distributed to growing regions: root tips, shoot apices, developing leaves, and the vascular cambium. At the cambium, a cylindrical layer of dividing cells just beneath the bark, the imported carbohydrates are used to construct macromolecules.

Cellulose, hemicellulose, and lignin form the principal constituents of wood. Cellulose assembles into long, unbranched chains that crystallize into fibrils, giving tensile strength to cell walls. Hemicellulose binds the fibrils into a cohesive matrix, while lignin, a complex phenolic polymer, fills the spaces between them, adding compressive strength and water resistance. The polymers do not exist as free-floating products; they are laid down in precise geometric arrangements within the expanding walls of growing cells.

Tree growth is not an inflationary process — it is a spatially organized addition of new cells, localized in specific generative zones. At the vascular cambium, cell division proceeds laterally, producing xylem cells toward the center and phloem cells outward. The radial expansion creates the familiar pattern of growth rings. Each ring corresponds to a cycle of photosynthetic capture and biosynthetic deposition.

Elongation occurs at the apical meristems, where undifferentiated cells divide and specialize into tissue types. The regions at the tips of roots and shoots coordinate patterning, orientation, and organogenesis. As cells expand and walls thicken, the imported sugars are converted into permanent form. The tree grows by building new matter, atom by atom, layer by layer, guided by developmental constraints, resource availability, and physical principles of load-bearing and fluid transport.

Hydrogen atoms in the biomass originate from water. Water is absorbed by roots and pulled upward through the xylem under tension. Though over 99\% of it eventually evaporates through stomatal pores, a small fraction is chemically incorporated into organic molecules. The hydrogen forms part of the fixed material, bound into carbohydrates and lipids.

Water's functional role extends beyond hydrogen donation. It serves as a solvent for ions, a medium for transport, and a buffer against temperature fluctuations. It enables the tree's biochemical metabolism without being a primary contributor to its dry weight. What remains after desiccation is not water but the elements it helped mobilize and bind.

Oxygen atoms in biomass come from \(\mathrm{CO}_2\). During photosynthesis, water molecules are split to provide electrons, but their oxygen atoms are released directly to the atmosphere as \(\mathrm{O}_2\) gas. The oxygen atoms incorporated into cellulose, forming hydroxyl, carboxyl, and ether linkages, originate from the atmospheric carbon dioxide that was fixed. The high oxygen content of wood, about 40 to 45 percent by weight, is a direct record of atmospheric \(\mathrm{CO}_2\) that was captured and converted into solid form.

Mineral ions absorbed from the soil are essential but contribute little to total mass. Nitrogen, phosphorus, potassium, calcium, magnesium, and micronutrients serve catalytic and regulatory roles. They enable enzymatic function, membrane potential maintenance, and nucleic acid stability. Their aggregate proportion in dry matter is often less than 5 percent. They are facilitators, not substrates.

When all water is removed from a tree, what remains is a carbon-rich composite of organic polymers. Cellulose (C\(_6\)H\(_{10}\)O\(_5\))\(_n\), lignin, and related molecules form a lattice of energy-stored mass, chemically stabilized and mechanically resilient. The material records a long history of atomic coordination: gases captured, energy converted, atoms arranged.

The resulting form is not a byproduct of soil consumption. It is a physical record of electromagnetic energy transformed into covalent bonds. Each gram of wood contains photons absorbed years prior, transmuted into molecular architecture. The tree's height, girth, and density are signatures of what it has assembled from air and light.

The transformation exhibits chemical symmetry. Photosynthesis builds sugar units from atmospheric inputs, 6\(\mathrm{CO}_2\) + 6\(\mathrm{H}_2\mathrm{O}\) + light energy → \(\mathrm{C}_6\mathrm{H}_{12}\mathrm{O}_6\) + 6\(\mathrm{O}_2\), which are then polymerized into cellulose by removing water: n(\(\mathrm{C}_6\mathrm{H}_{12}\mathrm{O}_6\)) → (\(\mathrm{C}_6\mathrm{H}_{10}\mathrm{O}_5\))\(_n\) + n\(\mathrm{H}_2\mathrm{O}\). When wood burns, the process reverses exactly: (\(\mathrm{C}_6\mathrm{H}_{10}\mathrm{O}_5\))\(_n\) + 6n\(\mathrm{O}_2\) → 6n\(\mathrm{CO}_2\) + 5n\(\mathrm{H}_2\mathrm{O}\) + heat. The stored solar energy is released, and every atom returns to its original atmospheric or aqueous state. The carbon dioxide and water vapor that rise from the flame are identical to the molecules that entered the tree decades earlier. Only the temporal direction distinguishes construction from destruction. A tree is a temporary configuration of atmospheric components, held together by captured light.

Richard Feynman remarked that trees are "made of air," not as a metaphor but as a physical statement. When a tree burns, the carbon returns to the atmosphere, and the stored sunlight is released as heat. What remains as ash is the minor residue of earthbound elements. Trees are not extracted from ground but condensed from flow. They are equilibrium-defying artifacts of solar patterning and atmospheric participation.

\newpage
\vfill
\begin{center}
\begin{tikzpicture}[remember picture,overlay]
    \fill[green!8] (current page.north west) rectangle (current page.south east);
    \draw[green!40, line width=3pt, rounded corners=20pt] 
        ([xshift=1.5cm,yshift=-1.5cm]current page.north west) rectangle 
        ([xshift=-1.5cm,yshift=1.5cm]current page.south east);
    % Add some decorative elements
    \foreach \i in {1,...,12} {
        \fill[green!50] ([xshift=\i*1.5cm-0.5cm,yshift=-0.3cm]current page.north west) circle (0.1cm);
        \fill[green!50] ([xshift=\i*1.5cm-0.5cm,yshift=0.3cm]current page.south west) circle (0.1cm);
        \fill[green!70] ([xshift=\i*1.5cm-0.3cm,yshift=-0.1cm]current page.north west) circle (0.05cm);
        \fill[green!70] ([xshift=\i*1.5cm-0.3cm,yshift=0.1cm]current page.south west) circle (0.05cm);
    }
\end{tikzpicture}

\fontfamily{cmss}\selectfont
\begin{minipage}{0.85\textwidth}
\begin{multicols}{2}
{\Huge\textbf{The Growing Tree}}\\[0.3em]
{\large\textit{(NOT by Shel Silverstein)}}

\vspace{1.5em}

{\large Once there was a tree,\\
and she loved a little boy.}

\vspace{0.5em}

{\large Every day the boy would come—\\
gather her leaves to make crowns,\\
climb her trunk,\\
swing from her branches,\\
eat apples,\\
and rest in her shade.}

\vspace{0.5em}

{\large And the boy loved the tree.\\
And the tree was happy.}

\vspace{0.5em}

{\large But time passed,\\
and the boy grew older.\\
The tree often stood alone.}

\vspace{0.5em}

{\large One day the boy came back.\\
The tree said,\\
"Come, climb my trunk, swing, eat, and be happy."}

\vspace{0.5em}

{\large "I'm too big to play," said the boy.\\
"I want money, to buy things and have fun."}

{\large "I don't have money," said the tree,\\
"but you can take some apples—\\
sell a few, share a few, and plant one or two."}

\vspace{0.5em}

{\large And the boy did.\\
And the tree was happy.}

\vspace{0.5em}

\columnbreak

{\large Years passed. The boy returned.\\
"I want a house," he said,\\
"for warmth, for family."}

\vspace{0.5em}

{\large "I don't have a house," said the tree,\\
"but take my fallen branches.\\
Leave enough for me to grow."}

\vspace{0.5em}

{\large And the boy did.\\
And the tree was happy.}

\vspace{0.5em}

{\large More time went by.\\
The boy returned, older.\\
"I want a boat to go far away."}

\vspace{0.5em}

{\large "Some of my thicker branches grew wild," said the tree.\\
"You can use them."}

\vspace{0.5em}
{\large And the boy did.\\
And the tree was happy.}

{\large After many years, the boy returned, tired.\\
\vspace{0.5em}

"I don't need much now," he said,\\
"just a place to rest."}

{\large The tree said,\\
\vspace{0.5em}

"Come sit.\\
There is shade again.\\
And I'm glad you're here."}

{\large And the boy did.\\
\vspace{0.5em}

and the boy was happy.}

\vspace{1.5em}

{\large And the tree was happy,
}

\vspace{0.5em}
\end{multicols}
\end{minipage}
\end{center}
\vfill

