\begin{SideNotePage}{
  \textbf{Photosynthesis and Carbon Fixation:} \par
  This diagram illustrates the process by which trees build their mass from atmospheric CO₂ rather than soil nutrients. The top section shows the molecular mechanism of photosynthesis: light energy driving the conversion of carbon dioxide and water into glucose through the Calvin-Benson cycle. Chloroplasts in leaf cells capture photons to power the fixation of atmospheric carbon into organic compounds. The bottom section demonstrates the carbon flow: from diffuse CO₂ in the atmosphere, through stomatal uptake, into the structural polymers (cellulose, lignin, hemicellulose) that comprise wood. Each ring of tree growth represents a year's accumulation of atmospheric carbon, transformed by solar energy into solid biomass. As Feynman noted, trees are literally "made of air" — their dry weight consists primarily of carbon atoms that were once distributed throughout the atmosphere.
}{35_TreesFromAir/35_ From Air to Arbor.pdf}
\end{SideNotePage}