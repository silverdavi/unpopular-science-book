\begin{SideNotePage}{
  \textbf{Top (Maxwell's Demon):} The classic Maxwell’s Demon thought experiment. A demon monitors molecules in a gas and selectively opens a door to let fast-moving (hot) molecules through in one direction and slow-moving (cold) molecules through the other. Over time, this creates a temperature gradient without performing mechanical work, seemingly violating the second law of thermodynamics. \par
  \textbf{Bottom Left (Work Cost):} First resolution—work-based measurement cost (Szilard, Brillouin). Determining each molecule’s speed and deciding whether to open the door requires physical measurement processes that generate entropy or consume work, preventing a net decrease in total entropy. \par
  \textbf{Bottom Right (Memory Erasure):} Second resolution—memory erasure cost (Landauer, Bennett). Even if measurement and control were performed reversibly with no work cost, the demon’s finite memory would eventually fill. Erasing past measurement records to store new data increases entropy by at least $k \ln 2$ per bit erased (Landauer’s Principle), ensuring the second law remains intact. Either way, the apparent paradox dissolves when information-processing steps are properly accounted for.
}{45_MaxwellDemon/45_ Entropy_s Gatekeeper.pdf}
\end{SideNotePage}
