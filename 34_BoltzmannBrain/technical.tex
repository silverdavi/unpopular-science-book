\begin{technical}
{\Large\textbf{Entropy Bounds and Fluctuation Probabilities}}\\

\noindent\textbf{Core Result}\\
Comparing probabilities: a 20-second brain fluctuation vs. 13.8 billion years of cosmic evolution:
\[
\frac{P(\text{Brain})}{P(\text{Universe})} \sim \exp(10^{120} - 10^{50}) \sim e^{10^{120}}
\]
The brain wins by a factor that dwarfs all numbers in physics.

\noindent\textbf{De Sitter Equilibrium}\\
The de Sitter horizon carries thermodynamic entropy:
\[
S_\text{dS} = \frac{\pi c^3 R^2}{\hbar G} \sim 10^{120} \text{ (Planck units)}
\]
This sets the equilibrium baseline for our universe.

\noindent\textbf{Fluctuation Probability}\\
Any departure from equilibrium is exponentially suppressed:
\[
P(\mathcal{F}) \propto \exp(-\Delta S), \quad \Delta S = S_\text{dS} - S(\mathcal{F})
\]

\noindent\textbf{Standard Cosmological Evolution}\\
A universe starting from Big Bang requires extremely low initial entropy:
\[
S_\text{early} \sim 10^{10} \quad \Rightarrow \quad \Delta S_\text{cosmo} \sim 10^{120}
\]
\[
P_\text{cosmo} \sim \exp(-10^{120})
\]

\noindent\textbf{Boltzmann Brain Fluctuation}\\
A minimal conscious configuration needs far less entropy reduction:
\[
\Delta S_\text{BB} \sim 10^{40} \text{ to } 10^{50}
\]
\[
P_\text{BB} \sim \exp(-10^{50})
\]

\noindent\textbf{Probability Ratio}\\
\[
\frac{P_\text{BB}}{P_\text{cosmo}} \sim \exp(10^{120} - 10^{50}) \sim \exp(10^{120})
\]
Boltzmann Brains dominate by this incomprehensible factor.

\noindent\textbf{Duration Scaling}\\
Entropy cost increases with coherence time:
\begin{align}
\tau \sim 1\,\mu\text{s}: & \quad \Delta S \sim 10^{40} \\
\tau \sim 1\,\text{ms}: & \quad \Delta S \sim 10^{45} \\
\tau \sim 1\,\text{s}: & \quad \Delta S \sim 10^{60}
\end{align}
Even second-long brains remain exponentially more likely than cosmic evolution.

\noindent\textbf{Time Scales}\\
First Boltzmann Brain appearance (Page):
\[
t_\text{BB} \sim e^{10^{120}} \text{ years}
\]
Vast yet infinitesimal compared to eternity. In eternal de Sitter, such fluctuations repeat infinitely.

\noindent\textbf{References:}\\
Dyson, L., Kleban, M., Susskind, L. (2002). \textit{Disturbing Implications of a Cosmological Constant}.\\
Page, D. N. (2006). \textit{Is Our Universe Likely to Decay within 20 Billion Years?} arXiv:hep-th/0610079.
\end{technical}
