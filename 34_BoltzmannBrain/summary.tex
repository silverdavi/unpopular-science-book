The Boltzmann Brain paradox shows that statistical mechanics predicts a disturbing outcome: random fluctuations in a high-entropy universe would produce isolated conscious entities more frequently than entire ordered universes like ours. A single brain with false memories requires orders of magnitude fewer unlikely coincidences than 13.8 billion years of cosmic evolution. These hypothetical observers would experience coherent thoughts and apparent histories, yet exist only as momentary statistical fluctuations. It is not easy to dismiss this proposterous theory based on scientific reasoning alone.
