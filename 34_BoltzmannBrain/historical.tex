\begin{historical}
In 1895, Ludwig Boltzmann proposed that the observed low entropy of the universe might arise as a rare fluctuation within a larger equilibrium state. His goal was to reconcile the second law of thermodynamics with the possibility of eternal time: if the universe is statistically dominated by high-entropy configurations, then any low-entropy region — such as our observable cosmos — would have to be an exceptional, temporary departure.

This explanation faced immediate challenges. Boltzmann's assistant Schuetz pointed out the flaw: smaller fluctuations are exponentially more probable than large ones. If we explain our ordered universe as a fluctuation, why did it fluctuate so much more than necessary? It would be more likely for a single galaxy, solar system, or a single observer — complete with illusory memories of a larger cosmos — to emerge briefly from equilibrium. The entropy required for a functioning brain lasting seconds is negligible compared to that needed for billions of years of cosmic evolution. Arthur Eddington later quantified this disparity, showing that Boltzmann's hypothesis made our observations exponentially unlikely over finite timescales.

The problem lay dormant until the late 20th century, when it re-emerged. The 1998 discovery of cosmic acceleration implied a positive cosmological constant, suggesting our universe would expand forever into a de Sitter state. Sean Carroll, Lisa Dyson, and Matthew Kleban demonstrated in 2002 that such universes face Boltzmann's original problem in its extreme form: eternal de Sitter space acts as a thermal bath that will fluctuate into any possible configuration, with smaller fluctuations exponentially dominating larger ones.

The "Boltzmann Brain" problem — named by Andreas Albrecht and Lorenzo Sorbo — became recognized not as philosophical speculation but as a philosophical threat to cosmology. If the universe lasts long enough, and if thermal or quantum fluctuations occur eternally, then you are likely a momentary self-aware configuration with a fabricated past — including false memories of other people. Boltzmann's original idea became a cautionary tale about reasoning backward from present experience in a probabilistic universe.

The crisis deepened as cosmologists realized that avoiding Boltzmann Brains required either abandoning eternal inflation, modifying the measure of observers in cosmology, or accepting that something was wrong with their models. Don Page calculated that in de Sitter space, the time until a Boltzmann Brain fluctuates into existence is approximately $e^{10^{120}}$ years — vast, yet infinitesimal compared to eternity. The problem forces a stark choice: reject our best model of dark energy, accept solipsism, or find new principles that privilege causal observers.

Similar logic appears outside physics. Young-Earth creationists propose that fossils, rock strata, and starlight were created in transit — a universe with apparent but not actual age. This bypasses historical causality in favor of constructed appearance, identical to the Boltzmann Brain scenario, though motivated by theology rather than thermodynamics. Both sever the link between evidence and history, rendering empirical investigation meaningless.
\end{historical}
