\begin{SideNotePage}{

  \textbf{Top (Prime Distribution):} Natural numbers 1-25 with primes (red) and composites (teal), showing twin prime pairs like (3,5), (5,7), (11,13), (17,19).
  \newline
  \textbf{Second (Gap Sizes):} Consecutive primes with gap sizes marked by blue arrows, from the minimal gap of 2 to larger separations like 8 between 23 and 31.
  \newline
  \textbf{Third (Gap Analysis):} Scatter plot of prime gaps $p_{n+1} - p_n$ versus $p_n$ for the first 10,000 primes, showing most gaps are small with rare large exceptions. Reference lines mark key bounds: twin primes (gap = 2), Polymath8's refined bound (246), and Zhang's original breakthrough (70 million). Infinitely many gaps stay below some fixed bound, guaranteeing points always appear below these lines.
  
  Note the key distinction: for small gaps we seek absolute bounds (fixed numbers like 246), while for large gaps the bounds are functions of $p_n$ that grow as primes get larger.
  
  \textbf{Bottom (Maximum Gaps):} How large can prime gaps become? This shows the largest observed gaps (light blue) compared to theoretical predictions. The red lines show upper bound conjectures for maximum gap size, while colored lower bounds prove that gaps must occasionally be large. The scatter demonstrates that while most gaps are small relative to the local prime density, some gaps are much larger than the typical spacing in their region.
}{08_BoundedPrimeGaps/08_ 70-Million Steps for Primes, One Giant Leap for Number Theory.pdf}
\end{SideNotePage}