\begin{technical}
{\Large\textbf{Bounded Gaps Between Primes}}

\textbf{1. Statement of Result and Outline of Method}\\[0.5em]
In 2013, Yitang Zhang proved that there exists a constant \( N \) such that infinitely many prime pairs \( (p, q) \) satisfy \( q - p \leq N \). The initial bound was \( N < 7\times10^7 \). Zhang's approach refined the Goldston–Pintz–Yıldırım (GPY) method through two components:

\begin{itemize}[leftmargin=*]
    \item \textbf{Distribution in Arithmetic Progressions:} Primes remain evenly distributed across residue classes beyond the Bombieri–Vinogradov range.
    \item \textbf{Weighted Sieve:} Modified sieve to detect multiple primes within admissible tuples \( n + h_i \).
\end{itemize}

\vspace{0.7em}
\textbf{2. Sieve Setup and Admissibility}\\[0.5em]
A tuple \( \mathcal{H} = \{h_1, \dots, h_k\} \) is admissible if for every prime \( p \), the set \( \mathcal{H} \mod p \) does not cover all residue classes modulo \( p \).

The GPY strategy constructs a weighted sum:
\[
S(n) := \left( \sum_{i=1}^k \Lambda(n + h_i) \right) w(n),
\]
where \( \Lambda \) is the von Mangoldt function and \( w(n) \) is a smooth function supported on \( n \in [x, 2x] \). The weights emphasize values where multiple \( n + h_i \) are likely prime:
\[
\sum_{n} S(n) = \sum_{n} \left( \sum_i \Lambda(n + h_i) \right) w(n).
\]
If this exceeds the random baseline, then for some \( n \), at least two \( n + h_i \) are prime.

\vspace{0.7em}
\textbf{3. Example}\\[0.5em]
The admissible set \( \{0, 2, 6\} \) avoids covering all residue classes modulo any prime. For \( n = 5 \), we get \( \{5, 7, 11\} \) — three primes. The method proves such cases occur infinitely often.

\vspace{0.7em}
\textbf{4. Zhang’s Level of Distribution}\\[0.5em]
Zhang proved primes remain equidistributed up to moduli \( q \le x^\theta \) for \( \theta > 1/2 \), surpassing the Bombieri–Vinogradov barrier (\( \theta = 1/2 \)).

Define:
\[
\theta(x; q, a) = \sum_{\substack{p \le x \\ p \equiv a \,(\mathrm{mod}\,q)}} \log p.
\]
The deviation of \( \theta(x; q, a) \) from \( x / \phi(q) \) remains small across a wide range of moduli, enabling uniform error control. Zhang bypassed the Elliott–Halberstam conjecture by achieving a weaker but sufficient level of distribution.

\vspace{0.7em}
\textbf{5. Maynard’s Modification and Polymath Refinements}\\[0.5em]
Maynard introduced new sieve weights detecting primes in admissible tuples without requiring \( \theta > 1/2 \), simplifying the construction.

The Polymath8 project refined and extended both approaches:
\begin{itemize}[leftmargin=*]
    \item \textbf{Polymath8a:} Improved error analysis, reduced \( N \) to 4,680.
    \item \textbf{Maynard's Variant:} Lowered \( N \) further, generalized to \( m \) primes in bounded intervals.
    \item \textbf{Polymath8b:} Reduced bound below 250.
\end{itemize}

\vspace{0.5em}
\textbf{References:}\\[0.3em]
Zhang, Y. (2014). Bounded gaps between primes. \textit{Annals of Mathematics}, 179(3), 1121–1174.\\
Maynard, J. (2015). Small gaps between primes. \textit{Annals of Mathematics}, 181(1), 383–413.

\end{technical}
