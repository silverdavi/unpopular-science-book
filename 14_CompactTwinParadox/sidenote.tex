\begin{SideNotePage}{
  \textbf{Top (Twin Paradox Variants):} Variants of the twin paradox. The first shows the standard scenario: one twin remains on Earth while the other travels to a distant star and back at relativistic speed, experiencing less proper time due to time dilation and the non-inertial turnaround. The second panel sets the paradox in a compactified spacetime, where space is wrapped into a cylinder. Here, a twin can traverse the compact dimension at constant velocity —-- never accelerating or changing frames —-- yet still accumulate less proper time than the twin who remains stationary. \par
  \textbf{Bottom (Non-orientable Topology):} Non-orientable topology and global spacetime orientation. Depicted as a Möbius-like strip, this represents a spacetime where following a continuous timelike path can bring a traveler back to their starting point with their internal orientation flipped.
}{14_CompactTwinParadox/14_ A Circle of Time.pdf}
\end{SideNotePage}
