\begin{historical}
Albert Einstein’s 1905 introduction of special relativity redefined time and motion, emphasizing the role of inertial frames and leading to the now-familiar notion of time dilation. A few years later, Hermann Minkowski introduced a four-dimensional spacetime geometry, allowing relativistic effects to be understood geometrically. The so-called “twin paradox” became an iconic thought experiment illustrating the asymmetric aging of two travelers, one of whom undergoes acceleration.

Parallel to these developments, mathematicians in the late 19th and early 20th centuries, including Eugenio Beltrami, August Möbius, and Felix Klein, investigated the implications of identifying the edges of geometric surfaces. These ideas laid the groundwork for understanding non-orientable and compact spaces, such as the Möbius strip and Klein bottle. Henri Poincaré introduced foundational concepts in topology that allowed physicists to study global properties of space beyond Euclidean structure.

In 1949, Kurt Gödel proposed a rotating cosmological model that permitted closed timelike curves, demonstrating that general relativity allowed for causal paths that returned to earlier points in time. Later work by John Wheeler, Bryce DeWitt, and others examined exotic topologies in general relativity, where spacetime could be globally identified in non-intuitive ways, even if it remained locally flat.
\end{historical}
