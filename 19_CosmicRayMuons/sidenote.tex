\begin{SideNotePage}{

    \textbf{Top (Muon Time Dilation):} Cosmic rays strike the upper atmosphere, producing showers of muons. At rest, muons have a lifetime of only about 2.2 microseconds, which should not allow many ofthem to reach detectors on the surface. However, due to relativistic time dilation, their internal clocks run slower from the Earth’s frame of reference, allowing far more muons to survive and be detected than expected under classical assumptions. \par
    \textbf{Bottom (Neutrino Flavor Oscillations):} When produced in cosmic-ray interactions, neutrinos come in three flavors: electron, muon, and tau. Early experiments detected only about two-thirds of the expected muon neutrinos, a puzzle known as the solar neutrino problem. The resolution came from neutrino oscillations: as neutrinos travel, they can change “hats” between flavors, converting among types. This explains why detectors only saw a fraction of the muon neutrinos originally predicted. \par

}{19_CosmicRayMuons/19_ Consider the Muon_s PoV.pdf}
\end{SideNotePage}
