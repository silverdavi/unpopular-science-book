\begin{historical}
Carl D. Anderson's cosmic ray research in the early 1930s had already yielded the positron, the first discovered antiparticle, earning him the Nobel Prize in 1936. Working at Caltech with graduate student Seth Neddermeyer, Anderson continued photographing particle tracks in cloud chambers placed in high-altitude locations and deep underground to study cosmic radiation. In late 1936, they noticed tracks that curved less than electrons in magnetic fields but more than protons — evidence of a particle with intermediate mass.

The discovery paper, published in 1937, was cautious. The tracks suggested a mass roughly 200 times that of the electron, but the particle's identity remained unclear. Anderson and Neddermeyer called it a "mesotron," a name reflecting its intermediate mass between electrons and protons. Physicist Isidor Rabi, upon hearing of the discovery, reportedly quipped: "Who ordered that?" The particle seemed superfluous — it played no obvious role in atomic structure or known nuclear processes.

Theoretical physicists initially tried to identify the mesotron with Hideki Yukawa's predicted meson, which was supposed to mediate the strong nuclear force binding protons and neutrons. Yukawa had calculated in 1935 that such a particle should have a mass around 200 electron masses and interact strongly with nuclei. But experiments quickly revealed problems: the mesotron penetrated matter far too easily and interacted too weakly with atomic nuclei to be Yukawa's particle. By the early 1940s, physicists realized they had found something unexpected.

The confusion persisted until 1947, when Cecil Powell, César Lattes, and Giuseppe Occhialini, using improved photographic emulsion techniques at high altitude, discovered the pion — the true Yukawa particle. They showed that pions produced in cosmic ray collisions decay into the lighter mesotron. The nomenclature shifted: pions became "pi-mesons" and the lighter particle was eventually renamed the muon, recognizing it as a heavier cousin of the electron rather than a nuclear force carrier.

Meanwhile, physicists studying cosmic ray showers noticed an anomaly. Muons produced 10–20 kilometers above Earth's surface were reaching sea-level detectors in numbers far exceeding expectations. With a measured lifetime of 2.2 microseconds at rest, a muon traveling even at light speed should cover less than 700 meters before decaying. Yet detectors routinely observed muons at sea level, having traversed more than ten kilometers through the atmosphere.

Bruno Rossi, an Italian physicist who had fled fascist Italy and was working at Los Alamos and later MIT, recognized the discrepancy. In 1940–1941, Rossi and David B. Hall conducted systematic measurements comparing muon counts at different altitudes. Their data, published in 1941, showed that relativistic time dilation explained the observations: muons traveling at velocities exceeding 0.99$c$ experience dilated lifetimes, allowing them to reach Earth's surface before decaying.

The experiment provided one of the first natural confirmations of special relativity outside of controlled laboratory settings. Unlike synchronized-clock experiments or particle accelerator measurements, atmospheric muons offered a passive test using naturally occurring particles traveling macroscopic distances. The agreement between predicted and observed muon flux at various altitudes removed lingering doubts about the physical reality of relativistic time transformation.

By the 1950s and 1960s, muons had transitioned from mysterious intruders to standard tools in particle physics. Their relatively long lifetime, clean decay signature, and penetrating power made them invaluable for testing quantum electrodynamics, probing weak interaction dynamics, and serving as high-energy probes in collider experiments. The particle that seemed to serve no purpose became central to understanding the structure of matter.
\end{historical}
