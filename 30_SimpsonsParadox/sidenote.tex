\begin{SideNotePage}{
  \textbf{Top (Simpson’s Paradox – Admissions):}  
  At the aggregate level, a higher percentage of men are admitted compared to women, suggesting gender bias. But disaggregating by department reveals that women applied more often to highly competitive departments with low acceptance rates, while men applied to departments with higher admission rates. Within departments, women were often admitted at equal or higher rates. This reversal — where a trend in subgroups contradicts the overall trend — is known as \emph{Simpson’s Paradox}. \par
  
  \textbf{Bottom (Survivorship Bias – Cancer in Smokers):}  
  Cancer incidence increases sharply with age, but smokers have higher early mortality from other causes. As a result, fewer smokers survive into the high-risk age brackets where cancer becomes common. This skews the population-level data, making it seem as though smokers have lower cancer rates than non-smokers. The truth is hidden by \emph{survivorship bias}: smokers often die before cancer can occur. \par
  
}{30_SimpsonsParadox/30_ Divide & Conquer.pdf}
\end{SideNotePage}

