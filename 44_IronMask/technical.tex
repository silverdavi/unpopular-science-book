\begin{technical}
{\Large\textbf{Construction of Historical Fiction}}\\[0.2em]

This technical study reconstructs the historical profile of the state prisoner later mythologized as “the Man in the Iron Mask.” It evaluates the verifiability and continuity of archival sources — primarily the correspondence between Louvois and Saint-Mars, prison registers, burial records, and journals kept by staff — and contrasts them with later literary augmentations by Voltaire and Dumas. Emphasis is placed on methodological tools used to separate historical plausibility from ideological fiction.

\textbf{The Documentary Spine: Continuity and Custody}\\[0.3em]
The prisoner’s existence is traceable through a continuous chain of archival records from 1669 to 1703. The arrest order — a \textit{lettre de cachet} dated 19 July 1669 — names “Eustache Dauger,” ordering his confinement under Saint-Mars at Pignerol. Subsequent letters from Louvois, though avoiding names, refer to “the prisoner whom you know,” with consistent logistical details.

Each transfer — Pignerol (1669), Exilles (1681), Sainte-Marguerite (1687), Bastille (1698) — parallels Saint-Mars’s own promotions. Du Junca’s Bastille journal confirms the masked prisoner’s arrival in 1698 and death in 1703. He was buried under the alias “Marchioly.” No court records, criminal charges, or trial transcripts exist over this 34-year period.

\textbf{Evidentiary Elimination: Mask, Alias, and Erasure}\\[0.3em]
The famous mask appears only once in primary sources — du Junca’s 1703 journal — described as “black velvet.” There is no mention of iron. Its use is restricted to moments of public exposure, such as transport or chapel attendance. Early orders emphasize secrecy and isolation but not continuous masking.

The burial alias “Marchioly” evokes “Mattioli,” a separate prisoner captured in 1679 and dead by 1694. However, Dauger’s imprisonment begins a decade earlier and ends nine years later. No source places Mattioli at the Bastille. The alias appears to be administrative misdirection rather than a clue to true identity.

Post-mortem protocols — including burning of bedding and wall-scraping — are recorded in Saint-Mars’s letters and deviate sharply from standard Bastille procedure, indicating deliberate suppression rather than routine sanitation.

\textbf{Historiographical Filtering: Dauger, Mattioli, and Royal Invention}\\[0.3em]
Three major identification theories remain:

1. \textbf{Dauger} is named in the 1669 warrant and confirmed as Fouquet’s valet. Some historians argue that he may have uncovered sensitive financial or political information, warranting extreme secrecy. This theory aligns with timeline, treatment, and the absence of formal charges.

2. \textbf{Mattioli}, although a real prisoner, is chronologically misaligned. He was arrested in 1679, died in 1694, and is never recorded in the Bastille. The alias “Marchioly” is insufficient for identification given the common use of placeholders.

3. \textbf{Royal identity hypotheses} — twin, brother, or secret heir — have no archival foundation. No court record, diplomatic note, or genealogical account supports them. These theories originate with Voltaire’s speculative writings and were dramatized by Dumas. They reflect political allegory, not evidence-based history.

\textbf{Conclusion: Historical Confinement as Narrative Substrate}\\[0.3em]
The verified record depicts a man systematically anonymized, transferred, masked on occasion, and ultimately erased from memory. These measures suggest not noble origin but sensitive knowledge. Later literary versions reframe bureaucratic silencing into a fable of royal injustice, but the legend’s core is not who he was — it is how thoroughly the state erased him.

\vspace{0.5em}
\textbf{References:}\\
Sonnino, P. (2016). \textit{The Search for the Man in the Iron Mask}. Rowman \& Littlefield.\\
Voltaire (1751). \textit{Le Siècle de Louis XIV}.\\
Archives Nationales, Série K. (1669–1703).\\
du Junca, E. (1703). \textit{Journal de la Bastille}.
\end{technical}
