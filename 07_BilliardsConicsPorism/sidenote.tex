\begin{SideNotePage}{
  \textbf{Top (Poncelet Trajectories):} \par
  Each subplot in the top section displays a pair of conics: the outer circle $x^2 + y^2 = 1$ and an inner ellipse of the form $\left(\frac{x}{a}\right)^2 + \left(\frac{y}{b}\right)^2 = 1$, where $(a, b)$ are derived from sampled Cayley invariants $(x, y)$. A five-step Poncelet trajectory is drawn in blue, starting from a fixed angle and iteratively constructing tangents from the circle to the ellipse. If the polygon closes after five steps, the pair lies on the Poncelet curve.
  \vspace{1.5em}

  \textbf{Bottom (Poncelet Curve):} \par
  The curve shown is the Poncelet curve associated with pentagonal (5-periodic) Poncelet polygons inscribed in one conic and circumscribed around another. Each point on this curve corresponds to a set of conic pairs for which a closed 5-gon exists that satisfies Poncelet’s closure condition. The coordinates $(x, y)$ represent algebraic invariants of the conic pair, specifically $x = e_2 / e_1^2$ and $y = e_3 / e_1^3$, where $e_k$ are the elementary symmetric functions of the characteristic multipliers derived from the conic configuration. A point on the curve means that a 5-periodic polygon can be inscribed and circumscribed for that particular combination of invariants. For more, see the excellent blog of Oliver Nash at \texttt{http://olivernash.org\\/2018/07/08/poring-over-poncelet/index.html}.
}{07_BilliardsConicsPorism/07_ A Complex (Projective) Billiard Game.pdf}
\end{SideNotePage}