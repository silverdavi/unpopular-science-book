Poncelet's Porism describes an unexpected property of billiard trajectories between two nested ellipses: if one path returns to its starting point after a finite number of bounces, then all starting points generate periodic trajectories with the same number of bounces. This geometric result connects to elliptic curves in number theory and measure-preserving dynamical systems. The theorem exemplifies how problems in distinct mathematical fields — from billiards to Gelfand's question about decimal digits — reduce to the same equations when expressed through appropriate frameworks.
