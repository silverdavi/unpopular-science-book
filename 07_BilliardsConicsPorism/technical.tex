\begin{technical}
\section*{Poncelet’s Theorem for Two Ellipses}

Let \( C \) and \( D \) be two smooth, nested ellipses in the plane, with \( D \subset \operatorname{int}(C) \). The \emph{Poncelet map} \( T: C \to C \) is defined as follows: for a point \( p \in C \), draw the line through \( p \) tangent to the inner ellipse \( D \); let \( T(p) \) be the second point of intersection of this line with \( C \), following a fixed orientation. This defines an orientation-preserving homeomorphism of \( C \).

We show that \( T \) preserves a natural measure and is topologically conjugate to a circle rotation. This yields a complete classification of the dynamics of \( T \), and with it, a proof of Poncelet's closure theorem.

\subsection*{Affine Reduction and the Invariant Measure}

To analyze \( T \), we apply an affine transformation to simplify the geometry. Suppose \( C \) is given by
\[
{x^2}/{a^2} + {y^2}/{b^2} = 1, \quad \text{with } a > b > 0.
\]
Let \( U(x, y) = (x, \frac{a}{b} y) \), a linear map sending \( C \) to a circle \( C' \), and \( D \) to an ellipse \( D' \). Let \( \tilde{p} = U(p) \), and let \( \tilde{s} \) denote the arc-length on \( C' \), which we use to reparametrize \( C \). Let \( m \) be the point of tangency between the line from \( p \) to \( T(p) \) and the inner ellipse \( D \). Define $\rho(p) := |p - m|$ and the measure $d\mu(p) := {d\tilde{s}(p)}/{\rho(p)}$.

This measure captures the rate of change of arc-length with respect to tangency.

\paragraph{Invariance of the Measure.} Consider nearby points \( p \) and \( p' \in C \), with images \( T(p) = p_1 \), \( T(p') = p_1' \). The chords \( pp_1 \) and \( p'p_1' \) intersect at a point \( n \), and the similarity of triangles \( \triangle pp'n \) and \( \triangle p_1'p_1n \) gives:
\[
\frac{|p_1' - p_1|}{|p' - p|} = \frac{\rho(p_1)}{\rho(p)}.
\]
Taking the limit as \( p' \to p \):
\[
\frac{d\tilde{s}_1}{d\tilde{s}} = \frac{\rho(T(p))}{\rho(p)}.
\]
Rewriting:
\[
\frac{d\tilde{s}}{\rho(p)} = \frac{d\tilde{s}_1}{\rho(T(p))},
\]
showing that \( \mu \) is preserved by \( T \).

\subsection*{Topological Conjugacy to a Circle Rotation}
A classical theorem states: if an orientation-preserving homeomorphism of a circle admits a finite, non-atomic invariant measure that gives positive weight to every arc, then it is topologically conjugate to a rotation.

Our measure \( \mu \) satisfies:\\
\textbf{Finite:} \( C \) is compact and \( \rho(p) > 0 \) is continuous. \textbf{Non-atomic:} points have zero measure. \textbf{Positive on arcs:} both \( \rho \) and arc-length are positive.

Thus, there exists a homeomorphism \( \varphi: C \to S^1 \) and \( \alpha \in [0,1) \) such that:$ \varphi \circ T \circ \varphi^{-1} = R_\alpha, \quad \text{where } R_\alpha(\theta) = \theta + \alpha \mod 1$

The number \( \alpha \), called the \emph{rotation number} of \( T \), quantifies the average angular displacement per iteration. It is defined as $
\alpha := \lim_{n \to \infty} \frac{1}{n} \tilde{s}(T^n(p)) \mod 1$, where \( \tilde{s} \) is arc-length on the circle under the conjugacy \( \varphi: C \to S^1 \). This limit exists and is independent of the choice of \( p \). (In smooth settings, it coincides with the normalized measure of the arc from \( p \) to \( T(p) \), i.e., \( \alpha = \mu([p, T(p))) \mod 1 \).)

\subsection*{Implications for Poncelet’s Theorem}

The rotation number classifies the dynamics:
\begin{itemize}
  \item If \( \alpha \in \mathbb{Q} \), say \( \alpha = \frac{m}{n} \), then all orbits are periodic with period \( n \). Every point on \( C \) traces a closed \( n \)-gon tangent to \( D \).
  \item If \( \alpha \notin \mathbb{Q} \), then no orbit is periodic. The sequence \( p, T(p), T^2(p), \ldots \) becomes dense in \( C \), and no Poncelet polygon closes.
\end{itemize}

\subsection*{References}
Leopold Flatto, \textit{Poncelet’s Theorem}. Mathematical Surveys and Monographs, Vol. 56. American Mathematical Society, 2009 (beautiful book!)

For interactive visualization, see \par
  \href{https://bit.ly/poporism}{\texttt{bit.ly/poporism}}.
\end{technical}
