\begin{SideNotePage}{
  \textbf{Top (Dimensional Scaling of Inverse Laws):}  
  The same point-source strength spreads differently depending on spatial dimension. In 1D, the influence remains constant along a line. In 2D, the effect dilutes like $1/r$ as it spreads over a circle. In 3D, it falls off as $1/r^2$, spreading over the surface of a sphere. This explains why gravitational and electrostatic forces scale as inverse-square laws in 3D space. \par

  \textbf{Bottom (Gabriel’s Horn and the Painter’s Paradox):}  
  A surface of revolution formed by rotating $y = 1/x$ around the $x$-axis for $x > 1$. Though the horn extends infinitely, it encloses a finite volume ($\int_1^\infty \pi (1/x)^2 dx = \pi$) but has infinite surface area ($2\pi \int_1^\infty (1/x) \sqrt{1 + (1/x)^2} dx = \infty$). Paradoxically, one could “fill” it with a finite amount of paint, but never coat its inner surface. \par

}{24_FourDSpacetime/24_ Put on Your 4D Glasses.pdf}
\end{SideNotePage}
