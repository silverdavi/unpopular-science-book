
Three spatial dimensions plus one time dimension ($D=4$) satisfy multiple independent physical constraints that fail in other dimensionalities. Gravitational orbits destabilize, quantum fields become non-renormalizable, wave propagation develops trailing echoes, and atomic structures collapse when $D \ne 4$. These failures arise from mathematics, not observation or biological perception.

In classical potential theory, the spatial decay of fields from a point source follows a general scaling law determined by Gauss’s theorem: the flux through a sphere in $n$ spatial dimensions scales with its surface area, yielding a radial dependence of $1/r^{n-1}$ for the field and $1/r^{n-2}$ for the associated potential. Only in $n=3$ does this produce the inverse-square law that governs Newtonian gravity and electrostatics. This particular falloff enables stable bound orbits under central forces, since it balances centripetal acceleration with potential curvature. In $n>3$, forces diminish too rapidly for stability; in $n<3$, confinement becomes excessive and destabilizes motion through excessive curvature.

Wave propagation obeys Huygens' principle only in odd-dimensional spaces. In $3+1$ spacetime, a localized disturbance generates a sharp spherical wavefront without trailing components. Lower or higher dimensions produce persistent field residuals after the main wave passes, blurring temporal boundaries between cause and effect.

Quantum field theory imposes stringent dimensional restrictions on the consistency of interaction terms. The renormalizability of a field theory — the ability to absorb divergences into a finite set of physical parameters — depends on the dimensional scaling of its coupling constants. In $D=4$, key interactions such as those of $\phi^4$ theory, quantum electrodynamics, and non-abelian gauge theories feature dimensionless couplings. This renders loop corrections manageable via renormalization group techniques. In $D>4$, the same interactions become non-renormalizable, requiring an infinite tower of counterterms. In $D<4$, interactions are super-renormalizable and lose sensitivity to ultraviolet structure, compromising their predictive reach.

The manifold $\mathbb{R}^4$ exhibits an anomaly in differential topology: it admits uncountably many smooth structures that are pairwise non-diffeomorphic yet topologically equivalent. These so-called exotic $\mathbb{R}^4$s violate the standard equivalence between smooth and topological manifolds. No analogous phenomenon occurs in dimensions $n \ne 4$. This breakdown in smooth uniqueness is tightly linked to the failure of the smooth Poincaré conjecture in $D=4$, and its resolution has required gauge-theoretic tools such as Donaldson invariants and Seiberg–Witten theory. The fact that deep gauge-theoretic phenomena become topologically meaningful only in four dimensions underscores a critical dimensional threshold.

In the algebraic classification of normed division algebras over $\mathbb{R}$, there exist only four: $\mathbb{R}$ (dimension 1), $\mathbb{C}$ (dimension 2), $\mathbb{H}$ (dimension 4), and $\mathbb{O}$ (dimension 8). Of these, only the quaternions $\mathbb{H}$ preserve associativity while extending beyond the complex numbers. They form the algebraic underpinning of spinor representations and enable the group isomorphism $\mathrm{SU}(2) \cong \mathrm{Spin}(3)$, which double-covers the rotation group $\mathrm{SO}(3)$. This structure supports the representation theory of spin-$\tfrac{1}{2}$ particles and the construction of Dirac spinors. No higher-dimensional associative division algebra exists, and the non-associativity of octonions prevents their integration into comparable representation frameworks.

In general relativity, the uniqueness of black hole solutions — encapsulated by the no-hair theorems — holds only in four-dimensional spacetime. Theorems by Israel, Carter, and Robinson prove that stationary black holes in $D=4$ are characterized entirely by mass, charge, and angular momentum. In higher dimensions, this rigidity fails. New solutions emerge with toroidal or ring-like horizons, including black rings and black strings. The breakdown of uniqueness introduces moduli and phase transitions into black hole classification, rendering the solution space unstable. Four dimensions represent the only case where Einstein’s vacuum equations enforce full solution rigidity for asymptotically flat, non-pathological metrics.

The quantum mechanical stability of atomic matter depends sensitively on the dimension $n$ of the configuration space. For a hydrogen-like atom, the Schrödinger equation with a $1/r^{n-2}$ potential yields bound states with discrete energy levels only if the kinetic energy dominates sufficiently near the origin to prevent collapse, while still allowing a negative-energy minimum. This balance occurs only for $n=3$, where the Coulomb potential produces a discrete spectrum bounded below. In $n>3$, the potential decays too rapidly to maintain binding; in $n<3$, singular behavior emerges due to excessive localization. The existence of atomic structure and periodic elements is thus contingent on this dimensional constraint.

Chemical bonding requires three spatial dimensions. Tetrahedral carbon in methane and chiral centers in amino acids depend on the angular momentum structure of $\mathrm{SO}(3)$ symmetry. In $n=2$, bonding becomes planar; in $n>3$, excessive rotational freedom destroys the geometric specificity needed for biochemical recognition.

I recommend watching \href{http://youtu.be/u5DLpAqX4YA&t=1170s}{Mikhail Gromov's lecture on the topic} (minute 19:30 in the video titled "What is a Manifold? - Mikhail Gromov") where Gromov traces the exceptional nature of four dimensions to the arithmetic identity $4 = 2 + 2$. A four-element set partitions into two pairs in exactly three ways: $\{\{1,2\}, \{3,4\}\}$, $\{\{1,3\}, \{2,4\}\}$, and $\{\{1,4\}, \{2,3\}\}$. The number of such partitions for a set of size $2n$ is $(2n-1)!! = (2n-1)(2n-3)\cdots 3 \cdot 1$. For $n=2$: three partitions. For $n=3$: fifteen partitions. For $n=4$: one hundred and five partitions. Only when $n=2$ does the partition count (3) remain smaller than the set size (4), enabling the symmetric group $S_4$ to map onto the smaller group $S_3$.

For odd-sized sets, no symmetric pair partitions exist. A three-element set partitions into $\{2,1\}$ in three ways, but these asymmetric partitions (one pair, one singleton) fail to generate a useful homomorphism. A five-element set partitions into $\{2,2,1\}$ in fifteen ways — again too many, and again asymmetric. Only the four-element set achieves both symmetry (all parts equal) and economy (minimal number of pairwise partitions to track set structure via symmetry action).

This homomorphism $\varphi: S_4 \to S_3$ tracks how permutations of four elements permute the three partitions. Its kernel contains precisely those permutations preserving all partitions: the Klein four-group $V_4 = \{e, (12)(34), (13)(24), (14)(23)\}$. This renders $A_4$ non-simple — the sole exception among alternating groups.

This manifests in Lie theory through the decomposition $\mathrm{SO}(4) \cong (\mathrm{SU}(2) \times \mathrm{SU}(2))/\mathbb{Z}_2$, splitting the Lie algebra $\mathfrak{so}(4) \cong \mathfrak{so}(3) \oplus \mathfrak{so}(3)$. This corresponds to decomposing 2-forms into self-dual and anti-self-dual components: $\Lambda^2(\mathbb{R}^4) = \Lambda^2_+ \oplus\Lambda^2_-$, where the Hodge star operator satisfies $*^2 = 1$, yielding eigenspaces with eigenvalues $\pm 1$.

In gauge theory, this splitting transforms second-order Yang-Mills equations into first-order conditions. A connection with curvature $F$ satisfying $F = *F$ (self-dual) or $F = -*F$ (anti-self-dual) automatically solves $D*F = 0$ since the Bianchi identity guarantees $DF = 0$. These instanton solutions exist only in four dimensions.

Donaldson's theorem, a hallmark of four-dimensionality, uses this. The moduli space of anti-self-dual connections on a 4-manifold yields polynomial invariants distinguishing smooth structures. Two homeomorphic 4-manifolds may have different Donaldson invariants, proving they are not diffeomorphic — a phenomenon occurring in no other dimension. The identity $4 = 2 + 2$ enables the entire apparatus.

The three pairings of four elements also correspond to three complex structures on $\mathbb{R}^4$. Viewing $\mathbb{R}^4$ as the quaternions $\mathbb{H}$, multiplication by the units $i$, $j$, $k$ gives three anti-commuting complex structures satisfying $i^2 = j^2 = k^2 = ijk = -1$. The action $(p,q) \cdot h = phq^{-1}$ by unit quaternions from left and right yields the $\mathrm{SU}(2) \times \mathrm{SU}(2)$ structure.

So the identity $4 = 2 + 2$ creates the alternating group exception, the Lie algebra splitting, the self-duality decomposition, and the instanton solutions that distinguish four-dimensional gauge theory. Stable orbits, renormalizable interactions, exotic smooth structures — these may indeed stem from this single combinatorial fact. The most sophisticated features of our universe follow from the simplest patterns in the integers.

For more details on the underlying mathematics, I recommend the book \href{https://bookstore.ams.org/FOURMAN}{The Wild World of 4-Manifolds} by Alexandru Scorpan.

\begin{commentary}[Why Four Might Be “Special”]
Why does our universe have four dimensions? One view appeals to anthropic reasoning: only in four do stable orbits, long-range forces, and anomaly-free field theories coexist — allowing for chemistry, stars, and observers. Another sees this as a coincidence of perspective — in a five-dimensional world, we might find five “necessary.” Alternatively, four may be inherently special: a small number where multiple constraints converge, like $2$ in spin systems or $8$ in normed algebras. Either way, four sits at a mathematical and physical crossroads.
\end{commentary}

