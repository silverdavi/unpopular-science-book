Why is the sky blue? Children ask this question expecting a simple answer. The truth involves quantum mechanics, molecular physics, and the peculiar sensitivity curve of human vision. More interesting: why do other planets' skies display different colors? Mars shows butterscotch, Venus gleams white, Neptune glows deep azure. Each atmosphere tells its own story through scattered light.

Lord Rayleigh solved the Earth problem in 1871. Light waves interact with air molecules much smaller than their wavelength. This interaction — Rayleigh scattering — depends on wavelength to the inverse fourth power. Blue light at 450 nanometers scatters 5.5 times more efficiently than red at 650 nanometers. Violet scatters even more, but solar output peaks in green and human eyes respond weakly to violet. The result: perceived blue dominance.

John Tyndall demonstrated this in his London laboratory by shining light through clarified air. He expected uniform illumination. Instead, the beam appeared blue from the side and reddish when viewed head-on. Particles in the air were separating colors through differential scattering. Tyndall thought dust caused the effect. Rayleigh proved air molecules sufficed.

At sunset, geometry changes everything. Sunlight traverses 40 times more atmosphere than at noon. Blue photons scatter away before reaching observers, leaving red and orange to paint the horizon. The same physics that makes noon skies blue creates crimson sunsets.

Mars presents a different problem. NASA's Viking landers in 1976 transmitted the first surface images showing a salmon-pink sky. Mission scientists initially "corrected" the colors to match Earth expectations — blue sky, red rocks. But the raw data insisted: Martian sky glows butterscotch to rust, depending on dust load.

The answer lies in particle size. Mars's atmosphere, 100 times thinner than Earth's, cannot support Rayleigh scattering as the dominant effect. Instead, dust particles 1-10 micrometers across create Mie scattering — a size regime where wavelength dependence weakens. These particles scatter all colors nearly equally while iron oxide content preferentially absorbs blue. The result: a peachy, dusty sky that grows yellower during the frequent planet-wide dust storms.

Carl Sagan faced skepticism when proposing the pinkish Martian sky before Viking landed. Critics argued any atmosphere would produce blue through Rayleigh effects. Sagan countered that dust overwhelms molecular scattering in Mars's thin air. Viking proved him right, though even Sagan underestimated how dramatically dust dominates Martian optics.

Venus hides different mysteries under perpetual clouds. Soviet Venera probes in the 1970s and 1980s revealed a bright but deeply filtered world. The thick atmosphere — 90 times Earth's pressure — supports multiple cloud layers of sulfuric acid droplets. These droplets, 1-3 micrometers across, scatter visible light efficiently while absorbing infrared.

At Venus's surface, only 2-3\% of sunlight penetrates. The light that arrives has been scattered dozens of times, creating uniform, shadowless illumination. The sky appears brilliant white fading to yellow-orange near the horizon — not from dust but from the sheer optical thickness traversed. Soviet images show a world lit like the inside of a ping-pong ball.

Moving outward reveals ice giant mysteries. Voyager 2's 1986 Uranus flyby and 1989 Neptune encounter showed worlds of ethereal blue, but different blues with different causes. Both atmospheres contain methane, which absorbs red light beyond 600 nanometers. But Neptune appears deeper blue despite similar methane content.

The solution emerged from spectroscopic analysis. Uranus maintains a high-altitude hydrocarbon haze produced by methane photolysis. This haze adds white scattered light, diluting the blue. Neptune's atmosphere proves more dynamic, with vertical mixing that clears the upper atmosphere. Less haze means purer methane absorption, deeper blue.

Titan offers the strangest sky in our solar system. Cassini's Huygens probe descended through Titan's atmosphere in 2005, revealing an orange world where methane rain falls through nitrogen air. The thick atmosphere — 1.5 times Earth's pressure — supports complex photochemistry. Ultraviolet light breaks methane into fragments that recombine into complex hydrocarbons called tholins.

These tholins create Titan's signature orange haze, absorbing blue and ultraviolet while scattering red. The haze proves so thick that surface illumination equals deep twilight on Earth. Huygens's images show a dim orange world where the sun appears as a vague bright region, 1/1000th its brightness on Earth.

Each world's sky chronicles its atmospheric history. Earth's transparent blue speaks of oxygen and life — our atmosphere contains little dust because rain washes it out, rain that exists because of our temperature and pressure regime. Mars's dusty sky reveals a world where low pressure prevents liquid water, allowing dust to accumulate for millions of years. Venus's acid clouds trace a runaway greenhouse. The ice giants' methane traces primordial composition preserved in cold storage.

Beyond planets, color tells deeper stories. Stars themselves span a rainbow sequence tied to surface temperature. The hottest O-type stars blaze blue-white at 30,000 Kelvin. Our G-type Sun glows yellow-white at 5,800 Kelvin. Cool M-dwarfs smolder red at 3,000 Kelvin. Annie Jump Cannon classified thousands of stellar spectra at Harvard Observatory, establishing the temperature sequence we still use. Her work revealed that star color directly encodes physical properties.

Nebulae paint space in emission colors. The Orion Nebula glows red from hydrogen-alpha emission at precisely 656.3 nanometers — the quantum transition from n=3 to n=2 in hydrogen atoms. The Ring Nebula adds blue-green from doubly-ionized oxygen. These aren't aesthetic choices but quantum mechanics made visible. Each color corresponds to specific electron transitions in specific atoms under specific conditions.

William Huggins pioneered nebular spectroscopy in the 1860s, discovering that some "nebulae" showed emission lines rather than stellar spectra. This proved them to be glowing gas clouds rather than unresolved star clusters. His work established that color in astronomy carries quantitative information about composition and physical state.

Even galaxy colors encode history. Edwin Hubble's galaxy classification connected color to stellar populations. Blue spiral arms mark regions of active star formation where massive, hot stars dominate. Red elliptical galaxies contain older stellar populations with the massive blue stars long dead. The cosmic color palette traces stellar evolution on billion-year timescales.

Modern astronomy pushes beyond visible light, but the principle remains: wavelength encodes physics. Radio telescopes map cold hydrogen. Infrared reveals hidden star formation. X-rays trace million-degree gas. Each "color" — visible or not — provides unique information about matter and energy in extreme conditions.

Color in astronomy is never arbitrary. From a child's question about Earth's blue sky to the deep mysteries of cosmic chemistry, spectrum analysis reveals the universe's workings. Every photon carries information about its origin and journey. Learning to decode these messages transforms simple color into profound understanding.

The sky is blue because air molecules dance with light in wavelength-dependent patterns. But that same dance, played out across the cosmos with different partners, creates the full spectrum of astronomical color. Each hue tells a story. Physics provides the dictionary to read them.
