\begin{technical}
{\Large\textbf{Quantitative Analysis of Astronomical Color}}\\[0.7em]

\noindent\textbf{Introduction}\\[0.5em]
The observed color and spectrum of an astronomical object result from the emission, absorption, and scattering of electromagnetic radiation, governed by the principles of radiative transfer, thermodynamics, and atomic physics. These processes determine the distribution of intensity across wavelengths. Analyzing the spectral energy distribution provides quantitative access to temperature, composition, density, and velocity.

\noindent\textbf{Radiative Transfer and Blackbody Radiation}\\[0.5em]
The propagation of specific intensity \( I_\nu \) along a path \( s \) is governed by the equation of radiative transfer:
\[
\frac{dI_\nu}{ds} = j_\nu - \alpha_\nu I_\nu - \sigma_\nu I_\nu + \iint \sigma_\nu(\Omega', \Omega) I_\nu(\Omega')\, d\Omega'
\]
where \( j_\nu \) is the emission coefficient, \( \alpha_\nu \) is the absorption coefficient, and \( \sigma_\nu \) is the scattering coefficient. The optical depth is \( d\tau_\nu = (\alpha_\nu + \sigma_\nu)\, ds \). In local thermodynamic equilibrium (LTE), the source function \( S_\nu = j_\nu/\alpha_\nu \) approaches the Planck function:
\[
B_\nu(T) = \frac{2h\nu^3}{c^2} \left[ \exp\left( \frac{h\nu}{kT} \right) - 1 \right]^{-1}
\]
The wavelength of peak emission is given by Wien's law:
\[
\lambda_{\text{max}} T = b \approx 2.898 \times 10^{-3} \, \text{m} \cdot \text{K}
\]
and the total emitted flux per unit area follows the Stefan--Boltzmann law:
\[
F = \sigma T^4, \quad \sigma = \frac{2\pi^5 k^4}{15c^2 h^3}
\]
Stellar temperatures can be inferred by fitting observed continua or via color indices (e.g., \( B-V \)), which measure differences in magnitude across filtered bands.

\noindent\textbf{Spectral Lines and Atmospheric Composition}\\[0.5em]
Spectral lines originate from electronic transitions with energy \( \Delta E = h\nu = E_u - E_l \). In LTE, population ratios follow the Boltzmann distribution:
\[
\frac{N_u}{N_l} = \frac{g_u}{g_l} \exp\left( -\frac{E_u - E_l}{kT} \right)
\]
Ionization states are governed by the Saha equation:
\[
\frac{N_{i+1} N_e}{N_i} = \frac{2 g_{i+1}}{g_i} \left( \frac{2\pi m_e kT}{h^2} \right)^{3/2} \exp\left( -\frac{\chi_i}{kT} \right)
\]
where \( \chi_i \) is the ionization energy and \( N_e \) is the electron density. Line shapes are broadened by natural width, thermal Doppler broadening:
\[
\Delta \lambda_D = \lambda_0 \left( \frac{2kT \ln 2}{mc^2} \right)^{1/2}
\]
and collisional (pressure) broadening, which scales with density. Observed line intensities allow reconstruction of chemical abundances and physical conditions.

\noindent\textbf{Motion, Redshift, and Extinction}\\[0.5em]
The Doppler effect shifts wavelengths by
\[
\frac{\Delta \lambda}{\lambda_0} = \frac{v_r}{c}, \quad (v_r \ll c)
\]
where \( v_r \) is the radial velocity. Redshift (\( \Delta \lambda > 0 \)) indicates recession; blueshift indicates approach. For distant galaxies, the cosmological redshift \( z \) is related to the scale factor \( R(t) \) via:
\[
1 + z = \frac{R(t_{\text{obs}})}{R(t_{\text{emit}})}
\]
and follows Hubble's law for \( z \ll 1 \): \( v = H_0 d \).

Extinction by dust is quantified by
\[
A_\lambda = 1.086\, \tau_\lambda = -2.5 \log_{10} \left( {F_\lambda}/{F_{\lambda,0}} \right)
\]
where \( \tau_\lambda \) is the optical depth. Reddening is the differential extinction between bands:
\[
E(B - V) = A_B - A_V
\]
and the total-to-selective extinction ratio \( R_V = A_V / E(B - V) \) typically has a value near 3.1 for the diffuse interstellar medium. For small particles \( a \ll \lambda \), Rayleigh scattering dominates, with efficiency \( \propto \lambda^{-4} \). For \( a \sim \lambda \), Mie scattering applies, with weaker wavelength dependence.

\vspace{0.5em}
\noindent\textbf{References:}\\
Rybicki, G. B., \& Lightman, A. P. (1979). \textit{Radiative Processes in Astrophysics}. Wiley.\\
Draine, B. T. (2003). Interstellar Dust Grains. \textit{Annual Review of Astronomy and Astrophysics}
\end{technical}
