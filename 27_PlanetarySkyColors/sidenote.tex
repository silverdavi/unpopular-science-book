\begin{SideNotePage}{
  
    \textbf{Top (Atmospheric Filtering of Stellar Radiation):}  
    Sunlight reaching Earth undergoes selective scattering by the atmosphere. Shorter wavelengths (blue, violet) are scattered in all directions, while longer wavelengths (red, orange) pass through more directly. This results in both the blue sky and the reddening of the sun near the horizon. \par
  
    \textbf{Bottom (Hertzsprung–Russell Diagram):}  
    Stars are plotted by surface temperature (x-axis, decreasing rightward) and luminosity (y-axis, log scale). Main sequence stars form a diagonal band; giants and supergiants lie above, white dwarfs below. The Sun sits in the middle of the main sequence. The temperature scale reflects stellar classification: blue-hot stars (left) are hotter and more luminous; red stars (right) are cooler. \par
  
}{27_PlanetarySkyColors/27_ A Spectrum of Skies.pdf}
\end{SideNotePage}
