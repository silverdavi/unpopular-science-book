Sky colors arre determined by light scattering and spectral signatures. Earth's blue sky emerges from Rayleigh scattering, where nitrogen and oxygen molecules preferentially scatter shorter wavelengths by factors of 10-100 times more efficiently than longer ones. Mars's butterscotch-orange haze results from suspended dust particles 1-10 micrometers across, scattering all wavelengths equally while absorbing blue light. Titan's perpetual methane fog produces deeper orange through photochemical hazes, while Venus's perpetual cloud deck creates brilliant white from sulfuric acid droplets. Beyond planets, astronomical spectroscopy reveals stellar temperatures from blue giants at 30,000K to red dwarfs at 3,000K, chemical compositions through absorption lines, and cosmic expansion via redshift scaling with distance. These colors allow measurements of atmospheric chemistry, stellar physics, and universal evolution.