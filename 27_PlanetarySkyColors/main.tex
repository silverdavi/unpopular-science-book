The blue color of Earth's sky results from the interaction of solar radiation with atmospheric gases. Molecules in the air interact with incoming sunlight, scattering it in multiple directions and producing a diffuse background of illumination across the sky.

Sunlight entering the atmosphere is redirected in all directions by air molecules, making the sky appear uniformly bright even when the Sun itself is not in the observer's direct line of sight. This scattering process depends strongly on the wavelength of the incoming light.

When the particles responsible for scattering are much smaller than the wavelength of light, the scattering cross-section varies inversely with the fourth power of wavelength. This is Rayleigh scattering, and it favors shorter wavelengths --- such as blue and violet --- over longer wavelengths like red.

Although violet light scatters more than blue, human eyes are less sensitive to violet, and the solar spectrum itself emits slightly less in that band. The combined result is that the sky appears dominantly blue to human observers.

At sunrise and sunset, sunlight reaches the observer after passing through a much longer segment of the atmosphere. During this path, blue and green wavelengths are more likely to scatter out of the direct line of sight, leaving a predominance of red and orange light reaching the eye.

These effects result entirely from well-characterized electromagnetic interactions between photons and molecules. No anthropocentric interpretation is needed --- perceived color corresponds directly to differences in scattering efficiency and path length through the atmosphere.

The same mechanisms --- scattering, absorption, and emission --- operate throughout astronomical systems. Whenever light interacts with gas, dust, or solid matter, its spectrum is altered in a way that encodes the physical properties of the medium.

These spectral changes include both redistribution of intensity across wavelengths and selective suppression or enhancement of specific bands. The net result is an observable spectrum whose structure reflects temperature, chemical composition, and geometric configuration.

The apparent color of any luminous or reflective object is determined by this emergent spectrum. Color, in astronomy, is a quantitative summary of the integrated distribution of intensity across wavelength bands.

Stars are not observed by reflected light but by their own emission, which arises from thermal radiation. A star approximates a blackbody, and its emission spectrum depends primarily on its surface temperature.

As temperature increases, the peak of the emitted spectrum shifts to shorter wavelengths, a relationship governed by Wien's displacement law. Hotter stars peak in the blue or ultraviolet; cooler stars peak in the red or infrared.

The Sun, with a surface temperature of approximately 5800 K, emits across the entire visible range and slightly beyond, producing a spectrum that, when integrated, appears white-yellow under Earth's atmospheric conditions.

In stellar classification, the spectral type (O, B, A, F, G, K, M) is assigned according to both temperature and absorption line structure. The color is thus tied both to continuum shape and to the atomic transitions in the star's outer layers.

Planets, unlike stars, reflect solar radiation. Their apparent color depends on both the wavelength-dependent reflectivity of their surfaces and the selective absorption of their atmospheres.

Mars appears red because its regolith contains iron oxide particles, which reflect red and absorb shorter wavelengths. The result is a consistently reddish hue under varying solar angles.

Jupiter's atmosphere contains layered cloud structures composed of ammonia ice and deeper chromophores, which produce alternating white and reddish bands depending on altitude and composition.

Uranus and Neptune exhibit deep blue coloration due to methane in their atmospheres, which absorbs red light more efficiently than blue. Neptune's darker tone is attributed to a thinner haze layer, allowing deeper penetration and stronger blue return.

In interstellar space, nebulae derive their color from interaction with nearby stars and from internal composition. Emission nebulae glow because they are ionized by adjacent high-energy sources.

The dominant emission line in such cases is the H$\alpha$ line of hydrogen at 656.3 nm, which produces red glow. Additional contributions from doubly ionized oxygen [OIII] near 500 nm yield green-blue hues in many planetary nebulae.

Reflection nebulae contain interstellar dust that scatters nearby starlight. Like Earth's atmosphere, this scattering favors shorter wavelengths and produces a blue-tinted glow even when the dust itself is not self-luminous.

Dark nebulae are dense molecular clouds that neither emit nor scatter significantly. They block light from background stars and appear as sharply defined voids in otherwise luminous regions of the sky.

Galaxies, composed of billions of stars and clouds of gas and dust, exhibit integrated colors that reflect the average stellar population. Young, massive stars contribute disproportionately to blue light, while older stars dominate in red.

Spiral galaxies undergoing active star formation tend to appear blue-white due to the presence of hot O and B type stars. Elliptical galaxies, which contain older and cooler stars, appear yellow or red in color.

Dust within galaxies modifies this picture by introducing extinction --- wavelength-dependent absorption and scattering that preferentially attenuates blue light and reddens the observed spectrum.

Relative motion also affects color. The Doppler effect shifts all spectral features of a moving source: objects approaching the observer exhibit shorter wavelengths (blueshift), while receding sources appear redshifted.

For individual stars or galaxies, this shift reveals radial velocity. In cosmology, redshift also encodes the expansion of space itself, stretching light as it travels and shifting entire spectra toward longer wavelengths.

Astronomical images capture these spectral properties in a variety of encoding schemes. In true-color imaging, red, green, and blue filters are assigned to wavelengths matching human cone sensitivity.

In false-color imaging, colors are mapped to spectral bands outside the visible range, or to narrow-band emissions of diagnostic importance. This includes X-ray, radio, or emission-line imaging of ionized gas.

These color assignments are analytical, not decorative. They reveal spatial distributions of temperature, chemical species, velocity gradients, or magnetic structures that would otherwise be invisible.

Color in astronomy is a physically determined quantity. It is not an aesthetic overlay but a compressed expression of spectrally resolved measurements. Every visible hue corresponds to an underlying interaction with matter.

These conditions act on light through well-specified mechanisms --- absorption, scattering, reemission, and redshift --- to produce each observable spectrum.


\begin{commentary}[Why This Story]
This was the first chapter I wrote. I wanted a more sophisticated answer to a question my children often ask: why is the sky blue? I knew the basic explanation --- Rayleigh scattering and wavelength dependence --- but I wanted a version that would remain meaningful even after they learned calculus, optics, or astrophysics. The question becomes more, not less, interesting with each layer of generalization: from sunlight and air to blackbody curves, stellar classifications, and interstellar dust. Writing this chapter helped establish the book's tone --- clear, structured, physically grounded --- and set the standard for treating simple questions with full scientific seriousness.
\end{commentary}