\begin{SideNotePage}{
  \textbf{Top (Molecular Reaction):}  
  At the Ångström scale, the luminescent reaction begins: luciferin is enzymatically converted into luciferyl-adenylate, then oxidized into oxyluciferin. The process releases a visible photon ($h\nu$), producing light. \par

  \textbf{Second (Enzyme Active Site):}  
  At nanometer scale, the luciferase enzyme forms a pocket where the reaction occurs. This active site holds luciferin in the correct orientation, enabling efficient light-producing catalysis. \par

  \textbf{Third (Subcellular Photocyte):}  
  At the 1–10 μm scale, we zoom into a photocyte— a specialized light-producing cell. Organelles like mitochondria and endoplasmic reticulum support energy-demanding luminescence. \par

  \textbf{Fourth (Cell/Tissue Cross-Section):}  
  At the 10–100 μm scale, we see a full photocyte embedded in a structured tissue. The densely packed spheres represent light-emitting units; structural organization optimizes light output and diffusion. \par

  \textbf{Fifth (Lantern Organ):}  
  At the 1–5 mm scale, the firefly's lantern organ is visible in cross-section. Photocytes, tracheal tubes (for oxygen), and reflector layers are arranged to maximize brightness and directionality. \par

  \textbf{Sixth (Whole Firefly):}  
  At the centimeter scale, we see the full insect with its ventral lantern exposed. \par

  \textbf{Bottom (Population-Level Output):}  
  At the meter scale, we zoom out to an ecological view: a jar of ~40 fireflies generates light that seems equivalent to a candle.\par

}{25_FireflyBioluminescence/25_ Let There Be Bioluminsecence.pdf}
\end{SideNotePage}
