\fullpageexercises[Exercise: Constructing RSA Encryption from Modular Arithmetic]{

\noindent \textbf{Objective:} Construct a working RSA encryption and decryption system using small integers, to see how modular arithmetic underpins real-world cryptography.

\noindent \textbf{Instructions:} Work through each step. Use the provided primes and verify your calculations carefully.

\begin{enumerate}
    \item \textbf{Choose Two Distinct Prime Numbers.}  
    Let \( p = 7 \), \( q = 11 \). Compute:
    \[
    n = pq \qquad \phi(n) = (p - 1)(q - 1)
    \]

    \item \textbf{Choose a Public Exponent \( e \).}  
    Choose a number \( e \) such that \( 1 < e < \phi(n) \) and \( \gcd(e, \phi(n)) = 1 \). Suggestion: try \( e = 13 \). Verify that it is valid.

    \item \textbf{Find the Private Exponent \( d \).}  
    Find \( d \) such that:
    \[
    ed \equiv 1 \mod \phi(n)
    \]
    Use the extended Euclidean algorithm.

    \item \textbf{Publish the Public Key.}  
    The public key is \( (n, e) \). The private key is \( d \). Write down both.

    \item \textbf{Encrypt a Message.}  
    Let the message be \( m = 9 \). Compute:
    \[
    c = m^e \mod n
    \]

    \item \textbf{Decrypt the Ciphertext.}  
    Using the private key \( d \), compute:
    \[
    m = c^d \mod n
    \]
    Verify that it matches the original message.

    \item \textbf{Why Does This Work?}  
    Briefly explain why decryption recovers the original message. Use Euler’s theorem or Fermat’s little theorem to justify:
    \[
    m^{ed} \equiv m \mod n
    \]

    \item \textbf{Reflect.}  
    What part of this scheme would be hard to reverse without the private key? What role does factorization play in securing the system?
\end{enumerate}

\noindent\hrulefill

\begin{center}
\rotatebox[origin=c]{180}{%
\begin{minipage}{0.95\textwidth}
\small
\noindent
\textbf{Answers:} \\
1. \( n = 77, \phi(n) = 60 \) \\
2. \( e = 13 \), and \( \gcd(13, 60) = 1 \) — valid \\
3. \( d = 37 \), since \( 13 \cdot 37 \equiv 1 \mod 60 \) \\
4. Public key: \( (77, 13) \), Private key: \( d = 37 \) \\
5. \( c = 9^{13} \mod 77 = 58 \) \\
6. \( m = 58^{37} \mod 77 = 9 \) \\
7. Because \( ed \equiv 1 \mod \phi(n) \), Euler’s theorem gives \( m^{ed} \equiv m \mod n \) \\
8. Recovering \( d \) requires factoring \( n = 77 \) to find \( \phi(n) \), which is computationally hard for large \( n \)
\end{minipage}
}
\end{center}
}
