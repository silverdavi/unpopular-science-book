\begin{historical}
The history of secure communication traces back to the intersection of two 20th-century revolutions: cryptography and wireless technology. For much of the century, radio systems prioritized reach and reliability over confidentiality. During World War II, breakthroughs like the German Enigma and Allied SIGSALY systems introduced large-scale encryption in radio, but these were bespoke wartime inventions, not standardized infrastructure.

In the postwar decades, civilian telecommunication networks expanded rapidly, but radio links remained analog and vulnerable. First-generation mobile systems (1G), such as AMPS in the United States and NMT in Scandinavia, used frequency modulation without encryption. Voice data was transmitted as analog waveforms, easily intercepted by anyone with a scanner. This vulnerability, while tolerated during the early novelty of mobile phones, became untenable as usage grew.

By the 1980s, Europe pursued a unified digital cellular standard under the banner of the Groupe Spécial Mobile (GSM). The aim was not only cross-border interoperability, but also improved spectrum efficiency and — critically — built-in security. The digital transition allowed for integration of error correction, time-multiplexing, and cryptography into the protocol stack. Unlike analog predecessors, GSM was designed from the outset to offer some degree of confidentiality on the radio link.
\end{historical}
