\begin{technical}
{\Large\textbf{Timeline of the Early Universe: Dynamical Regimes and Governing Equations}}\\[0.7em]

\noindent\textbf{Introduction}\\[0.5em]
The evolution of the universe from \( t \gtrsim t_P \) onward can be described as a sequence of physically distinct regimes, each governed by different dominant components and effective equations. This section presents a timeline of cosmic evolution in proper time \( t \), detailing the relevant physics, scales, and equations that govern each epoch.

\noindent\textbf{1. Planck Epoch \boldmath(\( t < 10^{-43} \, \text{s} \))}\\[0.5em]
At or before the Planck time,
\[
t_P = \sqrt{\frac{\hbar G}{c^5}} \approx 5.39 \times 10^{-44} \, \text{s},
\]
quantum gravitational effects dominate. The metric itself is expected to fluctuate, and no classical description holds. A complete treatment would require a theory of quantum gravity, e.g., loop quantum gravity or string theory.

\noindent\textbf{2. Inflationary Epoch \boldmath(\(10^{-36} \, \text{s} \lesssim t \lesssim 10^{-32} \, \text{s} \))}\\[0.5em]
Inflation is modeled by a scalar field \(\phi\) with a flat potential \(V(\phi)\), leading to exponential expansion:
\[
a(t) \propto e^{H t}, \quad H = \sqrt{\frac{8\pi G}{3} V(\phi)}.
\]
During this period, quantum fluctuations of the inflaton generate nearly scale-invariant perturbations:
\[
\mathcal{P}_\zeta(k) \sim \left( \frac{H^2}{\dot{\phi}} \right)^2 \sim \frac{V}{\epsilon M_{\text{Pl}}^4}.
\]
These seed all later structure in the universe.

\noindent\textbf{3. Reheating and Preheating \boldmath(\( t \sim 10^{-32} \, \text{s} \))}\\[0.5em]
When inflation ends, the inflaton oscillates around its minimum and decays. Reheating converts vacuum energy to particles and entropy:
\[
\ddot{\phi} + 3H\dot{\phi} + V'(\phi) = -\Gamma_\phi \dot{\phi},
\]
where \(\Gamma_\phi\) is the decay rate. The reheat temperature is estimated as:
\[
T_{\text{reh}} \sim \left( \frac{90}{\pi^2 g_*} \right)^{1/4} \sqrt{\Gamma_\phi M_{\text{Pl}}}.
\]

\noindent\textbf{4. Radiation-Dominated Era \boldmath(\( 10^{-32} \, \text{s} \lesssim t \lesssim 47{,}000 \, \text{years} \))}\\[0.5em]
After reheating, the universe is dominated by relativistic species. The Friedmann equation simplifies to:
\[
H^2 = \frac{8\pi G}{3} \rho_r, \quad \rho_r \propto a^{-4}, \quad a(t) \propto t^{1/2}.
\]
This era includes several key events:
\begin{itemize}
\item \textbf{Baryogenesis} (\( t \sim 10^{-10} \, \text{s} \)): CP-violating processes establish matter-antimatter asymmetry.
\item \textbf{Electroweak symmetry breaking} (\( t \sim 10^{-12} \, \text{s} \), \( T \sim 100\,\text{GeV} \)).
\item \textbf{QCD confinement} (\( t \sim 10^{-5} \, \text{s} \)): Quarks bind into hadrons.
\item \textbf{Big Bang Nucleosynthesis (BBN)} (\( t \sim 1 - 200 \, \text{s} \)): Nuclear reactions form light elements. Equilibrium abundance equations (e.g., Boltzmann equations) govern species evolution:
\[
\frac{dY_i}{dt} = - \sum_j \langle \sigma v \rangle_{ij} \left( Y_i Y_j - Y_i^{\text{eq}} Y_j^{\text{eq}} \right),
\]
where \( Y_i = n_i/s \) is the abundance per entropy.
\end{itemize}

\noindent\textbf{5. Matter-Dominated Era \boldmath(\( 47{,}000 \, \text{years} \lesssim t \lesssim 9\,\text{billion years} \))}\\[0.5em]
When \(\rho_m > \rho_r\), the scale factor evolves as:
\[
a(t) \propto t^{2/3}, \quad \rho_m \propto a^{-3}.
\]
Structure formation begins as gravitational collapse overcomes expansion. Linear perturbation theory describes early growth:
\[
\ddot{\delta} + 2 H \dot{\delta} - 4\pi G \rho_m \delta = 0,
\]
with \(\delta = \delta\rho/\rho\).

\noindent\textbf{6. Dark Energy Domination \boldmath(\( t \gtrsim 9\,\text{billion years} \))}\\[0.5em]
Today’s expansion is accelerating due to a cosmological constant \(\Lambda\):
\[
H^2 = \frac{8\pi G}{3} (\rho_m + \rho_\Lambda), \quad \rho_\Lambda = \frac{\Lambda c^2}{8\pi G} = \text{const.}
\]
For \(\Lambda\)-dominated expansion:
\[
a(t) \propto e^{H_\Lambda t}, \quad H_\Lambda = \sqrt{\frac{\Lambda c^2}{3}}.
\]
The universe asymptotically approaches de Sitter space.

\vspace{0.5em}
\noindent\textbf{References:}\\
Dodelson, S. (2003). \textit{Modern Cosmology}. Academic Press.\\
Kolb, E. W., \& Turner, M. S. (1990). \textit{The Early Universe}. Addison-Wesley.
\end{technical}
