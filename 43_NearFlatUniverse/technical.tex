\begin{technical}
{\Large\textbf{Timeline of the Early Universe}}\\[0.2em]

\noindent\textbf{1. Planck Epoch \boldmath(\( t < 10^{-43} \, \text{s} \))}\\[0.3em]
At or before the Planck time,
\[
t_P = \sqrt{\hbar G/c^5} \approx 5.39 \times 10^{-44} \, \text{s},
\]
quantum gravitational effects dominate. The metric itself is expected to fluctuate, and no classical description holds. A complete treatment would require a theory of quantum gravity, e.g., loop quantum gravity or string theory.

\noindent\textbf{2. Inflationary Epoch \boldmath(\(10^{-36} \, \text{s} \lesssim t \lesssim 10^{-32} \, \text{s} \))}\\[0.3em]
Inflation is modeled by a scalar field \(\phi\) with a flat potential \(V(\phi)\), leading to exponential expansion:
\begin{align}
a(t) \propto e^{H t}, \quad H = \sqrt{8\pi G V(\phi)/3}.
\end{align}
During this period, quantum fluctuations of the inflaton generate nearly scale-invariant perturbations:
\[
\mathcal{P}_\zeta(k) \sim (H^2/\dot{\phi})^2 \sim V/(\epsilon M_{\text{Pl}}^4).
\]
These seed all later structure in the universe.

\noindent\textbf{3. Reheating and Preheating \boldmath(\( t \sim 10^{-32} \, \text{s} \))}\\[0.3em]
When inflation ends, the inflaton oscillates around its minimum and decays. Reheating converts vacuum energy to particles and entropy:
\begin{align}
\ddot{\phi} + 3H\dot{\phi} + V'(\phi) = -\Gamma_\phi \dot{\phi},
\end{align}
where \(\Gamma_\phi\) is the decay rate. The reheat temperature is estimated as:
\[
T_{\text{reh}} \sim (90/\pi^2 g_*)^{1/4} \sqrt{\Gamma_\phi M_{\text{Pl}}}.
\]

\noindent\textbf{4. Radiation-Dominated Era \boldmath(\( 10^{-32} \, \text{s} \lesssim t \lesssim 47{,}000 \, \text{years} \))}\\[0.3em]
After reheating, the universe is dominated by relativistic species. The Friedmann equation simplifies to:
\begin{align}
H^2 = 8\pi G \rho_r/3, \quad \rho_r \propto a^{-4}, \quad a(t) \propto t^{1/2}.
\end{align}
This era includes several key events:
\begin{itemize}
\item \textbf{Baryogenesis} (\( t \sim 10^{-10} \, \text{s} \)): CP-violating processes establish matter-antimatter asymmetry.
\item \textbf{Electroweak symmetry breaking} (\( t \sim 10^{-12} \, \text{s} \), \( T \sim 100\,\text{GeV} \)).
\item \textbf{QCD confinement} (\( t \sim 10^{-5} \, \text{s} \)): Quarks bind into hadrons.
\item \textbf{Big Bang Nucleosynthesis (BBN)} (\( t \sim 1 - 200 \, \text{s} \)): Nuclear reactions form light elements.
\end{itemize}

\noindent\textbf{5. Matter-Dominated Era \boldmath(\( 47{,}000 \, \text{years} \lesssim t \lesssim 9\,\text{Gyr} \))}\\[0.3em]
When \(\rho_m > \rho_r\), the scale factor evolves as:
\begin{align}
a(t) \propto t^{2/3}, \quad \rho_m \propto a^{-3}.
\end{align}
Structure formation begins as gravitational collapse overcomes expansion. Linear perturbation theory describes early growth:
\[
\ddot{\delta} + 2 H \dot{\delta} - 4\pi G \rho_m \delta = 0,
\]
with \(\delta = \delta\rho/\rho\).

\noindent\textbf{6. Dark Energy Domination \boldmath(\( t \gtrsim 9\,\text{Gyr} \))}\\[0.3em]
Today's expansion is accelerating due to a cosmological constant \(\Lambda\):
\begin{align}
H^2 &= 8\pi G (\rho_m + \rho_\Lambda)/3, \\
\rho_\Lambda &= \Lambda c^2/(8\pi G) = \text{const.}
\end{align}
For \(\Lambda\)-dominated expansion:
\begin{align}
a(t) \propto e^{H_\Lambda t}, \quad H_\Lambda = \sqrt{\Lambda c^2/3}.
\end{align}
The universe asymptotically approaches de Sitter space.

\vspace{0.3em}
\noindent\textbf{References:}\\
Dodelson, S. (2003). \textit{Modern Cosmology}.\\
Kolb, E. W., \& Turner, M. S. (1990). \textit{The Early Universe}.\\
\end{technical}
