A geometric space is defined by the relations among its points: distances, angles, and the behavior of geodesics — paths that locally minimize distance. In Euclidean geometry, these structures are governed by axioms such as the parallel postulate, which ensures that parallel lines never intersect and that triangle angles sum to 180 degrees. When these properties fail, the space is said to be curved.

Curvature quantifies how a space deviates from the rules of Euclidean geometry. Positive curvature causes initially parallel lines to converge, as on the surface of a sphere. Negative curvature causes them to diverge, as in a hyperbolic plane. Zero curvature preserves their parallelism indefinitely. These cases define the three canonical geometries in two dimensions: spherical, hyperbolic, and flat.

Curvature is a local property: it describes how space behaves in an infinitesimal neighborhood. Compactness is a global property: it describes whether space is bounded and complete. The surface of a sphere is compact and positively curved. A flat plane is non-compact and uncurved. A cylinder is flat but compact in one direction. These examples show that curvature and compactness are independent notions.

Dimensionality extends these ideas. A two-dimensional surface has curvature defined through how it bends in three-dimensional space, but intrinsic curvature does not depend on external embedding. A three-dimensional space can have its own curvature, defined purely through internal measurements of distance and angle. General relativity models the universe using such three-dimensional spatial geometries evolving in time.

The mathematical classification of homogeneous, isotropic three-dimensional spaces yields three possibilities: positive curvature (a 3-sphere), negative curvature (a 3-hyperboloid), and zero curvature (Euclidean $\mathbb{R}^3$). Each corresponds to a constant value of spatial curvature and admits a rigorous metric structure. These are the geometric possibilities for the shape of the universe on large scales.

A cosmological model describes the structure and evolution of space and time on the largest scales. In general relativity, such a model is not a visual rendering of stars and galaxies, but a mathematical solution to Einstein’s field equations. It specifies the metric tensor — a geometric object encoding distances, angles, and causal structure across spacetime. Given assumptions about symmetry and matter content, the metric determines how space stretches, curves, and evolves in time.

The standard model of cosmology is called the Lambda–Cold Dark Matter model ($\Lambda$CDM). It assumes that, at sufficiently large scales, the universe is homogeneous and isotropic. These two conditions — uniformity in position and direction — restrict the possible spatial geometries to three: constant positive curvature (spherical), constant negative curvature (hyperbolic), or zero curvature (flat). These are the same geometries classified earlier in purely mathematical terms, now applied to cosmic structure.

The spatial curvature in $\Lambda$CDM is not a free assumption. It is determined by the total energy density of the universe relative to a critical threshold. If the density exceeds this critical value, spatial curvature is positive. If it falls short, the curvature is negative. If it equals the critical density, space is flat. This relation is not a convention but a calculable consequence of Einstein’s equations applied to a homogeneous, isotropic universe.

In this framework, the Big Bang is not a point in space, but a boundary in time: a moment when the scale factor — the function describing the distance between any two comoving points — reaches zero. It represents the earliest definable state of the metric itself, beyond which classical general relativity ceases to apply. The Big Bang is not an explosion of matter into space. It is the dynamical expansion of space itself, governed by the evolving metric. All regions of the universe were arbitrarily close together in the past and have since expanded away from each other in a coordinated, metric-driven evolution.

This expansion is not centered at any specific location. It occurs everywhere simultaneously. Each observer sees distant galaxies receding, not because they are moving through space, but because the space between them is increasing. In a homogeneous universe, every region participates equally in the expansion, and the large-scale structure remains statistically uniform over time. The observable consequence is a redshift in the light from distant galaxies — a stretching of wavelengths that encodes the history of spatial expansion.

The strongest evidence for the geometry of the universe comes from the cosmic microwave background (CMB) — a relic radiation field that permeates all of space. The CMB originates from a time roughly 380,000 years after the Big Bang, when the universe cooled enough for protons and electrons to combine into neutral hydrogen atoms. This event, known as recombination, allowed photons to travel freely for the first time. The CMB is the redshifted remnant of that photon field, now observed at microwave wavelengths with a temperature of approximately 2.73 Kelvin.

The CMB is not perfectly uniform. It contains small anisotropies — tiny temperature fluctuations at the level of one part in 100,000. These fluctuations correspond to density variations in the early universe, which later seeded the formation of large-scale structures such as galaxies and clusters. The angular size of these fluctuations provides a direct measurement of spatial geometry. In particular, one can ask how large a primordial region appears on the sky today, given the time it took for light to reach us and the curvature of space through which it traveled.

Before recombination, the universe consisted of a hot, dense plasma of photons, electrons, and baryons. Photons scattered continuously off free electrons, coupling radiation tightly to matter. In this medium, density perturbations propagated as pressure waves — oscillations in the density and temperature of the plasma driven by the competition between gravitational infall and photon pressure. These waves are analogous to sound in air: compressions and rarefactions traveling through a fluid, though in this case, the fluid is a relativistic plasma and the dominant restoring force is radiation pressure. The term “acoustic oscillation” refers to these standing wave patterns, which left an imprint on the temperature distribution of the CMB when photon decoupling occurred.

The most important feature in the CMB spectrum is the first acoustic peak. This peak corresponds to the largest sound waves that had time to compress and rarefy a region of plasma before recombination. The physical size of such regions is determined by known physics — the speed of sound in the early universe and the duration before decoupling. However, the observed angular size depends on the spatial curvature of the universe. If space is positively curved, such a region appears larger than in flat geometry. If negatively curved, it appears smaller.

High-precision observations, particularly from the WMAP and Planck satellites, have measured the angular scale of the first acoustic peak to great accuracy. The result is consistent with a flat universe: the peak appears at an angle of approximately 1 degree, matching the prediction from a zero-curvature model. This agreement constrains the spatial curvature of the universe to be within approximately 0.4\% of flatness. Such precision transforms the notion of flatness from a philosophical ideal to an empirical conclusion.

A flat geometry implies that space, on its own, does not bend back onto itself or reach a spatial edge. In a flat model, space does not close like the surface of a sphere, nor does it require an external boundary. It continues indefinitely. This leads to an unsettling but mathematically coherent conclusion: if the universe is spatially flat everywhere, then it may be infinite in extent. No observational boundary marks the end of space; there is no enclosing shell, no edge beyond which space ceases to exist.

This conflicts with everyday reasoning. Intuition suggests that all things should be either bounded or looped back. But geometry, not intuition, governs spatial structure at cosmological scales. General relativity does not assign a preferred topology based on human expectations. It calculates curvature from energy density and evolves the metric accordingly. Whether or not the result “feels right” is irrelevant. The universe is not obligated to conform to habits of spatial imagination evolved in finite, curved environments.

The observable universe does have a limit — the particle horizon. This defines the maximum distance from which light has had time to reach us since the beginning of the metric expansion. Its current radius is approximately 46 billion light-years. However, this boundary is observational, not physical. It is not the edge of space, nor is it a wall. If the universe is flat and homogeneous beyond this radius, then it likely continues with the same statistical properties. Nothing in current data suggests a termination.

That infinite space can arise from a finite beginning may seem paradoxical, but this, too, is a confusion of categories. Time is not the same as space. The Big Bang represents a finite past moment — the origin of the metric and the expansion dynamics. But if space was already infinite at that time, it remained infinite as it expanded. The scale factor increased, but the global extent need not have changed from finite to infinite; it may always have been unbounded.

Although this section avoids detailing the composition of the universe, it is essential to understand that the model's flatness is not a theoretical preference. It is an outcome. Measurements of the cosmic microwave background, galaxy clustering, and large-scale structure all point to a consistent flat metric. The $\Lambda$CDM model achieves this without assuming it a priori. It infers flatness from data, within a well-defined dynamical framework.

\newpage
\begin{center}

\begin{tcolorbox}[
  colback=blue!3,
  colframe=blue!45,
  coltitle=white,
  fonttitle=\bfseries\large,
  title={$\Lambda$CDM Cosmology Timeline},
  sharp corners,
  boxrule=1pt,
  left=3pt,
  right=3pt,
  top=2pt,
  bottom=2pt,
  before upper={\footnotesize\setlength{\parskip}{2pt}},
  height=23cm,
  width=\textwidth
]

\textbf{1. Planck Era ($<10^{-43}$ s, $\sim10^{19}$ GeV)}\\
\textit{Forces:} Four forces unified. Quantum gravity dominates.\\
\textit{Matter:} No particles; quantum foam.\\
\textit{Scale:} Fluctuations at Planck length ($\sim10^{-35}$ m).

\textbf{2. Inflation ($10^{-36}$–$10^{-32}$ s)}\\
\textit{Forces:} Strong force separates from electroweak.\\
\textit{Matter:} Vacuum energy drives exponential expansion.\\
\textit{Scale:} Universe expands by factor $\geq10^{26}$; quantum fluctuations seed future galaxies.

\textbf{3. Reheating \& Baryogenesis ($10^{-32}$–$10^{-6}$ s)}\\
\textit{Forces:} Electroweak breaks into weak + electromagnetic.\\
\textit{Matter:} Quarks, leptons, gluons. CP violation creates 1:$10^9$ matter excess.\\
\textit{Temperature:} $10^{15}$–$10^{12}$ K.

\textbf{4. Quark–Gluon Plasma ($10^{-6}$–$10^{-5}$ s)}\\
\textit{Forces:} All four forces distinct.\\
\textit{Matter:} Quarks confine into protons and neutrons.\\
\textit{Temperature:} $\sim10^{13}$ K.

\textbf{5. Nucleosynthesis (1 s–20 min)}\\
\textit{Matter:} Protons and neutrons fuse: 75\% H, 25\% He, trace Li.\\
\textit{Scale:} Photons coupled to matter; uniform plasma.

\textbf{6. Photon Era (20 min–380,000 yr)}\\
\textit{Matter:} Ionized plasma. Neutrinos decouple at $\sim$1 s.\\
\textit{Temperature:} $10^9$ K $\rightarrow$ 3000 K.\\
\textit{Scale:} Density fluctuations ($\sim10^{-5}$) grow slowly.

\textbf{7. Recombination \& CMB (380,000 yr)}\\
\textit{Matter:} Electrons bind to nuclei $\rightarrow$ neutral atoms.\\
\textit{Scale:} Photons decouple $\rightarrow$ Cosmic Microwave Background.

\textbf{8. Dark Ages (0.4–0.5 Gyr)}\\
\textit{Matter:} Neutral hydrogen, cold dark matter clumps.\\
\textit{Scale:} Gravitational growth; no stars yet.

\textbf{9. First Stars \& Reionization (0.5–1 Gyr)}\\
\textit{Matter:} Nuclear fusion creates first stars (Pop III); heavier elements forged.\\
\textit{Scale:} Ionizing radiation clears fog; protogalaxies form.

\textbf{10. Galaxy Formation (1–5 Gyr)}\\
\textit{Scale:} Cold dark matter halos seed galaxies; cosmic web emerges (100 Mpc filaments).\\
\textit{Activity:} Quasars peak ($\sim$3 Gyr).

\textbf{11. Present (13.8 Gyr)}\\
\textit{Forces:} Dark energy ($\Lambda$) dominates since $\sim$5–6 Gyr.\\
\textit{Composition:} 69\% dark energy, 26\% cold dark matter, 5\% baryons.\\
\textit{Scale:} Hierarchical: stars $\rightarrow$ galaxies $\rightarrow$ clusters $\rightarrow$ superclusters. Galaxy separation $\sim$1 Mpc.

\textbf{12. Future ($\Lambda$ domination)}\\
\textit{10–100 Gyr:} Galaxies outside Local Group cross event horizon.\\
\textit{$10^{12}$–$10^{14}$ yr:} Star formation ceases.\\
\textit{$10^{15}$–$10^{100}$ yr:} Black holes evaporate; heat death.

\hrule
\vspace{2pt}
\textbf{Evolution Summary:} Unified forces $\rightarrow$ separate by $10^{-12}$ s. Quantum fields $\rightarrow$ quarks $\rightarrow$ hadrons $\rightarrow$ atoms $\rightarrow$ stars $\rightarrow$ galaxies. Planck scale $\rightarrow$ cosmic horizon. Radiation era $\rightarrow$ matter era $\rightarrow$ $\Lambda$ era.

\end{tcolorbox}
\end{center}
