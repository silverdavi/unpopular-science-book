\begin{SideNotePage}{
  \textbf{Top (ΛCDM Infinite Flat Universe):}  
  In the standard cosmological model, ΛCDM, space is spatially flat and infinite. Galaxies drift apart as distances expand — there is no center, no edge, and every point sees the same large-scale dynamics. The grid represents comoving coordinates: even as distances grow, local structure remains stationary in this frame. \par

  \textbf{Bottom (Compact Finite Universe):}  
  If space is compact—like a 3-torus or a hypersphere—then the universe can be finite in volume without having a boundary. Expansion still occurs, but the global topology loops back on itself. Light could, in principle, circle the cosmos. The grid warps around, indicating periodicity or curvature that reconnects spatial locations. \par
}{43_NearFlatUniverse/43_ Infinite Yet Born.pdf}
\end{SideNotePage}
