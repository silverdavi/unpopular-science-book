\begin{historical}
The debate over whether the universe had a beginning or existed eternally has shaped cosmology for over two thousand years. In classical antiquity, the dominant view — especially in Aristotelian physics — was eternalism: the cosmos had no origin, existing in a state of perpetual motion and balance. Aristotle’s model featured concentric spheres rotating around a stationary Earth, upheld by the idea that a perfect, eternal order governed the heavens. The notion of cosmic creation was seen as unnecessary, even philosophically inferior, to an eternal and self-contained universe.

This changed with the rise of monotheistic religions, which introduced a radically different concept: a universe created ex nihilo (from nothing), by a singular act of divine will. Medieval thinkers such as Augustine and Maimonides incorporated this creationist framework into their metaphysics, contrasting sharply with the Greek eternalist paradigm. However, for centuries, this remained a theological stance, largely separate from natural philosophy.

Modern cosmology inherited this tension. When Albert Einstein formulated general relativity in 1915, he initially introduced the cosmological constant \(\Lambda\) to maintain a static universe — an implicit nod to eternalism. Yet in the 1920s, Alexander Friedmann and Georges Lemaître independently found that Einstein’s equations naturally described an expanding cosmos. Lemaître, a Belgian priest and physicist, explicitly interpreted this expansion as evidence of a beginning — a "day without yesterday." His model, known as the “primeval atom,” implied a definite origin in time.

This idea clashed with the philosophical preferences of many physicists. Fred Hoyle, Hermann Bondi, and Thomas Gold proposed the steady-state model in 1948, maintaining that the universe had always existed and would continue to expand eternally, with matter continuously created to preserve density. Hoyle coined the term “Big Bang” as a dismissive label for Lemaître’s model, viewing it as tainted by religious overtones.

Ironically, it was empirical evidence that vindicated the “creationist” model. The discovery of the cosmic microwave background in 1965 by Penzias and Wilson provided direct observational support for a hot, dense early universe — an echo of its origin. This shifted the consensus dramatically. What began as a scientifically controversial, seemingly theological notion — that the universe had a beginning — became the foundation of modern cosmology. Today’s standard model, Lambda-CDM, descends directly from this creation-based framework, though now couched in empirical and mathematical precision rather than metaphysical doctrine.
\end{historical}
