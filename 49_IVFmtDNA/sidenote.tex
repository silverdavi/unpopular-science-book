\begin{SideNotePage}{
  \textbf{Top (Mitochondrial Replacement Therapy):}  
  A fertilized nucleus is inserted into a donor egg with healthy mitochondria, creating an embryo with parental nuclear DNA and donor mtDNA—preventing transmission of mitochondrial disorders.

  \textbf{Middle Left (CCR5 and HIV Resistance):}  
  HIV entry requires CCR5. Individuals with the CCR5-Δ32 mutation lack a functional receptor, blocking viral entry. This natural resistance underpins CCR5-targeted therapies.

  \textbf{Middle Right (CRISPR-Cas9 Genome Editing):}  
  Cas9 cuts DNA at a targeted site guided by RNA. This enables correction or knockout of specific genes, forming the basis of precise gene therapy.

  \textbf{Bottom Left (Embryonic Germline Editing):}  
  CRISPR introduced at the zygote stage edits all descendant cells, including germline—causing permanent, heritable changes. Ethically and legally restricted.

  \textbf{Bottom Right (Embryo Development):}  
  Post-fertilization, the embryo progresses through cleavage stages to blastocyst
}{49_IVFmtDNA/49_ Tattered.pdf}
\end{SideNotePage}
