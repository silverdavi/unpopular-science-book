\begin{technical}
{\Large\textbf{Micromanipulation and Mitochondrial Heteroplasmy}}\\[0.7em]

\noindent\textbf{Technical Challenges in Nuclear Transfer}

Maternal spindle transfer requires extracting the metaphase II spindle-chromosome complex without disrupting chromosome alignment. The spindle, a 15–20 μm birefringent structure visible under polarized light (Oosight™), must be aspirated with minimal surrounding cytoplasm. Tachibana et al. (2013) demonstrated this technique in rhesus macaques, achieving no detectable donor mtDNA carryover and producing healthy, fertile offspring.

Critical technical parameters:
\begin{itemize}[leftmargin=*]
\item \textbf{Timing}: Within 2 hours of oocyte retrieval before spindle depolymerization
\item \textbf{Pipette diameter}: 20 μm beveled, approaching at 30° to minimize membrane deformation
\item \textbf{Cytoplasmic volume}: <5 picoliters co-aspirated to keep mitochondrial carry-over below 2\%
\item \textbf{Fusion method}: Electrofusion (1.0 kV/cm, 50 μs pulses) or HVJ-E (Sendai virus extract)
\end{itemize}

Pronuclear transfer exploits the 8–10 hour window post-fertilization when pronuclei are visible but unfused. Both pronuclei must be extracted together within a karyoplast — a membrane-bound cytoplasmic package containing ~5\% of oocyte volume. This preserves their relative positioning, which carries epigenetic information essential for proper development.

\noindent\textbf{Mitochondrial Carry-over and Detection}

Even stringent micromanipulation cannot eliminate all donor mitochondria. Hyslop et al. (2016) reported carry-over levels in human embryos:
\begin{itemize}[leftmargin=*]
\item PNT: typically <2\%, occasionally higher
\item MST: 1–2\% average (range 0.5–4\%)
\item PB1T: <0.5\% (minimal cytoplasm in polar body)
\end{itemize}

Deep sequencing now achieves 0.1\% sensitivity, with digital droplet PCR detecting heteroplasmy as low as 0.01\%.

\noindent\textbf{Heteroplasmy Dynamics in Development}

Low-level heteroplasmy behaves unpredictably during development due to:

\textbf{The mitochondrial bottleneck}: During oogenesis, mtDNA copy number drops from ~100,000 in mature oocytes to ~200 in primordial germ cells before clonal expansion. This bottleneck allows random genetic drift to dramatically shift heteroplasmy levels between generations.

\textbf{Tissue-specific segregation}: Post-mitotic tissues show divergent heteroplasmy patterns:
\begin{itemize}[leftmargin=*]
\item \textbf{Muscle}: Can amplify from 1\% to 50\% by adulthood
\item \textbf{Blood}: Typically maintains stable levels
\item \textbf{Brain}: Shows regional variation, with high-energy regions (substantia nigra) potentially enriching for wild-type mtDNA
\end{itemize}


\noindent\textbf{Nuclear-Mitochondrial Compatibility}

The mitochondrial proteome comprises ~1,500 proteins: 13 encoded by mtDNA, the remainder nuclear-encoded and imported. OXPHOS complexes require precise stoichiometry between nuclear and mitochondrial subunits.

Potential incompatibilities arise where mtDNA-encoded subunits must interact with nuclear-encoded partners (e.g., Complex I: 7 mitochondrial, 37 nuclear subunits). However, the Tachibana primate studies showed reassuring cross-species compatibility, with healthy offspring from rhesus nuclear DNA and cynomolgus mtDNA.

\noindent\textbf{Clinical Implementation}

Clinical implementation follows the Newcastle protocol: women with >20\% pathogenic mtDNA mutations undergo MST (preferred for higher success rates), accepting <5\% carry-over. The Hyslop study achieved blastocyst development in most reconstructed human embryos, though with lower efficiency than controls. Current clinical outcomes show normal development in children born after mitochondrial replacement, but the oldest are <10 years.

\vspace{0.5em}
\noindent\textbf{References}\\
Tachibana, M. \textit{et al.} (2013). Towards germline gene therapy of inherited mitochondrial diseases. \textit{Nature} \textbf{493}, 627–631.\\
Hyslop, L. A. \textit{et al.} (2016). Towards clinical application of pronuclear transfer to prevent mitochondrial DNA disease. \textit{Nature} \textbf{534}, 383–386.
\end{technical}