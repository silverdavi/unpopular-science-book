\begin{technical}
{\Large\textbf{Micromanipulation and Mitochondrial Heteroplasmy}}\\[0.7em]

\noindent\textbf{Technical Challenges in Nuclear Transfer}

Maternal spindle transfer requires extracting the metaphase II spindle-chromosome complex without disrupting chromosome alignment. The spindle, a 15–20 μm birefringent structure visible under polarized light (Oosight™), must be aspirated with minimal surrounding cytoplasm to reduce mitochondrial carry-over.

Critical technical parameters:
\begin{itemize}[leftmargin=*]
\item \textbf{Timing}: Within 2 hours of oocyte retrieval before spindle depolymerization
\item \textbf{Pipette diameter}: 20 μm beveled, approaching at 30° to minimize membrane deformation
\item \textbf{Cytoplasmic volume}: <5 picoliters co-aspirated to keep mitochondrial carry-over below 2\%
\item \textbf{Fusion method}: Electrofusion (1.0 kV/cm, 50 μs pulses) or HVJ-E (Sendai virus extract)
\end{itemize}

Pronuclear transfer exploits the 8–10 hour window post-fertilization when pronuclei are visible but unfused. Both pronuclei must be extracted together within a karyoplast — a membrane-bound cytoplasmic package containing ~5\% of oocyte volume. This preserves their relative positioning, which carries epigenetic information essential for proper development.

\noindent\textbf{Mitochondrial Carry-over and Detection}

Even stringent micromanipulation cannot eliminate all donor mitochondria. Deep sequencing reveals carry-over levels:
\begin{itemize}[leftmargin=*]
\item MST: 1–2\% average (range 0.5–4\%)
\item PNT: 0.5–2\% average
\item PB1T: <0.5\% (minimal cytoplasm in polar body)
\end{itemize}

Detection methods have evolved from restriction fragment length polymorphism (RFLP) analysis to next-generation sequencing with 0.1\% sensitivity. Digital droplet PCR can detect heteroplasmy as low as 0.01\%.

\noindent\textbf{Heteroplasmy Dynamics in Development}

Low-level heteroplasmy behaves unpredictably during development due to:

\textbf{The mitochondrial bottleneck}: During oogenesis, mtDNA copy number drops from ~100,000 in mature oocytes to ~200 in primordial germ cells before clonal expansion. This bottleneck allows random genetic drift to dramatically shift heteroplasmy levels between generations.

\columnbreak

\textbf{Tissue-specific segregation}: Post-mitotic tissues show divergent heteroplasmy patterns:
\begin{itemize}[leftmargin=*]
\item \textbf{Muscle}: Can amplify from 1\% to 50\% by adulthood
\item \textbf{Blood}: Typically maintains stable levels
\item \textbf{Brain}: Shows regional variation, with high-energy regions (substantia nigra) potentially enriching for wild-type mtDNA
\end{itemize}

\textbf{Replicative segregation}: Mutant mtDNA may have replicative advantages due to:
\begin{itemize}[leftmargin=*]
\item Smaller genome size (deletions replicate faster)
\item Relaxed replication control
\item Resistance to mitophagy signals
\end{itemize}

\noindent\textbf{Nuclear-Mitochondrial Compatibility}

The mitochondrial proteome comprises ~1,500 proteins: 13 encoded by mtDNA, the remainder nuclear-encoded and imported. OXPHOS complexes require precise stoichiometry between nuclear and mitochondrial subunits.

Potential incompatibilities:
\begin{itemize}[leftmargin=*]
\item \textbf{Complex I}: 7 mtDNA-encoded subunits must interact with 37 nuclear subunits
\item \textbf{Complex IV}: Subunit interactions optimized over evolutionary time
\item \textbf{Ribosomal proteins}: Nuclear-encoded proteins must recognize mitochondrial rRNAs
\end{itemize}

Primate studies show reassuring cross-species compatibility (rhesus nuclear DNA with cynomolgus mtDNA produces healthy offspring). Human haplogroup mixing appears tolerated, though subtle metabolic effects may emerge under stress or aging.

\noindent\textbf{Clinical Implementation}

The Newcastle Fertility Centre protocol:
\begin{itemize}[leftmargin=*]
\item Patient selection: Women with >20\% pathogenic mtDNA mutations
\item Technique: MST preferred for higher success rates
\item Heteroplasmy threshold: Accept <5\% carry-over
\item Follow-up: Annual assessment including blood/urine heteroplasmy
\end{itemize}

Current outcomes show normal development in most children, but the oldest are <10 years. Long-term risks remain theoretical.

\vspace{0.5em}
\noindent\textbf{References}\\
Tachibana, M. \textit{et al.} (2013). Towards germline gene therapy of inherited mitochondrial diseases. \textit{Nature} \textbf{493}, 627–631.\\
Hyslop, L. A. \textit{et al.} (2016). Towards clinical application of pronuclear transfer to prevent mitochondrial DNA disease. \textit{Nature} \textbf{534}, 383–386.
\end{technical}