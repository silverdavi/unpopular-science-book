\begin{technical}
{\Large\textbf{Mechanistic Basis of Mitochondrial Replacement Techniques (MRT)}}\\[0.7em]

Mitochondrial replacement techniques (MRT) enable the reconstruction of oocytes or zygotes by transferring nuclear material from a carrier of mitochondrial mutations into a donor cytoplasm containing healthy mitochondria. Five principal approaches have been developed, distinguished by the developmental stage at which transfer occurs and the nuclear material selected.

\vspace{0.4em}

\textbf{Germinal Vesicle Transfer (GVT)}\\
In GVT, primary oocytes arrested in prophase I are used. The maternal oocyte is first immobilized and the germinal vesicle (intact nucleus surrounded by nuclear envelope) is isolated by micropipette aspiration. The donor oocyte, matched for maturation stage, is enucleated by removing its own germinal vesicle. The maternal germinal vesicle is then transferred into the enucleated donor oocyte. After cytoplasmic fusion (typically induced by Sendai virus extract or electrofusion), the reconstructed oocyte undergoes in vitro maturation to reach metaphase II before fertilization. GVT is technically demanding due to the fragility of immature chromatin and variable maturation competence.

\vspace{0.4em}

\textbf{Maternal Spindle Transfer (MST)}\\
MST is performed on metaphase II oocytes immediately before fertilization. The maternal spindle-chromosome complex is visualized using polarized light microscopy or fluorescent DNA dyes, then aspirated with a micromanipulation pipette. The donor oocyte, also at metaphase II, is enucleated by spindle removal. The maternal spindle is introduced into the enucleated donor oocyte and cytoplasmic fusion is induced. Fertilization follows, typically via intracytoplasmic sperm injection (ICSI) to avoid polyspermy risks. MST achieves high rates of blastocyst formation and minimizes mitochondrial carry-over, usually below 2–3\%.

\vspace{0.4em}

\textbf{Pronuclear Transfer (PNT)}\\
PNT occurs at the zygote stage, post-fertilization but before pronuclear fusion. Both the maternal and donor oocytes are fertilized with sperm. Using micromanipulation, both pronuclei (female and male) are extracted together from the maternal zygote. The donor zygote is simultaneously enucleated, removing its pronuclei while preserving cytoplasm. The extracted pronuclei are transferred into the donor cytoplasm. Cell membrane fusion is induced, and the reconstructed zygote proceeds to normal cleavage. PNT has robust developmental outcomes but raises ethical considerations due to the creation and sacrifice of donor zygotes.

\vspace{0.4em}

\textbf{First Polar Body Transfer (PB1T)}\\
In PB1T, the first polar body, extruded at metaphase II, is harvested from the maternal oocyte. This haploid genome, corresponding to the chromosomes destined for discard during meiosis I, is microinjected into an enucleated donor oocyte. The reconstructed oocyte is fertilized thereafter. PB1T benefits from minimal cytoplasmic contamination because the polar body contains little cytoplasm, but its biological equivalence to the egg genome is still under investigation.

\vspace{0.4em}

\textbf{Second Polar Body Transfer (PB2T)}\\
PB2T targets the second polar body, extruded after fertilization. The second polar body from the maternal zygote is isolated and injected into a donor zygote that has had its female pronucleus removed. This method parallels pronuclear transfer but avoids transferring the entire zygotic cytoplasm. It offers a strategy for minimizing mitochondrial carry-over, though technical complexity and limited clinical validation remain challenges.

\vspace{0.4em}

\textbf{Key Variables, Risks, and Biological Considerations}\\
The success of MRT procedures is primarily evaluated by blastocyst formation rates, reflecting the proportion of reconstructed embryos reaching advanced developmental stages, and by mitochondrial carry-over levels, measured as the percentage of maternal mtDNA retained and typically quantified by PCR-based or sequencing assays. Carry-over below 5\% is generally regarded as acceptable for preventing disease manifestation. Additional parameters, including spindle integrity, chromosome segregation fidelity, and embryo ploidy status, are increasingly incorporated into assessments to ensure procedural safety. 

MRT carries inherent risks: mechanical disruption during micromanipulation can destabilize the spindle apparatus and elevate aneuploidy rates; low-level mitochondrial heteroplasmy, even when initially minimal, may undergo replicative expansion during development; and nuclear-mitochondrial mismatches, arising from the introduction of a foreign mitochondrial background, may disrupt coadapted genome interactions, although clinical significance appears limited based on current human data.

Standardized protocols, operator experience, and careful patient selection are critical for minimizing risks and ensuring the clinical utility of MRT.
\end{technical}
