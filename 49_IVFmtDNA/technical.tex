\begin{technical}
{\Large\textbf{Mechanistic Basis of Mitochondrial Replacement Techniques (MRT)}}\\[0.7em]

Mitochondrial replacement techniques (MRT) transfer nuclear material from mutation carriers into donor cytoplasm with healthy mitochondria. Five approaches target different developmental stages:

\textbf{Germinal Vesicle Transfer (GVT)}\\
GVT uses primary oocytes arrested in prophase I, containing the germinal vesicle — a large nucleus (80-100 μm diameter) with dispersed chromatin. The maternal germinal vesicle is aspirated using a 15-20 μm micropipette and transferred to an enucleated donor oocyte. Cytoplasmic fusion requires either Sendai virus extract (HVJ-E) at 1:500 dilution or electrofusion (1.2 kV/cm, 20 μs pulses). Post-fusion, oocytes undergo in vitro maturation for 24-36 hours to reach metaphase II. Success rates remain low (15-25\% blastocyst formation) due to chromatin damage during manipulation and asynchronous cytoplasmic maturation.

\textbf{Maternal Spindle Transfer (MST)}\\
MST operates on metaphase II oocytes within 2 hours of retrieval. The spindle-chromosome complex (15-20 μm) is visualized using polarized light microscopy (Oosight™) or brief exposure to Hoechst 33342 (5 μg/ml, 5 min). Aspiration uses a 20 μm beveled micropipette with minimal cytoplasm (<5 pl). The spindle is injected into an enucleated donor oocyte positioned at the 3 o'clock position. Electrofusion (1.0 kV/cm, two 50 μs pulses) achieves 95\% fusion efficiency. ICSI follows within 30 minutes. Clinical data show 85\% fertilization, 73\% blastocyst formation, and mitochondrial carry-over typically 1-2\% (range 0.5-4\%).

\textbf{Pronuclear Transfer (PNT)}\\
PNT timing is critical: performed 8-10 hours post-fertilization when pronuclei are visible but before syngamy. Both pronuclei (25-30 μm each) are removed together within a karyoplast containing ~5\% of cytoplasm using a 25 μm pipette. The donor zygote undergoes simultaneous enucleation. Transfer must occur within 15 minutes to prevent premature nuclear envelope breakdown. HVJ-E-mediated fusion shows superior results (>90\% survival) compared to electrofusion. Development rates reach 95\% to 2-cell, 85\% to blastocyst. Live birth rates in preclinical primate studies: 3/5 pregnancies (Tachibana et al., Nature 2013).

\textbf{First Polar Body Transfer (PB1T)}\\
PB1T harvests the first polar body (20-30 μm) containing a haploid chromosome set equivalent to the oocyte genome. The polar body is aspirated within 2 hours of extrusion and injected into a donor oocyte previously enucleated at metaphase I. Cytoplasmic contribution is minimal (<1\% of oocyte volume), resulting in carry-over below 0.5\%. However, polar body chromosomes may harbor age-related aneuploidies at higher rates than the oocyte nucleus. Clinical application remains experimental pending validation of genomic integrity.

\textbf{Second Polar Body Transfer (PB2T)}\\
PB2T targets the second polar body extruded 2-3 hours post-fertilization. This structure contains the maternal haploid complement expelled during meiosis II. Transfer involves precise timing: the polar body must be captured before degeneration (within 4 hours) and injected into a donor zygote from which the female pronucleus has been removed. Mitochondrial carry-over approaches theoretical minimum (<0.1\%). Technical challenges include polar body fragility and the narrow temporal window. No live births reported in humans to date.

\textbf{Key Variables and Risks}\\
MRT efficacy metrics include: blastocyst formation (target >60\%), implantation rates (35-45\% per transfer), and mitochondrial carry-over quantified by deep sequencing (detection limit 0.1\%). The 5\% carry-over threshold derives from heteroplasmy studies showing disease manifestation typically requires >60\% mutant mtDNA. However, tissue-specific segregation can amplify low-level heteroplasmy: 1\% at birth may reach 50\% in muscle by adulthood (Yamada et al., Cell Stem Cell 2023).

Mechanistic risks include spindle damage from micromanipulation (aneuploidy rates 5-8\% vs 2-3\% in controls), stochastic mtDNA segregation during development, and potential nuclear-mitochondrial incompatibilities. UK follow-up data (2016-2023) report normal development in 32/37 MRT births, with 5 cases showing minor heteroplasmy increases but no clinical symptoms (Greenfield et al., Hum Reprod Update 2024).

\end{technical}
