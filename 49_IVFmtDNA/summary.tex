Every human inherits two distinct genomes: nuclear DNA from both parents and mitochondrial DNA almost exclusively from the mother. This second genome — 37 genes controlling cellular energy production — mutates 10-100 times faster than nuclear DNA, causing devastating diseases when defective. Traditional IVF cannot prevent mothers from passing faulty mitochondria to children. Enter mitochondrial replacement therapy: scientists transfer nuclear DNA from an affected mother's egg into a donor egg with healthy mitochondria. From single-base edits to chromosome transfers — many ethical questions arise to be discussed.