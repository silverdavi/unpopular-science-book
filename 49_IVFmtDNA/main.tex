Fertilization in humans begins when a sperm cell successfully penetrates the outer membrane of the oocyte. Upon entry, the sperm delivers its haploid set of 23 chromosomes into the oocyte cytoplasm. The oocyte, already arrested in metaphase II of meiosis, completes its meiotic division and expels the second polar body. The fusion of the male and female pronuclei forms a single diploid nucleus, establishing the genomic foundation of the zygote.

Following fertilization, the zygote undergoes a series of rapid mitotic divisions known as cleavage. These divisions increase cell number without increasing the overall size of the embryo, partitioning the cytoplasm into progressively smaller blastomeres. Around the 16- to 32-cell stage, the embryo compacts to form a morula, and by the fifth to sixth day post-fertilization, a fluid-filled cavity called the blastocoel develops, creating a blastocyst. The blastocyst consists of an inner cell mass, destined to form the embryo, and an outer trophoblast layer that facilitates implantation into the uterine endometrium.

The nuclear DNA of the zygote comprises 46 chromosomes, organized into 23 homologous pairs. One chromosome of each pair is inherited from the mother via the oocyte, and the other from the father via the sperm. These chromosomes encode the genetic information required for human development, regulating processes from cell cycle control to tissue differentiation. The nuclear genome is distributed across the nucleus of each embryonic cell, and its faithful replication is critical for maintaining genomic integrity throughout embryogenesis.

In addition to nuclear DNA, each human cell contains mitochondria in the cytoplasm that produce cellular energy. Within each mitochondrion exists a small, circular DNA molecule known as mitochondrial DNA (mtDNA). Unlike nuclear DNA, which is packaged into chromosomes within the nucleus, mtDNA is physically separate and exists in multiple copies per mitochondrion. The mitochondria are inherited maternally through the oocyte's cytoplasm.

Mitochondrial DNA encodes components for oxidative phosphorylation, the biochemical pathway that generates adenosine triphosphate (ATP) through electron transport and proton gradient-driven synthesis. Specifically, mtDNA contains 37 genes: 13 encoding protein subunits of the respiratory chain complexes, 22 encoding transfer RNAs, and 2 encoding ribosomal RNAs necessary for mitochondrial protein synthesis. These elements are indispensable for cellular metabolism, particularly in tissues with high energy demands such as muscle, brain, and heart.

Unlike nuclear DNA, mtDNA is highly susceptible to mutations. It lacks protective histone proteins, possesses limited DNA repair mechanisms, and sits near the electron transport chain, a major source of reactive oxygen species. These conditions result in a mutation rate for mtDNA that is approximately 10 to 100 times higher than that of nuclear DNA. Mutations in mtDNA accumulate over time and can disrupt the efficiency of oxidative phosphorylation, impairing cellular energy production.

Mutations in mitochondrial DNA cause diseases characterized by impaired energy metabolism. Because mitochondria are responsible for supplying the majority of cellular ATP, defects in oxidative phosphorylation have the greatest impact on tissues with high metabolic demands. Clinical manifestations include neurological disorders such as encephalopathy and seizures, muscular disorders such as myopathy and exercise intolerance, cardiomyopathies, and sensory deficits including optic neuropathy and hearing loss. The severity of these diseases often correlates with the proportion of mutated mtDNA within affected cells, a condition known as heteroplasmy, and with the energy thresholds required by different tissues.

In vitro fertilization (IVF) combines retrieved oocytes and prepared sperm outside the human body under controlled laboratory conditions. The process begins with controlled ovarian hyperstimulation, during which exogenous gonadotropins are administered to stimulate the development of multiple follicles. Once sufficient follicular maturation is confirmed by ultrasound and hormone measurements, oocyte retrieval is performed transvaginally under ultrasound guidance. Retrieved oocytes are assessed for maturity and subsequently exposed to motile sperm, either by conventional insemination or by intracytoplasmic sperm injection (ICSI), in which a single sperm cell is mechanically introduced into the oocyte cytoplasm.

Fertilized embryos are cultured in specialized media supporting preimplantation development. Embryos are monitored for cleavage patterns, morphology, and progression to the blastocyst stage, typically over a period of five to six days. Selection criteria based on morphological quality and developmental timing guide the choice of embryos for transfer. One or more embryos are transferred into the uterine cavity using a catheter, aiming to establish implantation and initiate a clinical pregnancy. Remaining viable embryos may be cryopreserved for future use.


In standard in vitro fertilization (IVF) procedures, the oocyte’s cytoplasm, including its mitochondrial content, is transmitted unchanged to the resulting embryo. Because mitochondria and mitochondrial DNA (mtDNA) are maternally inherited, IVF does not prevent the transmission of pathogenic mtDNA mutations. Mothers with defective mtDNA pass these mutations to offspring through the oocyte cytoplasm in both natural conception and IVF.

Mitochondrial replacement therapies (MRT) prevent the transmission of mutated mtDNA. These techniques involve transferring the nuclear genetic material from an oocyte or zygote carrying pathogenic mtDNA into a donor cytoplasm containing healthy mitochondria. Two methods exist: maternal spindle transfer, performed before fertilization, and pronuclear transfer, performed after fertilization but before pronuclear fusion. Both methods aim to preserve the intended parents' nuclear genome while replacing the defective mitochondrial population with functional donor-derived mitochondria.

The strict maternal inheritance of mtDNA involves active mechanisms to eliminate paternal mitochondria, including degradation during spermatogenesis and post-fertilization mitophagy.

Rare reports have described cases of paternal mtDNA transmission. High-throughput sequencing technologies occasionally detect mtDNA sequences in offspring that do not match the maternal lineage, suggesting a potential contribution from the father. Initial interpretations proposed that paternal mitochondria might occasionally evade elimination mechanisms and be transmitted to the offspring at detectable levels.

 These observations often result from nuclear mitochondrial DNA segments (NUMTs) — fragments of mtDNA incorporated into the nuclear genome over evolutionary time that closely resemble true mitochondrial sequences. NUMTs are inherited in a Mendelian fashion and can be misinterpreted as paternal mtDNA when standard sequencing techniques co-amplify nuclear and mitochondrial DNA. Detailed studies have shown that many apparent cases of biparental mtDNA inheritance are artifacts caused by the presence of large, recently inserted NUMTs in the nuclear genome rather than true transmission of paternal mitochondria.


\begin{commentary}[On the Semantics of Genetic Parentage]
In laboratory settings, mitochondrial DNA replacement and more complex genomic interventions — ranging from whole chromosome transfers to single-base CRISPR edits — are becoming routine. Popular discourse often responds with labels such as “three-parent baby,” referring to cases where nuclear DNA comes from two individuals and mtDNA from a third. As such procedures proliferate and more nuanced manipulations emerge, questions like "who are the parents?" or "how many parents are there if 10\% of the genome is replaced?" become ill-posed.

This is not unlike the epistemological shift that occurred in physics a century ago. Questions such as “what is light?” or “what is an electron?” gave way to rigorously framed operational questions: “what signal will appear on a detector under specified experimental conditions?" Biology is undergoing a similar transition. Rather than asking “who are the parents?”, the relevant question becomes “what proportion of the genome is shared with each contributor?” Genetic parentage, in this view, becomes a quantitative map of biological contribution. This technical definition intentionally separates the biological facts from the equally valid concepts of social and emotional parentage, which are defined orthogonally by nurture, commitment, and care.
\end{commentary}
\clearpage

\vfill
\begin{shadedstory}[The CRISPR Experiment That Led to Prison]

In late 2018, a Chinese biophysicist named He Jiankui announced the birth of twin girls whose genomes had been edited at the embryonic stage. Using the CRISPR-Cas9 system, He targeted the CCR5 gene, seeking to introduce a mutation associated with resistance to HIV infection. The announcement, made through YouTube videos and public statements rather than peer-reviewed scientific channels, bypassed established academic and regulatory norms. This triggered international condemnation from scientists, ethicists, and government bodies.\\

The CCR5 gene, while involved in HIV susceptibility, also participates in brain development, immune response regulation, and other physiological processes. Editing this gene without comprehensive knowledge of its systemic effects introduced unknown biological risks. Established medical procedures such as sperm washing already enabled HIV-positive parents to conceive healthy children, making the intervention medically unnecessary.\\

 The consent forms misled participating families about the experimental nature and unknown risks of the intervention. Off-target edits — unintended mutations elsewhere in the genome — were not assessed before embryo implantation. The CRISPR modifications were heritable, meaning any unintended effects would be transmitted to future generations without their consent.\\

In December 2019, Chinese authorities sentenced He Jiankui to three years in prison for "illegal medical practices," alongside financial penalties and professional bans. Two of his collaborators received lesser sentences. The incident marked the first criminal prosecution for human germline genome editing and catalyzed global discussions about regulatory frameworks, ethical standards, and the governance of emerging biotechnologies.\\

The case shifted focus from whether human germline editing was technically achievable to the principles under which such interventions should be permitted. As with the regulation of nuclear physics a century earlier, the expansion of technological capability in genetics forced the construction of new ethical, legal, and societal structures to manage its consequences.

\end{shadedstory}
\vfill
