\begin{historical}

The effort to prevent the maternal transmission of mitochondrial DNA (mtDNA) diseases originated from the recognition, during the late twentieth century, that certain debilitating disorders such as Leigh syndrome, MELAS, and Leber’s hereditary optic neuropathy were caused by mutations in the mitochondrial genome. Unlike nuclear genetic diseases, mitochondrial disorders presented a unique challenge: strict maternal inheritance, random bottleneck effects during oogenesis, and heteroplasmy complicated genetic counseling and prediction of disease severity.

Early experiments in the 1990s explored the feasibility of cytoplasmic transfer between oocytes, aiming to improve oocyte competence rather than prevent disease. These procedures, known as ooplasmic transfer, resulted in the births of children carrying a mixture of maternal and donor mitochondria, raising ethical and regulatory concerns. In 2001, the U.S. Food and Drug Administration (FDA) halted cytoplasmic transfer procedures, citing insufficient safety data and concerns regarding heritable genetic modification.

Scientific focus then shifted toward targeted nuclear transfer techniques. In 2009, Tachibana et al. demonstrated in rhesus macaques that meiotic spindle transfer (MST) could successfully prevent the transmission of maternal mtDNA mutations without compromising embryo viability. This work provided the first preclinical evidence supporting mitochondrial replacement as a viable therapeutic strategy.

Pronuclear transfer (PNT), originally demonstrated in murine models in 1983, was adapted for use in human embryos by Craven et al. in 2010. Subsequent refinements by research groups in the United Kingdom and United States established that both MST and PNT could achieve low levels of mitochondrial carry-over and support normal embryonic development to the blastocyst stage.

In 2015, the United Kingdom became the first country to formally legalize mitochondrial replacement therapies under strict regulatory frameworks, following extensive public consultation and scientific review by the Human Fertilisation and Embryology Authority (HFEA). Clinical licenses were granted on a case-by-case basis for preventing the transmission of serious mitochondrial diseases.

The first reported live birth resulting from MRT occurred in 2016 via spindle transfer, performed by a clinical team led by Dr. John Zhang. The procedure was carried out partially in the United States and partially in Mexico to circumvent regulatory barriers, marking a controversial milestone in the field.

Parallel developments occurred at the Nadiya Clinic in Kyiv, Ukraine. In 2016–2017, researchers led by Dr. Valery Zukin and Dr. Paul Mazur implemented pronuclear transfer protocols adapted for infertility treatments, reporting multiple pregnancies and births. 

Research efforts have since expanded to include polar body transfer (PB1T and PB2T) as alternative MRT strategies, aiming to further minimize mitochondrial carry-over and ethical concerns related to zygote destruction. Long-term follow-up studies and multi-generational observations remain essential to fully assess the safety, efficacy, and societal implications of mitochondrial replacement technologies.

\end{historical}
