\begin{SideNotePage}{
  \textbf{Top (Diffusion Model of Osmosis):}  
  Historically, osmosis was explained as a type of diffusion: solute concentration differences cause water to move from low to high solute concentration to "even things out." This intuitive model treats the membrane as a passive barrier and water flow as driven by statistical mixing. \par

  \textbf{Second (Gas Pressure Analogy):}  
  Water molecules are thought of as a vapor-like phase. The side with more solute has fewer free water molecules, reducing its effective vapor pressure. This creates a pressure imbalance across the membrane, driving water toward the more concentrated side. \par

  \textbf{Third (Virial Theorem Approach):}  
  Here, osmotic pressure is derived from molecular interactions—akin to pressure in gases. Solute particles exert directional momentum transfer through collisions, and the semi-permeable membrane selectively blocks these, resulting in net force buildup. \par

  \textbf{Fourth (Chemical Potential Explanation):}  
  Osmosis is now more rigorously understood in terms of chemical potential gradients. Water flows from high to low chemical potential, and solutes lower the chemical potential of water. This framework aligns with thermodynamic definitions and governs equilibrium conditions. \par

  \textbf{Bottom (Membrane Force Model – Debye's View):}  
  In this modern mechanistic picture, the membrane itself plays an active role. It exerts differential mechanical forces on solute and solvent. Osmotic flow arises due to water being pulled across in response to these localized interactions, not because of bulk diffusion or vapor analogy. \par
}{31_osmosis_Debye/31_ Concentrate on Osmosis.pdf}
\end{SideNotePage}
