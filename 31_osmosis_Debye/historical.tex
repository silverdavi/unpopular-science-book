\begin{historical}
The term "osmose" was introduced by Jean-Antoine Nollet in 1748 after observing fluid movement through animal membranes. In the 1880s, Jacobus van 't Hoff derived a quantitative expression for osmotic pressure in dilute solutions that followed the same form as the ideal gas law. The result linked the behavior of solutes in solution to molecular motion, reinforcing the emerging statistical view of thermodynamics.

Pfeffer sealed sugar solution inside a porous ceramic pot lined with copper ferrocyanide, creating the first truly semipermeable membrane. He placed the pot in pure water and connected it to a mercury manometer (a pressure gauge). The mercury column climbed as water entered the pot, sometimes generating pressures over 20 atmospheres.

This was the first quantitative demonstration of osmotic pressure as a real mechanical force, measurable in the same way as gas or hydrostatic pressure. Van 't Hoff immediately recognized the analogy to ideal gases and used Pfeffer's results to formulate $\Pi = cRT$.

By the early 20th century, osmosis was widely interpreted through the lens of diffusion: water was thought to move from regions of high to low concentration. However, alternative models emerged. In 1908, Lars Vegard proposed that solutes excluded from a membrane could generate local pressure differences. Peter Debye formalized this idea in 1923, modeling how solute collisions with a semipermeable barrier result in a force imbalance that drives water flow. Debye's model treated osmotic flow as mechanically generated rather than purely thermodynamic.

Although consistent with van ’t Hoff’s law at equilibrium, Debye’s explanation emphasized local interactions at the membrane interface. The model was largely ignored in favor of equilibrium thermodynamics until it was revisited in the late 20th and early 21st centuries by researchers such as Gerald Manning and Alan Kay. Their work highlighted a discrpenacy between persistent yet inaccurate textbook descriptions and a the physical theory of osmosis.
\end{historical}
