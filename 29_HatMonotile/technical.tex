\begin{technical}
{\Large\textbf{Transfer Matrix Method for Domino Tilings}}\\[0.7em]

\noindent\textbf{Problem:} Count tilings of a $3 \times 2n$ cylinder with $1 \times 2$ dominoes.

\noindent\textbf{Method:} View the tiling column by column. Each column's state describes which cells are occupied by dominoes from the previous column. Valid states (0 = empty, 1 = occupied):
\[
|0\rangle = \begin{pmatrix}0\\0\\0\end{pmatrix}, \quad
|1\rangle = \begin{pmatrix}0\\1\\1\end{pmatrix}, \quad
|2\rangle = \begin{pmatrix}1\\0\\0\end{pmatrix}
\]
\[
|3\rangle = \begin{pmatrix}1\\0\\1\end{pmatrix}, \quad
|4\rangle = \begin{pmatrix}1\\1\\0\end{pmatrix}, \quad
|5\rangle = \begin{pmatrix}1\\1\\1\end{pmatrix}
\]

\noindent\textbf{Transfer Matrix:} Entry $T_{ij}$ counts ways to transition from state $i$ to state $j$. The $6 \times 6$ matrix is:
\[
T = \begin{pmatrix}
1 & 1 & 1 & 0 & 0 & 0 \\
0 & 0 & 0 & 1 & 1 & 0 \\
0 & 0 & 0 & 1 & 0 & 1 \\
1 & 0 & 0 & 0 & 0 & 0 \\
1 & 0 & 0 & 0 & 0 & 0 \\
0 & 0 & 0 & 0 & 1 & 1
\end{pmatrix}
\]
The number of tilings equals $\text{tr}(T^{2n})$ — cycles returning to the initial state.

\noindent\textbf{Spectral Analysis:} The characteristic polynomial is
\begin{align}
\det(T - \lambda I) = \lambda^6 - 3\lambda^4 - \lambda^3 + 3\lambda + 1
\end{align}
The dominant eigenvalue $\lambda_1 \approx 1.8393$ determines the asymptotic growth. For large $n$:
\begin{align}
\text{tr}(T^{2n}) \sim c \cdot (1.8393)^{2n}
\end{align}
Entropy per unit area: $h = \frac{1}{6}\ln(\lambda_1^2) \approx 0.2028$.

\noindent\textbf{Physical Connection:} This entropy equals the free energy of the dimer model on a cylindrical lattice at zero temperature. The transfer matrix method extends to compute correlations and phase transitions. For planar regions, this connects to the Arctic circle phenomenon where random tilings exhibit frozen regions near boundaries.

\noindent\textbf{Gerrymandering Analogy:} The transfer matrix counts legal configurations given constraints — like counting valid districtings given population and contiguity rules. Both problems reduce to eigenvalue calculations on state-transition graphs.

\vspace{0.5em}
\noindent\textbf{References:}\\
{\footnotesize
Kenyon, R. (2009). \textit{Lectures on Dimers}. arXiv:0910.3129.\\
Mohar, B., Thomassen, C. (2001). \textit{Graphs on Surfaces}. Johns Hopkins Press.
}
\end{technical}