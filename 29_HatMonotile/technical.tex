\begin{technical}
{\Large\textbf{Two Standard Euler-Formula Arguments}}\\[0.3em]

\textbf{A) Average face/vertex degrees in any edge-to-edge tiling}\\[0.2em]
Let a locally finite, edge-to-edge tiling of the plane have:
\begin{itemize}[topsep=0pt,itemsep=2pt]
\item $F$ faces (tiles), $E$ edges, $V$ vertices in a large simply connected patch.
\item Face degrees $p_f$ (sides per tile), vertex degrees $q_v$ (tiles meeting at a vertex).
\end{itemize}
Define
\begin{align*}
\bar{p} &= \frac{1}{F}\sum_f p_f, & \bar{q} &= \frac{1}{V}\sum_v q_v.
\end{align*}
Double counting:
\begin{align*}
\sum_f p_f &= 2E - e_\partial, & \sum_v q_v &= 2E - e'_\partial,
\end{align*}
where $e_\partial, e'_\partial$ are boundary corrections. For a large patch, $e_\partial/E \to 0$. Hence
\begin{align*}
\frac{2E}{F} &\to \bar{p}, & \frac{2E}{V} &\to \bar{q}.
\end{align*}
Euler on a simply connected patch: $V - E + F = 1$. Divide by $E$ and pass to the limit:
\[
\frac{V}{E} - 1 + \frac{F}{E} \to 0.
\]
Using $V/E \to 2/\bar{q}$ and $F/E \to 2/\bar{p}$, we get the \textbf{Euler balance}
\[
\boxed{\frac{1}{\bar{p}} + \frac{1}{\bar{q}} = \frac{1}{2}}.
\]
Immediate consequences:
\begin{itemize}[topsep=0pt,itemsep=2pt]
\item Since $q_v \geq 3$, $\bar{q} \geq 3$, so $1/\bar{p} = 1/2 - 1/\bar{q} \leq 1/2 - 1/3 = 1/6$, hence $\boxed{\bar{p} \leq 6}$.
\item Dually, $p_f \geq 3$ gives $\boxed{\bar{q} \leq 6}$.
\end{itemize}

\textbf{Corollary.} In any edge-to-edge tiling by convex polygons, either every tile is a hexagon with $\bar{p} = 6$ and $\bar{q} = 3$, or there exists at least one tile with $\leq 5$ sides.
(If all tiles had $\geq 6$ sides with some $> 6$, then $\bar{p} > 6$, contradicting $\bar{p} \leq 6$.)

\textbf{B) Classification of regular tilings via Euler}\\[0.2em]
A regular (uniform) tiling $\{p,q\}$ has every face a regular $p$-gon, and $q$ faces meet at each vertex. Then $\bar{p} = p$ and $\bar{q} = q$, so
\[
\frac{1}{p} + \frac{1}{q} = \frac{1}{2}.
\]
With integers $p, q \geq 3$, the only solutions are
\[
(p,q) \in \{(3,6), (4,4), (6,3)\}.
\]
So the only regular Euclidean tilings are by equilateral triangles, squares, and regular hexagons.
\end{technical}
