\begin{historical}
The study of tiling has ancient roots, with early explorations by Albrecht Dürer and Johannes Kepler hinting at complex patterning, though primarily from observational or aesthetic perspectives. Aperiodic tiling, in the rigorous mathematical sense, began to take shape in the 1960s through the work of Hao Wang, who examined whether square tiles with edge patterns — now called Wang tiles — could tile the plane only non-periodically. Wang conjectured that if a set of tiles could tile the plane, it must be able to do so periodically.

This conjecture was soon refuted. In 1966, Wang’s student Robert Berger constructed the first known aperiodic tile set, though it required over 20,000 distinct tiles. Later refinements reduced the number significantly. In the 1970s, Roger Penrose advanced the field by introducing sets of two tiles that enforced aperiodicity using geometric constraints and local matching rules. These configurations, such as the kite and dart, became iconic examples of non-periodic order. In 1982, the discovery of quasicrystals by Dan Shechtman revealed that certain metallic alloys naturally exhibit aperiodic atomic structure, connecting tiling theory with physical materials.

Despite this progress, the search for a single connected shape — a monotile — that could tile the plane only aperiodically remained unresolved for decades. This so-called “Einstein problem” (from the German "ein Stein," meaning “one tile”) stood as a central open question in tiling theory.

In 2023, a breakthrough came from an unexpected collaboration. David Smith, a hobbyist with a long-standing interest in tiling, discovered a 13-sided polygon constructed from kites, which he called the “hat.” Working with Joseph Myers, Craig Kaplan, and Chaim Goodman-Strauss, Smith demonstrated that this tile could indeed tile the plane only aperiodically, relying solely on its geometry without any matching rules or markings. Their result resolved the Einstein problem in its original geometric form and marked a major milestone in the mathematical theory of tilings.
\end{historical}
