\begin{historical}
Tiling problems bridge pure mathematics with the decorative arts across millennia. Islamic artisans developed intricate geometric patterns in the Alhambra and other architectural masterworks, exploring symmetry groups centuries before their mathematical formalization. The Roman opus tessellatum, Byzantine mosaics, and Japanese tatami arrangements each integrated cultural rules into the way shapes fit together. These artistic traditions posed implicit mathematical questions: which patterns are possible, which are forbidden, and why?

Mathematical study of tilings began with natural questions. Kepler's 1619 investigation of hexagonal packing arose from observing snowflakes and honeycombs. His conjecture that hexagonal packing is optimal for circles remained unproven until Thomas Hales's computer-assisted proof in 1998. Dürer's 1525 treatise on measurement included systematic constructions of periodic tilings, blending Renaissance art with geometric precision.

The modern theory of aperiodic tiling emerged in the 1960s through Hao Wang's study of the domino problem: given a set of square tiles with colored edges, can they tile the plane if adjacent edges must match colors? Wang initially conjectured that any tileable set must tile periodically, linking the question to formal logic and computability.

This conjecture was soon refuted. In 1966, Wang’s student Robert Berger constructed the first known aperiodic tile set, though it required over 20,000 distinct tiles. Later refinements reduced the number significantly. In the 1970s, Roger Penrose advanced the field by introducing sets of two tiles that enforced aperiodicity using geometric constraints and local matching rules. These configurations, such as the kite and dart, became iconic examples of non-periodic order. In 1982, the discovery of quasicrystals by Dan Shechtman revealed that certain metallic alloys naturally exhibit aperiodic atomic structure, connecting tiling theory with physical materials.

Despite this progress, the search for a single connected shape — a monotile — that could tile the plane only aperiodically remained unresolved for decades. This so-called "ein Stein"  (“one tile”) stood as a central open question in tiling theory.

In 2023, a breakthrough came from an unexpected collaboration. David Smith, a hobbyist with a long-standing interest in tiling, discovered a 13-sided polygon constructed from kites, which he called the "hat." Working with Joseph Myers, Craig Kaplan, and Chaim Goodman-Strauss, Smith demonstrated that this tile could indeed tile the plane only aperiodically, relying solely on its geometry without any matching rules or markings. Their result resolved the Einstein problem in its original geometric form and marked a major milestone in the mathematical theory of tilings.
\end{historical}
