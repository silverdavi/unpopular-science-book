Mathematics analyzes geometric arrangements by converting visual problems into numerical equations. Take any convex polyhedron — a cube, a pyramid, or something more exotic — and count three things: vertices (corners), edges (where faces meet), and faces (flat surfaces). No matter how complex the shape, these three numbers satisfy: $V - E + F = 2$.

This is Euler's formula, discovered in 1750. A cube has 8 vertices, 12 edges, and 6 faces: $8 - 12 + 6 = 2$. A triangular pyramid has 4 vertices, 6 edges, and 4 faces: $4 - 6 + 4 = 2$. A soccer ball (truncated icosahedron) has 60 vertices, 90 edges, and 32 faces: $60 - 90 + 32 = 2$. Always two. This invariant connects topology to combinatorics — the formula holds as we stretch or deform the polyhedron, as long as we don't tear it.

The Platonic solids — those perfect forms where identical regular $p$-gons meet $q$ at every vertex — obey this constraint. Each vertex contributes $360°/q$ degrees from each face, and each $p$-gon has interior angles of $(p-2)180°/p$ degrees. For the solid to close up properly, we need $q × (p-2)180°/p < 360°$, which simplifies to $(p-2)(q-2) < 4$. Since $p$ and $q$ must be at least 3 (you need three edges for a polygon, three faces to make a corner), only five solutions exist: $(3,3)$, $(3,4)$, $(4,3)$, $(3,5)$, and $(5,3)$. These correspond to the tetrahedron, octahedron, cube, icosahedron, and dodecahedron. Euler's formula converts "what perfect forms exist?" into "which integers satisfy $(p-2)(q-2) < 4$?"

In this chapter, we explore another geometric problem that is historically one of the largest intersection points between mathematics and art. When arranging shapes to cover an infinite plane without gaps or overlaps, local constraints again determine global outcomes. Here too, the interplay of vertices, edges, and faces obeys mathematical laws, but now applied to infinite configurations rather than closed surfaces.

In a large circular patch of any edge-to-edge tiling, along the boundary, the formula needs correction terms, but as the patch grows, the interior dominates. In the limit, $V - E + F$ approaches zero (not two — the plane has different topology than a sphere). Dividing by the number of faces and taking limits, we get $1/\bar{p} + 1/\bar{q} = 1/2$, where $\bar{p}$ is the average sides per tile and $\bar{q}$ is the average tiles per vertex. This constraint explains why only three regular tilings exist: triangles ($p=3$, $q=6$), squares ($p=4$, $q=4$), and hexagons ($p=6$, $q=3$). It also proves that in any tiling, the average polygon has at most six sides — explaining why bees chose hexagons and why Islamic artists never tiled mosques with regular heptagons.

Tiling the plane refers to covering the infinite flat surface of Euclidean geometry with repeated, gapless copies of one or more shapes. The problem: given a shape, can it tile the plane? If so, does it do so uniquely, periodically, or in multiple distinct ways? In mathematics, tilings model how local adjacency rules propagate to infinite configurations. The arrangement must leave no gaps or overlaps and must cover the entire plane. These criteria make tiling an ideal setting to study how geometry interacts with combinatorics.

Classical tilings exhibit periodicity. That is, a finite patch of tiles can be shifted — translated — along certain vectors to cover the entire plane without change. This is the case for squares, equilateral triangles, and regular hexagons, all of which tile the plane in grid-like or honeycomb arrangements. The periodicity implies symmetry: the whole tiling looks the same from multiple viewpoints. Once a unit cell is known, the rest of the pattern follows. This repetition reflects the symmetry group of the tiling, which includes discrete translations and, often, rotations or reflections.

Between periodic and random tilings lies a third category: aperiodic tilings. These are constructed by deterministic rules yet never repeat under any translation. Every finite patch reappears infinitely often throughout the plane, but always in new contexts — never aligning with a translated copy of itself. This paradox of local recurrence without global periodicity became central to twentieth-century tiling theory.

The question of how many tiles are needed to achieve different tiling behaviors evolved rapidly. With an infinite set of distinct tiles, constructing aperiodic tilings is straightforward — each new tile can be unique, forcing non-repetition. Berger (1966) first showed that aperiodic tilings could be achieved with a finite set of 20,426 tiles. This number dropped rapidly: Robinson (1971) reduced it to 6 tiles, then Penrose (1974) achieved it with just 2.

Penrose tiles represented a milestone: two shapes that, when used together with matching rules, could tile the plane only aperiodically. They gave the first explicit example of enforced aperiodicity in Euclidean space using only two tiles. The Penrose tilings exhibit hierarchical organization, rotational symmetries, and local rules that guarantee non-repetition globally. They spurred new directions in mathematical logic, including the undecidability of the domino problem, which asks whether a given set of tiles can tile the plane.

This progression — from infinite sets to thousands to just two tiles — led to the "ein Stein" problem: could a single tile enforce aperiodicity? For decades, every attempt failed. In 2023, David Smith, a retired printer, discovered a 13-sided polygon that solved it. The "hat" tile forces aperiodic tiling through its shape alone — a genuine monotile.

The hat's 13 edges meet at specific angles that force a unique arrangement. Place one tile, and its neighbors must fit into the concave and convex indentations in exactly one way. This local constraint propagates: each new tile placement further restricts its surroundings. The accumulation of these forced choices prevents any translational symmetry from emerging at larger scales.

Any cluster of hat tiles you identify will appear again elsewhere — rotated, reflected, embedded in different surroundings, but recognizable. This property, called local isomorphism, means the tiling contains infinite copies of every finite pattern, yet no two regions ever match exactly under translation.

The hat tiling exhibits hierarchical structure: clusters of tiles form larger units that themselves cluster at increasing scales. This recursive organization — characteristic of aperiodic tilings — emerges from the local fitting constraints.

The hat connects to a larger phenomenon. In 1984, Dan Shechtman discovered quasicrystals — materials whose atoms arrange aperiodically yet produce sharp diffraction spots. These materials challenged crystallography's foundations: order without periodicity. The hat demonstrates this principle through pure geometry, achieving what Penrose tiles needed matching rules to enforce.

The discovery process itself is notable. David Smith, a retired printer and amateur tiling enthusiast, first noticed the tile’s behavior through hands-on experimentation. He shared it with researchers Craig Kaplan, Joseph Myers, and Chaim Goodman-Strauss, who formalized the result and constructed rigorous proofs. Their work combined visual intuition, algorithmic exploration, and mathematical formalism — demonstrating that advances in mathematics can begin from curiosity and empirical play.

\clearpage

\begin{commentary}[Commentary: Amateur Insight and Formal Machinery]
The hat tile was discovered not by formal derivation but by direct experimentation — David Smith observed its non-repeating behavior through hands-on manipulation. This empirical observation was later developed into a full mathematical result through collaboration with experts in geometry and substitution theory. The Spectre resulted from this joint effort, combining computational enumeration, combinatorial reduction, and formal inflation proofs. The case illustrates how mathematical insight can originate outside institutional settings and be sharpened into theorem by precise formal methods. Discovery here was stabilized through interaction between intuition and method.
\end{commentary}

\vspace{2em}

\begin{tcolorbox}[
  colback=gray!2,
  colframe=gray!60,
  boxrule=0.4pt,
  width=\textwidth,
  arc=1pt,
  left=8pt,
  right=8pt,
  top=6pt,
  bottom=6pt,
  shadow={0mm}{-0.5mm}{0mm}{gray!30}
]
\textbf{Why does every soccer ball always have exactly 12 pentagons?}

\vspace{0.5em}
A soccer ball pattern tiles a sphere with pentagons and hexagons, where exactly three faces meet at each vertex. Let $P$ = number of pentagons and $H$ = number of hexagons.

Each pentagon has 5 edges and each hexagon has 6 edges, but every edge is shared by exactly 2 faces:
$$E = \frac{5P + 6H}{2}$$

At each vertex, 3 faces meet. Each pentagon contributes 5 vertices and each hexagon contributes 6, but every vertex is triple-counted:
$$V = \frac{5P + 6H}{3}$$

The total number of faces is simply:
$$F = P + H$$

Applying Euler's formula for a sphere ($V - E + F = 2$):
$$\frac{5P + 6H}{3} - \frac{5P + 6H}{2} + P + H = 2$$

Multiply through by 6:
$$2(5P + 6H) - 3(5P + 6H) + 6P + 6H = 12$$
$$10P + 12H - 15P - 18H + 6P + 6H = 12$$
$$P = 12$$

Therefore, any sphere tiled with pentagons and hexagons (with 3 faces per vertex) must have exactly 12 pentagons, regardless of the number of hexagons. This applies to soccer balls, fullerene molecules, and geodesic domes alike.
\end{tcolorbox}

