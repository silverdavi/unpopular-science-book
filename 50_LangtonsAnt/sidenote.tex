\begin{SideNotePage}{
  \textbf{Top (Scales of Infection and Pathogenic Agents):}  
  Infectious agents span eight orders of magnitude—from macroscopic parasites like tapeworms to sub-viral prions. Each has its own transmission pathway, lifecycle, and interaction mode with the host. Top-right: a brain, target of prion neurodegeneration, underscores that even the smallest agents can collapse the most complex systems. \par

  \textbf{Bottom (Post-Hoc Theories of Consciousness):}  
  Proposed mechanisms for consciousness map along a complexity–integrability plane: waterfall-like chaos, chip design, nematode circuits, human brains, and societal behavior. But all are descriptive, not explanatory. No theory yet derives subjective experience from first principles — only correlates it post hoc to system properties and for each example we can find synthetic constructs as a counter-example. \par
}{50_LangtonsAnt/50_Consciousness.pdf}
\end{SideNotePage}

