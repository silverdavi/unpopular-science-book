Propofol takes seconds to erase consciousness, yet after 175 years of routine anesthesia, we still don't know how. Different drugs achieve the same oblivion through various mechanisms: some enhance inhibition, others block excitation, and ketamine paradoxically increases brain activity while eliminating awareness. The Meyer-Overton rule linking potency to lipid solubility seemed promising until clear exceptions contradicted it. The opposite extreme is also mysterious: in fatal familial insomnia, prion-destroyed thalamic nuclei trap victims in permanent wakefulness until death. One condition subtracts consciousness too easily; the other makes its absence impossible. We've conquered flipping consciousness on and off without understanding what the switch actually is, and we definitely don't understand what consciousness really is.