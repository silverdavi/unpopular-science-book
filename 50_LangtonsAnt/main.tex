Anesthesia subtracts awareness without leaving residue. A standard induction — propofol infusion with an opioid and a volatile like sevoflurane — produces unresponsiveness within seconds, suppresses recall, and returns the patient intact minutes later. Mechanistically, this tidy clinical arc hides a contested landscape.

Early reduction to lipid solubility (Meyer--Overton) failed: potency correlates with partitioning into membranes, but outliers and non-immobilizers falsify sufficiency. Molecular specificity fragments the story by agent class. Isoflurane and propofol enhance inhibitory transmission by prolonging GABA\textsubscript{A} receptor openings and shifting input--output gain. Nitrous oxide and ketamine antagonize NMDA receptors; multiple volatiles also modulate two-pore K\textsuperscript{+} (K2P) and HCN1 channels, hyperpolarizing networks and damping resonance. None of these mechanisms alone explains loss of report across agents and species.

Network hypotheses move up a level. Thalamic “switch-off” models propose that sensory relay and intralaminar nuclei disengage cortical broadcasting. Alternatives hold that long-range cortico-cortical integration degrades: effective connectivity fragments, ignition-like reverberation collapses, and fronto-parietal synchrony decouples. Empirically, anesthetic depth tracks changes in spectral power, complexity, and coherence, but counterexamples persist. Ketamine increases overall cortical activity and high-frequency power, yet abolishes consciousness; dexmedetomidine reduces thalamic throughput yet permits vivid dreams.

The converse failure clarifies the boundary. In fatal familial insomnia (FFI), a PRNP mutation yields misfolded prion protein that selectively destroys thalamic nuclei critical for sleep. The San Giovanni pedigree maps a characteristic descent: difficulty initiating sleep, then inexorable insomnia, autonomic derangement, cognitive disintegration, and death within months. Here, consciousness persists without rest — not heightened awareness, but the failure of the system to traverse into sleep. Anesthesia removes experience; FFI prevents the transition away from it. Both reveal that conscious state transitions depend on precise, fragile routing through thalamocortical dynamics.

Measuring consciousness remains harder than turning it off. Clinical scales rely on responsiveness; neurophysiology adds proxies: cross-regional EEG coherence, perturbational complexity from TMS-evoked responses, and theoretical constructs like Integrated Information Theory’s $\Phi$. Each stumbles. Some unresponsive patients process speech; high $\Phi$ can be assigned to systems with no plausible subjectivity; EEG signatures of wakefulness can appear under amnestic sedation. Competing theories — Global Workspace, Integrated Information, Recurrent Processing — disagree on what makes a state conscious, and experiments often adjudicate proxies rather than experience itself.

The working picture is pragmatic: multiple molecular routes converge on a few network-level motifs — reduced ignition, impaired integration, altered thalamocortical gating — sufficient to block access to a reportable workspace. That picture explains much of practice and little of essence. We can flip the switch; the circuit diagram remains incomplete.

