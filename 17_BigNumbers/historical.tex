
\begin{historical}
    The challenge of expressing vast quantities appears across ancient texts. The Hebrew Bible uses \begin{hebrew}"רִבֵּי רְבָבוֹת"\end{hebrew} (ribei revavot) — myriads of myriads — to denote numbers beyond ordinary counting, as in Daniel 7:10 describing the heavenly host: "thousand thousands ministered unto him, and ten thousand times ten thousand stood before him." This poetic multiplication hinted at systematic ways to build larger numbers.

Archimedes formalized this intuition in \textit{The Sand Reckoner} (\begin{greek}Ψαμμίτης\end{greek}, Psammites), written around 250 BCE as a letter to King Gelon II of Syracuse. The work addressed Aristarchus's heliocentric model, which implied a universe vastly larger than previously imagined. Archimedes asked: could one count the grains of sand needed to fill such a cosmos? 

Greek numerals stopped at a myriad (10,000). Archimedes extended them by defining "orders" — the first order contained numbers up to a myriad myriads ($10^8$), the second order began at $10^8$ and continued to $(10^8)^2$, and so forth. Using this system, he estimated the universe could hold at most $10^{63}$ sand grains. The calculation was secondary to the method: showing that any finite quantity, however vast, could be systematically expressed and manipulated. This marked the birth of scientific notation and the separation of numbers from physical counting.

Edward Kasner faced a pedagogical problem in 1938: explaining infinity to his nine-year-old nephew, Milton Sirotta. To illustrate the difference between very large finite numbers and true infinity, Kasner asked Milton to invent a name for $10^{100}$. Milton suggested "googol," and they defined "googolplex" as $10^{\text{googol}}$. These terms, introduced in Kasner's book "Mathematics and the Imagination," demonstrated how mathematical notation could rapidly outpace physical reality.

Modern developments began with Wilhelm Ackermann's 1928 function that grows faster than any primitive recursive function. This revealed that computational growth rates form a hierarchy — some functions outpace others so dramatically that conventional notation fails.

Harvey Friedman transformed large numbers from recreational mathematics into serious research in the 1990s. His TREE sequence, derived from Kruskal's tree theorem, produced numbers dwarfing all previous constructions. TREE(3) is finite but so large it cannot be expressed using conventional operations iterated any reasonable number of times. The proof requires axioms beyond Peano arithmetic, connecting large numbers to deep results in proof theory and the limits of formal systems.
\end{historical}