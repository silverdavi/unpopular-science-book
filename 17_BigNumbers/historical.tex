Archimedes, in \textit{The Sand Reckoner}, estimated the number of sand grains needed to fill the universe at around $10^{63}$. He developed a notation system for expressing such magnitudes, extending Greek numerals beyond their conventional limits. This established that finite numbers, however large, could be systematically described and manipulated.

Edward Kasner faced a pedagogical problem in 1938: explaining infinity to his nine-year-old nephew, Milton Sirotta. To illustrate the difference between very large finite numbers and true infinity, Kasner asked Milton to invent a name for $10^{100}$. Milton suggested "googol," and they defined "googolplex" as $10^{\text{googol}}$. These terms, introduced in Kasner's book "Mathematics and the Imagination," demonstrated how mathematical notation could rapidly outpace physical reality.

Modern developments began with Wilhelm Ackermann's 1928 function that grows faster than any primitive recursive function. This revealed that computational growth rates form a hierarchy — some functions outpace others so dramatically that conventional notation fails.

Harvey Friedman transformed large numbers from recreational mathematics into serious research in the 1990s. His TREE sequence, derived from Kruskal's tree theorem, produced numbers dwarfing all previous constructions. TREE(3) is finite but so large it cannot be expressed using conventional operations iterated any reasonable number of times. The proof requires axioms beyond Peano arithmetic, connecting large numbers to deep results in proof theory and the limits of formal systems.