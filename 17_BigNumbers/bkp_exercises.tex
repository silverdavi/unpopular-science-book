\fullpageexercises{%
\textbf{Basic Exercises: Large Numbers and Infinity} \\[1em]
These exercises explore the mathematical concepts behind extremely large numbers, from computable functions to uncomputable growth rates and infinite cardinal hierarchies.

\vspace{1em}

\textbf{1. Computing Small Busy Beaver Values} \\[0.5em]
Design a 2-state Turing machine that achieves BB(2) = 4 steps before halting. Your machine should use states A and B, start in state A, and operate on a tape initially filled with zeros. \\
\emph{Task:} Trace through the execution step by step, showing the tape contents and machine state at each step.

\vspace{1em}

\textbf{2. Understanding Uncomputability} \\[0.5em]
Explain why the Busy Beaver function is uncomputable by reduction to the halting problem. \\
\emph{Question:} If we could compute BB(n) for any n, how would this allow us to solve the halting problem? Why does this lead to a contradiction?

\vspace{1em}

\textbf{3. Comparing Growth Rates} \\[0.5em]
Order these functions from slowest to fastest growing: $2^n$, $n!$, $A(n,n)$ (Ackermann), $2^{2^n}$, $BB(n)$ (Busy Beaver). \\
\emph{Discussion:} For each pair, explain which grows faster and approximately when the faster function overtakes the slower one.

\vspace{1em}

\textbf{4. Self-Reference in Rayo's Function} \\[0.5em]
Rayo's function is defined as "the smallest number bigger than any number definable in first-order set theory using ≤n symbols." \\
\emph{Question:} This definition appears self-referential—it uses the concept of "definable" within its own definition. How is this paradox resolved? Compare to Russell's paradox.

\vspace{1em}

\textbf{5. Cardinal Arithmetic} \\[0.5em]
In infinite cardinal arithmetic, we have unusual equations like $\aleph_0 + \aleph_0 = \aleph_0$ and $\aleph_0 \times \aleph_0 = \aleph_0$. \\
\emph{Tasks:} (a) Explain why $\aleph_0 + 1 = \aleph_0$ using Hilbert's Hotel. (b) Show that the set of rational numbers has cardinality $\aleph_0$. (c) Why is $2^{\aleph_0} > \aleph_0$?

\vspace{1em}

\textbf{6. Understanding the Cumulative Hierarchy} \\[0.5em]
Large cardinals exist within the cumulative hierarchy of sets: $V_0 = \emptyset$, $V_{\alpha+1} = \mathcal{P}(V_\alpha)$, $V_\lambda = \bigcup_{\alpha < \lambda} V_\alpha$ for limit $\lambda$. \\
\emph{Questions:} (a) What is $|V_\omega|$? (b) An inaccessible cardinal $\kappa$ satisfies $V_\kappa = H_\kappa$. What does this mean? (c) Why can't inaccessible cardinals be proven to exist in ZFC?
}
