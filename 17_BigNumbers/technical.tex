\begin{technical}
{\Large\textbf{Hierarchy of Growth Rates}}\\[0.7em]

\textbf{Level 1: Elementary Functions}
\begin{align}
f(n) &= n + c \text{ (linear)}\\
f(n) &= n^k \text{ (polynomial)}\\
f(n) &= k^n \text{ (exponential)}\\
f(n) &= n! \approx \sqrt{2\pi n}\left(\frac{n}{e}\right)^n
\end{align}

\textbf{Level 2: Iterated Exponentials}\\
Tetration: ${}^ka = \underbrace{a^{a^{\cdot^{\cdot^{\cdot^a}}}}}_{k \text{ times}}$\\
Growth: ${}^k2$ has $\sim 2^{k-2}$ digits.

\textbf{Level 3: Ackermann Function}
\begin{align}
A(0,n) &= n+1\\
A(m+1,0) &= A(m,1)\\
A(m+1,n+1) &= A(m, A(m+1,n))
\end{align}
Growth: $A(1,n) \approx 2n$, $A(2,n) \approx 2^n$, $A(3,n) \approx {}^{n+3}2$, $A(4,n)$ exceeds tetration.

\textbf{Level 4: Knuth Arrows}
\begin{align}
a \uparrow^0 b &= ab\\
a \uparrow^{k+1} b &= a \uparrow^k (a \uparrow^{k+1} (b-1))
\end{align}
$3 \uparrow 3 = 27$, $3 \uparrow\uparrow 3 = 7{,}625{,}597{,}484{,}987$\\
$3 \uparrow\uparrow\uparrow 3 = 3 \uparrow\uparrow 7{,}625{,}597{,}484{,}987$

\textbf{Level 5: Fast-Growing Hierarchy}\\
Indexed by ordinals:
\begin{align}
f_0(n) &= n+1\\
f_{\alpha+1}(n) &= f_\alpha^n(n)\\
f_\omega(n) &= f_n(n)\\
f_{\omega^2}(n) &= f_{\omega \cdot n}(n)\\
f_{\varepsilon_0}(n) &\text{ dominates finite } \omega \text{ towers}
\end{align}

\textbf{Level 6: TREE Function}\\
TREE$(n)$ = max sequence of $n$-labeled trees with no embeddings.\\
TREE$(1) = 1$, TREE$(2) = 3$\\
TREE$(3) > f_{\theta(\Omega^\omega)}(n)$ for any reasonable $n$.

\textbf{Level 7: Busy Beaver}\\
BB$(n)$ = max steps of halting $n$-state Turing machine.\\
BB$(4) = 13$, BB$(5) > 47{,}176{,}870$\\
BB$(6) > 3.5 \times 10^{18267}$\\
Eventually dominates all computable functions.

\textbf{Level 8: Rayo Function}\\
Rayo$(n)$ = smallest number not definable in set theory with $\leq n$ symbols.\\
Dominates any $n$-symbol definable function by diagonalization.

\textbf{Growth Comparison}\\
For large $n$: polynomial $\ll$ exponential $\ll$ Ackermann $\ll$ arrows $\ll f_{\varepsilon_0} \ll$ TREE $\ll$ BB $\ll$ Rayo

Each level uses fundamentally stronger recursion principles.

\vspace{0.5em}
\textbf{References:}\\
Löb \& Wainer, \textit{Arch. Math. Logik} \textbf{13}, 1 (1970).\\
Friedman, \textit{J. Comb. Theory A} \textbf{95}, 102 (2006).
\end{technical}