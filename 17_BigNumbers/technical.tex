\begin{technical}
{\Large\textbf{Large Numbers: Mathematical Formulation}}\\[0.7em]

\textbf{Busy Beaver Function}\\[0.5em]
The Busy Beaver function $\text{BB}(n)$ is defined as:
$$
\text{BB}(n) = \max\{s(M) : M \text{ is an $n$-state Turing machine that halts}\},
$$
where $s(M)$ is the number of steps taken by machine $M$ before halting. Known values include:
\begin{align}
\text{BB}(1) &= 1\\
\text{BB}(2) &= 4\\
\text{BB}(3) &= 6\\
\text{BB}(4) &= 13\\
\text{BB}(5) &> 47{,}176{,}870
\end{align}

The function is uncomputable: there is no algorithm that can compute $\text{BB}(n)$ for arbitrary $n$. This follows from the unsolvability of the halting problem.

\textbf{Ackermann Function}\\[0.5em]
The Ackermann function is defined recursively as:
\begin{align}
A(0, n) &= n + 1\\
A(m+1, 0) &= A(m, 1)\\
A(m+1, n+1) &= A(m, A(m+1, n))
\end{align}
It grows faster than any primitive recursive function but remains total recursive. Key values:
\begin{align}
A(3, 4) &= 2^{2^{2^2}} - 3 = 2^{16} - 3 = 65{,}533\\
A(4, 2) &= 2^{2^{2^{2^2}}} - 3 \approx 2.0 \times 10^{19{,}728}
\end{align}

\textbf{Rayo Function}\\[0.5em]
Rayo's function is defined as:
$$
\text{Rayo}(n) = \min\{k \in \mathbb{N} : k \text{ cannot be defined in first-order set theory using } \leq n \text{ symbols}\}.
$$
This definition uses diagonal argument techniques similar to those in Russell's paradox. The function grows much faster than any function definable in first-order set theory.

\textbf{TREE Function}\\[0.5em]
The TREE function arises from Kruskal's tree theorem. TREE$(n)$ is the maximum length of a sequence of labeled trees with labels from $\{1, 2, \ldots, n\}$ such that no tree in the sequence is inf-embeddable into a later tree.

Known bounds on TREE$(3)$ are astronomical:
$$
\text{TREE}(3) > \underbrace{A(A(\cdots A}_{A(9^{9^9}) \text{ iterations}}(9^{9^9})\cdots)).
$$
The number is so large that it exceeds the Fast-Growing Hierarchy evaluated at enormous ordinals.

\textbf{Fast-Growing Hierarchy}\\[0.5em]
The fast-growing hierarchy $\{f_\alpha\}_{\alpha < \varepsilon_0}$ is defined using ordinal indexing:
\begin{align}
f_0(n) &= n + 1\\
f_{\alpha+1}(n) &= f_\alpha^n(n) \text{ (iterate $f_\alpha$ $n$ times)}\\
f_\lambda(n) &= f_{\lambda[n]}(n) \text{ for limit ordinal $\lambda$}
\end{align}
where $\lambda[n]$ is the $n$-th element in the fundamental sequence for $\lambda$.

\textbf{Large Cardinals}\\[0.5em]
Large cardinals form a hierarchy of infinite numbers with increasing consistency strength:

\textbf{Inaccessible Cardinal:} A cardinal $\kappa$ such that:
\begin{itemize}
\item $\kappa$ is uncountable
\item $\kappa$ is a strong limit: $2^\lambda < \kappa$ for all $\lambda < \kappa$  
\item $\kappa$ is regular: $\text{cf}(\kappa) = \kappa$
\end{itemize}

\textbf{Mahlo Cardinal:} An inaccessible cardinal $\kappa$ such that the set of inaccessible cardinals less than $\kappa$ is stationary in $\kappa$.

\textbf{Strongly Compact Cardinal:} A cardinal $\kappa$ such that the logic $L_{\kappa,\kappa}$ satisfies a strong compactness property: every set of sentences of size $\kappa$ has a model if every subset of size less than $\kappa$ has a model.

\textbf{Supercompact Cardinal:} A cardinal $\kappa$ such that for every $\lambda \geq \kappa$, there exists an elementary embedding $j: V \rightarrow M$ with critical point $\kappa$, $j(\kappa) > \lambda$, and $M^\lambda \subseteq M$.

The hierarchy continues indefinitely with huge cardinals, n-huge cardinals, superhuge cardinals, and beyond.

\textbf{Consistency Strength Ordering}\\[0.5em]
These large cardinal concepts are ordered by consistency strength. If $\phi$ and $\psi$ are large cardinal properties, then $\phi$ has greater consistency strength than $\psi$ if:
$$
\text{ZFC} + \phi \vdash \text{Con}(\text{ZFC} + \psi).
$$

\vspace{0.5em}
\textbf{References:}\\
Radó, T. (1962). On non-computable functions. \textit{Bell System Technical Journal}, 41(3):877-884.\\
Ackermann, W. (1928). Zum Hilbertschen Aufbau der reellen Zahlen. \textit{Math. Ann.}, 99(1):118-133.\\
Rayo, A. (2007). On the length of proofs. \textit{Logique et Analyse}, 50(200):565-586.\\
Friedman, H. (2006). Long finite sequences. \textit{Journal of Combinatorial Theory}, Series A, 95(1):102-144.\\
Jech, T. (2003). \textit{Set Theory: The Third Millennium Edition}. Springer-Verlag.
\end{technical}
