\begin{SideNotePage}{
  \textbf{Top (Voting Methods):} \par Ranked-choice voting is a system in which voters express their preferences by submitting complete rankings of all candidates, and the system aggregates them into a ranked list or a single winner. Different aggregation methods (Borda count, IRV, Plurality, Condorcet) can produce distinct winners from identical voter rankings, demonstrating the inherent ambiguity in collective decision-making. The same preference profile can yield different outcomes depending on which features of the rankings are emphasized by the chosen method. This indeterminacy reveals that there is no canonical way to translate individual preferences into collective choices.


  \textbf{Bottom (Arrow's Theorem):} \par Arrow's impossibility theorem proves that no ranked-choice voting method can satisfy all four fairness criteria simultaneously when there are at least three alternatives and two voters: 1) no dictatorship (no single voter controls all outcomes), 2) Pareto unanimity (universal agreement is respected), 3) independence of irrelevant alternatives (pairwise rankings unaffected by third options), 4) completeness and transitivity (coherent group rankings). Every democratic aggregation method must compromise at least one criterion, making trade-offs unavoidable in social choice. 
}{09_ArrowTheoremTopology/09_ Real Democracy Has Never Been Tried.pdf}
\end{SideNotePage}