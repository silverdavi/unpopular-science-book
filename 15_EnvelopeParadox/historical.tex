\begin{historical}
The envelope paradox traces its origins to Belgian mathematician Maurice Kraitchik, who proposed a related puzzle in his 1943 book \textit{Recreational Mathematics}. Kraitchik's version involved two men comparing the value of their neckties, with the winner giving his necktie to the loser as consolation. He also discussed a variant where the men compared the contents of their purses, assuming each contained between 1 and some large number of pennies with equal probability.

The puzzle gained mathematical attention when John Edensor Littlewood mentioned it in his 1953 book on elementary mathematics, crediting the idea to physicist Erwin Schrödinger. Littlewood's formulation involved cards with numbers written on both sides — a player sees one side of a random card and must decide whether to flip it over. His version made explicit the role of improper prior distributions in generating the paradox.

Martin Gardner brought the puzzle to widespread attention in his 1982 book \textit{Aha! Gotcha}, presenting it as a wallet game between two equally wealthy individuals. Gardner's formulation captured the essential difficulty: each person could construct identical arguments for why the game favored them, yet by symmetry, the game had to be fair. Remarkably, Gardner confessed that while he could analyze the problem correctly, he struggled to pinpoint exactly what was wrong with the switching argument.

The modern envelope paradox resurfaced as probability theory matured into a rigorous mathematical discipline. The 20th century development of measure theory, decision theory, and Bayesian analysis provided new tools for understanding why such problems arose and how they might be resolved. 
\end{historical}
