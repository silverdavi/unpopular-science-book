\begin{technical}
{\Large\textbf{Resolving the Envelope Paradox}}\\[0.3em]

The paradox emerges from treating the observed amount $Y$ as equally likely to be the smaller or larger of two amounts related by a $1:2$ ratio. The naive expectation assumes: $
\mathbb{E}[\text{switch}] = 0.5 \cdot 2Y + 0.5 \cdot Y/2 = 5Y/4,$ suggesting a gain from switching. But this treats $Y$ as both $X$ and $2X$ in different terms — a misapplication of conditional expectation.

\textbf{Correct Conditioning}\\[0.2em]
Let $X$ be drawn from prior $f(x)$. The envelope pair $(X, 2X)$ is constructed, and one is presented at random. Let $Y$ denote the observed amount:
\begin{align}
Y = X &\Rightarrow \text{other envelope has } 2Y,\\
Y = 2X &\Rightarrow \text{other envelope has } \frac{Y}{2}.\\[0.3em]
\mathbb{P}(X = Y) &\propto f(Y), \\
\mathbb{P}(X = Y/2) &\propto f(Y/2).
\end{align}
The expected value of switching given $Y$ is:
\[
\mathbb{E}[\text{sw} \mid Y] 
= \frac{2Y \cdot f(Y) + \tfrac{1}{2}Y \cdot f(Y/2)}{f(Y) + f(Y/2)}.
\]
This depends on the shape of $f$. When $f$ decays rapidly, $f(Y/2) \gg f(Y)$ for large $Y$, implying $Y$ is likely the larger value and switching is unfavorable.

\textbf{Example: Pareto Prior}\\[0.2em]
For $X \sim \text{Pareto}(\alpha)$ with $f(x) = \alpha x^{-(\alpha + 1)}$ on $[1, \infty)$:
\[
f(Y/2) = \alpha \cdot 2^{\alpha + 1} \cdot Y^{-(\alpha + 1)} = 2^{\alpha + 1} f(Y).
\]
Thus:
\[
\mathbb{E}[\text{switch} \mid Y] = \frac{2Y + \tfrac{1}{2}Y \cdot 2^{\alpha + 1}}{1 + 2^{\alpha + 1}} = Y \cdot \frac{2 + 2^{\alpha}}{1 + 2^{\alpha + 1}}.
\]
This may exceed or fall below $Y$ depending on $\alpha$.

\textbf{Improper Priors}\\[0.2em]
For log-uniform $f(x) \propto 1/x$ on $[1, \infty)$ (improper since $\int_1^{\infty} \frac{1}{x} dx = \infty$):
\begin{align}
f(Y) &= \frac{1}{Y}, \quad f(Y/2) = \frac{2}{Y},\\
\mathbb{E}[\text{switch} \mid Y] &= \frac{2Y \cdot \frac{1}{Y} + \tfrac{1}{2}Y \cdot \frac{2}{Y}}{\frac{1}{Y} + \frac{2}{Y}} = Y.
\end{align}
This suggests switching is neutral, but the result is meaningless: the improper prior makes the marginal distribution undefined. 

\textbf{Finite Uniform Model}\\[0.2em]
Let $x \in \{2^0, 2^1, \dots, 2^{N-1}\}$ be uniform. For observed amount $A = 2^m$:
\begin{itemize}[topsep=0pt,itemsep=2pt]
\item Interior ($1 \le m \le N - 2$): $\mathbb{E}[\Delta \mid A] = \tfrac{1}{2}(2^m) + \tfrac{1}{2}(-2^{m-1}) = 2^{m-2}$
\item Boundaries: $\mathbb{E}[\Delta \mid A = 2^0] = +1$, $\mathbb{E}[\Delta \mid A = 2^{N-1}] = -2^{N-2}$
\end{itemize}
Global expectation:
\begin{align}
\mathbb{E}[\Delta] &= \frac{1}{N} \left( \sum_{m=1}^{N-2} 2^{m-2} + 1 - 2^{N-2} \right) \\
&= \frac{1}{N} \left( -2^{N-3} + \tfrac{1}{2} \right) < 0,
\end{align}
As $N \to \infty$, this approaches zero from below, confirming no long-run switching advantage.

\textbf{References:}\\
Nalebuff, B. (1989). The Other Person's Envelope Is Always Greener. \textit{J. Econ. Persp.}, \textbf{3}(1), 171--181.
\end{technical}
