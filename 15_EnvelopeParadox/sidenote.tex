\begin{SideNotePage}{
  \textbf{Top (Basic Setup):} \par The left scenario shows concrete values: one envelope contains \$100, with potential alternatives of \$50 or \$200. The right scenario generalizes this with variables: envelope $X$ could correspond to either $x/2$ or $2x$ in the alternative envelope. This illustrates how the same logical structure applies regardless of specific amounts.

  \vspace{1em}
  \textbf{Second (Bounded Finite):} \par Eight envelopes containing $1, 2, 4, 8, 16, 32, 64, 512$ represent a bounded geometric sequence. In this finite setup, edge effects matter: the smallest value $(1)$ can only be paired with $(2)$, while the largest $(512)$ can only be paired with $(256)$. These boundary constraints prevent the paradox from arising.

  \vspace{1em}
  \textbf{Third (Semi-Infinite):} \par The sequence $1, 2, 4, \ldots, 2^n, \ldots$ extends infinitely in one direction. This introduces asymmetry: there's a smallest possible value but no largest. The lower bound creates structure that affects the expectation calculation.

  \vspace{1em}
  \textbf{Fourth (Bi-Infinite):} \par The sequence $\ldots, 1/2, 1, 2, 4, \ldots, 2^n, \ldots$ extends infinitely in both directions. This symmetric case is where the classical paradox emerges most clearly, as there are no boundary effects to break the apparent symmetry.

  \vspace{1em}
  \textbf{Bottom (Probability Distributions):} \par Three different priors over envelope values: (1) Uniform distribution over $\{1, 2, \ldots, 100\}$ assigns equal $1/100$ probability to each value. (2) Exponentially decaying distribution over positive integers gives decreasing probability as values increase. (3) Gaussian distribution over all real numbers.
}{15_EnvelopeParadox/15_ The Better Envelope.pdf}
\end{SideNotePage}