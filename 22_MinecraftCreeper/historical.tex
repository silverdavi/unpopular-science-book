\begin{historical}
Minecraft began as the independent project of Markus "Notch" Persson, a Swedish programmer inspired by sandbox-building games like Infiniminer and Dwarf Fortress. He began development on May 10, 2009, using Java with the Lightweight Java Game Library (LWJGL), releasing the first public version just six days later on May 16. Early versions focused on creative construction in a procedurally generated block world, appealing to players' innate curiosity and design instincts. Persson incorporated player feedback rapidly, adding survival mechanics, crafting, hostile mobs, and multiplayer support. Development was open and iterative, creating a strong early community on forums like TIGSource.

In 2010, Persson founded Mojang to support ongoing development. Jens Bergensten joined and later took over lead development. The game officially launched on November 18, 2011, at MineCon in Las Vegas, having already sold millions of copies in beta. Mojang maintained a low-friction sales model: a single upfront purchase, no DRM, and support across multiple operating systems. This simplicity contributed to rapid global adoption.

In 2014, Microsoft acquired Mojang and Minecraft for \$2.5 billion. At that point, Minecraft had sold over 54 million copies. Since then, the game has expanded to nearly every platform, including consoles, mobile, and VR. As of April 2025, Minecraft has over 300 million copies sold and more than 140 million monthly active users, making it the best-selling video game of all time. It has been used in classrooms, cited in academic studies on spatial reasoning and collaboration, and remains a major force in online content creation — especially on YouTube, where Minecraft videos have accumulated trillions of views.

Minecraft’s success stems from a blend of simplicity and depth. It offers intuitive core mechanics (block placement, mining, crafting) with nearly unlimited creative and systemic potential. Procedural generation ensures novelty, while redstone logic introduces programmable mechanics akin to electrical engineering. The game fosters personal expression, exploration, and emergent storytelling. Its low system requirements and modding support further extended its reach and longevity. Minecraft’s development history is a case study in iterative design, community engagement, and the creative payoff of systems-first thinking.
\end{historical}