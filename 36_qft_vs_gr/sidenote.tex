\begin{SideNotePage}{

    \textbf{Top (Double Slit):}  
    In general relativity, each possible photon path carries mass-energy and should bend spacetime accordingly. In quantum field theory, superpositions of paths interfere — but how does spacetime metric respond to a superposed trajectory? \par
  
    \textbf{Second (Photoelectric Effect):}  
    Near a gravitating body like Earth, GR predicts redshift of both incoming photons and bound electronic orbitals. Does the threshold for photoelectric emission shift? GR lacks quantum orbitals; QFT lacks spacetime curvature. \par
  
    \textbf{Third (Black Hole Information):}  
    GR predicts thermal Hawking radiation with no memory of what fell in. But QFT requires unitary evolution: information, such as spin, must be preserved. \par
  
    \textbf{Fourth (Unruh Effect):}  
    In GR, an observer in free fall detects no force and is locally equivalent to inertial motion. But in QFT, an accelerating observer perceives a thermal particle bath. Do both observers agree on the particle content or not? \par
  
    \textbf{Fifth (Vacuum Energy):}  
    QFT assigns enormous energy density to the vacuum via zero-point fluctuations. GR says energy curves spacetime. Does the vacuum bend space violently? \par
  
    \textbf{Bottom (Definition of Space):}  
    In GR, space is a manifold with tensor fields assigning numbers to each point. In QFT, space is a passive backdrop where operators act to extract observables. GR side of the equation is numbers, QFT side of the equation is operators. \par
}{36_qft_vs_gr/36_ When Theories Collide.pdf}
\end{SideNotePage}
