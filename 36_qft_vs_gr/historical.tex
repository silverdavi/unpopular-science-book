

% Historical Context
\begin{historical}
Quantum mechanics, which underpins quantum field theory (QFT), took shape in the 1920s with the pioneering work of Planck, Heisenberg, Schrödinger, and Dirac. Around the same period, Einstein’s general relativity (GR) from 1915 was being tested and further confirmed through observations such as the bending of starlight during eclipses. 

Both theories revolutionized physics: GR reinterpreted gravity as the curvature of spacetime, while QFT unified quantum principles with special relativity to describe forces (electromagnetism, weak, and strong) via quantized fields. Attempts to merge GR and QFT began in the mid-20th century, with physicists like Feynman, Pauli, and later Weinberg exploring pathways to quantize gravity. Yet, unlike the other forces, gravity resisted such integration. The divergences encountered in high-energy regimes, plus fundamental contradictions in how time and space are treated, revealed an inherent incompatibility. Despite decades of effort — including approaches like supergravity and string theory — no complete, experimentally confirmed quantum theory of gravity has emerged. 
\end{historical}
