\begin{historical}
The concept of vacuum stability emerged from parallel developments in cosmology and particle physics. In 1980, Alan Guth proposed cosmic inflation to solve the horizon and flatness problems. His model required a scalar field temporarily trapped in a false vacuum state, driving exponential expansion before decaying to the true vacuum. This established that metastable vacuum states could have observable consequences.

Sidney Coleman and Frank De Luccia calculated the decay rate of false vacua in 1980, showing that quantum tunneling creates expanding bubbles of true vacuum. Their formalism demonstrated that gravitational effects could either enhance or suppress transitions, depending on the energy difference between vacua. The calculation revealed that vacuum decay proceeds through nucleation of critical bubbles whose walls accelerate outward at the speed of light.

The discovery of the Higgs boson at CERN in 2012 transformed vacuum stability from theoretical speculation to measurable physics. The Higgs mass of 125 GeV, combined with precision measurements of the top quark mass at 173 GeV, placed the Standard Model near the boundary between stable and metastable regimes. Giuseppe Degrassi and collaborators showed in 2012 that these values imply the Higgs self-coupling likely becomes negative at energies around $10^{10}$ GeV, creating a deeper minimum in the potential.

These calculations depend critically on the running of coupling constants with energy scale. The renormalization group equations track how the Higgs quartic coupling $\lambda$ evolves from low to high energies. At the measured Higgs and top masses, $\lambda$ decreases with increasing energy and may cross zero around $10^{10}$ GeV. Beyond this point, the effective potential develops a new minimum at large field values where the universe would have different physical laws.
\end{historical}