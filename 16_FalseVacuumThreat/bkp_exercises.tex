\fullpageexercises{%
\textbf{Basic Exercises: False Vacuum Decay} \\[1em]
Understanding vacuum stability requires combining concepts from quantum field theory, statistical mechanics, and cosmology. These exercises explore the basic physics behind metastable vacuum states and their potential decay.

\vspace{1em}

\textbf{1. Energy Landscapes and Metastability} \\[0.5em]
Draw a potential energy curve with two minima: a shallow local minimum at $\phi = v$ and a deeper global minimum at $\phi = v'$. Label the false vacuum, true vacuum, and energy barrier. \\
\emph{Question:} If a classical particle starts at rest in the false vacuum, what energy must it acquire to reach the true vacuum?

\vspace{1em}

\textbf{2. Quantum Tunneling Rates} \\[0.5em]
The tunneling rate through a barrier scales as $\Gamma \propto e^{-S/\hbar}$, where $S$ is the action. For a simple quadratic barrier of height $E_0$ and width $L$, the action is approximately $S \approx \sqrt{2mE_0} \cdot L$. \\
\emph{Task:} If $E_0 = 1$ GeV, $L = 10^{-15}$ m, and $m = 1$ GeV, calculate the exponential suppression factor $e^{-S/\hbar}$.

\vspace{1em}

\textbf{3. Bubble Expansion} \\[0.5em]
A vacuum decay bubble expands at nearly the speed of light. Consider a bubble nucleated at $t = 0$ with negligible initial size. \\
\emph{Questions:} (a) What is the bubble radius after time $t$? (b) How long would it take for such a bubble to engulf the entire observable universe (radius $\sim 10^{26}$ m)?

\vspace{1em}

\textbf{4. Cosmic Ray Energies} \\[0.5em]
The highest energy cosmic ray ever detected had energy $\sim 3 \times 10^{20}$ eV. The estimated threshold for vacuum decay nucleation is $\sim 10^{28}$ eV. \\
\emph{Task:} Calculate how many times more energetic the threshold is compared to the highest observed cosmic ray.

\vspace{1em}

\textbf{5. Vacuum Lifetime Estimation} \\[0.5em]
If the vacuum decay rate per unit volume is $\Gamma = 10^{-100}$ m$^{-3}$s$^{-1}$, estimate the average time before a decay event occurs somewhere in the observable universe (volume $\sim 10^{78}$ m$^3$). \\
\emph{Question:} How does this compare to the current age of the universe ($\sim 10^{17}$ s)?

\vspace{1em}

\textbf{6. Parameter Sensitivity} \\[0.5em]
The Higgs quartic coupling at high energy scales depends sensitively on the top quark mass. Current measurements give $m_t = 173.1 \pm 0.9$ GeV. \\
\emph{Discussion:} Research how vacuum stability changes if the top mass were 2 GeV higher or lower. What does this tell us about the precision required in fundamental measurements?
}
