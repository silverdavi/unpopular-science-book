\begin{SideNotePage}{
    \textbf{Top (Metastable State):}  
    A system stuck in a local energy minimum. Though not at the true lowest-energy state, it remains trapped unless perturbed. The ball resting in a shallow well represents temporary stability — common in physical systems like false vacuum decay or the Sun. \par
  
    \textbf{Middle (Inertial vs Gravitational Mass):}  
    Both sides read "1KG," but differ in interpretation. Inertial mass resists acceleration (right), while gravitational mass dictates weight in a field (left). Despite equivalence, they arise differently. The Higgs mechanism contributes to inertial mass through interaction with the Higgs field; why gravitational mass follows, remains an open question in quantum gravity. \par
  
    \textbf{Bottom (Higgs Mechanism):}  
    A particle (left) moves through the Higgs field (right), acquiring mass via constant interaction. The disturbance in the field (rippled line) corresponds to a Higgs boson. The mechanism explains how otherwise massless particles — like W and Z bosons — gain mass while preserving gauge symmetry. \par
}{16_FalseVacuumThreat/16_ The Most Dangerous Hill in the Universe.pdf}
\end{SideNotePage}
