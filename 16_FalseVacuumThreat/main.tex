
% 1. What Counts as Empty
The 2012 discovery of the Higgs boson completed the Standard Model but raised an existential question: precision measurements placed our universe near the boundary between stable and metastable regimes. Our vacuum may be stable for now but not forever.

Classical physics defines vacuum as empty space — the absence of matter. This is not the case in quantum field theory. The vacuum is not emptiness but a specific configuration of fields filling all space. Fields are fundamental entities that assign values to every point in space. In quantum field theory, every point contains a quantum state for the electromagnetic field, the Higgs field, quark fields, and others. The vacuum is the configuration where these fields minimize the total energy density.

A quantum field extends throughout all space and determines the probabilities of measurement outcomes. The electromagnetic field carries light waves and radio waves. When this field vibrates in a particular pattern (mode), we observe it as a photon. Similarly, the electron field's vibrations manifest as electrons. Each fundamental particle type — quarks, leptons, bosons — corresponds to its own field. These fields exist everywhere, even in "empty" space. What we call particles are localized excitations, like waves on an ocean that pervades the universe. The vacuum is the state where all these fields vibrate with their lowest possible energy.

This redefinition matters because fields in their lowest energy state have physical effects. The Higgs field has a nonzero value throughout space, approximately 246 GeV. This value gives mass to fundamental particles through their couplings to the field. Without it, electrons would be massless, atoms could not form, and matter would not exist.

Whether this vacuum state is permanent depends on the field potential — a mathematical term describing the energy associated with different field values. Just as a marble rolls to the bottom of a bowl, fields evolve toward configurations that minimize their potential energy. The shape of this potential determines whether our vacuum is truly stable or merely appears so.

Potentials can have multiple minima. A local minimum is a dip surrounded by higher terrain — stable against small disturbances but not the lowest point. A ball in a shallow depression on a hillside stays put despite a deeper valley elsewhere. A false vacuum is a field configuration at a local minimum when a deeper minimum exists. Climbing out requires energy, so the field appears stable despite not occupying the true ground state.

The Higgs field is a scalar field — it has spin 0 and a single degree of freedom at each point in space, unlike vector fields (spin 1) or tensor fields (spin 2).

Spin is an intrinsic quantum property, analogous to but distinct from classical rotation. A spin-0 particle (scalar) has no preferred direction — it looks identical from every angle, like a sphere. A spin-1/2 particle (fermion) requires two full rotations to return to its original state, a distinctly quantum behavior with no classical analog — electrons, quarks, and all matter particles have spin 1/2. A spin-1 particle (vector) has a direction, like an arrow pointing in space. The photon, with spin 1, must have its electric and magnetic fields oriented perpendicular to its motion. A spin-2 particle (tensor) has even more complex directional properties — the graviton (hypothetical particle mediating gravity), if it exists, would be spin 2. The spin determines how particles behave under rotations and what kinds of fields they can create. Scalar fields like the Higgs are the simplest: just a number at each point in space, no direction.

Its potential at tree level (before quantum corrections) takes the shape:
\[
V(\phi) = \lambda(\phi^2 - v^2)^2
\]
This creates a "Mexican hat" shape: high at the center, dropping to a circular valley at radius $v \approx 246$ GeV. The field settles in this valley, breaking electroweak symmetry — the W and Z bosons become distinguishable from photons by acquiring mass.

This minimum determines fundamental parameters. The W and Z bosons acquire masses proportional to $v$. Quarks and leptons gain mass through Yukawa couplings to the Higgs. The location in the valley sets the mass spectrum of the Standard Model.

The tree-level potential is an approximation. Virtual particles — quantum fluctuations that briefly borrow energy from the vacuum — modify the effective potential. These corrections depend on energy scale: at higher energies, different virtual processes dominate.

Virtual particles contribute through quantum loops. A virtual top quark can appear from the vacuum, interact with the Higgs field, then disappear. Though fleeting, these processes change the effective potential. Heavy particles like the top quark contribute most strongly because their coupling to the Higgs is proportional to their mass. The top quark's virtual loops pull the Higgs potential downward at large field values, while the Higgs self-interactions and gauge boson loops push it upward. The competition between these effects determines vacuum stability.

The renormalization group — a mathematical tool that tracks how physical parameters change at different energy scales — shows how couplings evolve with energy scale $\mu$. The Higgs self-coupling $\lambda(\mu)$ and top Yukawa coupling $y_t(\mu)$ satisfy coupled differential equations. The large top quark mass means $y_t$ is close to 1 (the Yukawa coupling $y_t = \sqrt{2}m_t/v \approx 0.99$), and its contribution drives $\lambda$ downward as energy increases.

If $\lambda(\mu)$ becomes negative at high scales, the potential bends downward for large field values. A second minimum forms far from the electroweak scale. This new minimum can be deeper than the original, making our vacuum metastable rather than stable.

The Higgs boson mass, measured at $125.25 \pm 0.17$ GeV, and the top quark mass at $172.9 \pm 1.5$ GeV, determine the boundary between stability and metastability. These values place the Standard Model near the critical line.

A 2 GeV decrease in the top mass would ensure stability. A 2 GeV increase would guarantee metastability with a shorter lifetime. Current measurements suggest metastability with lifetime exceeding $10^{100}$ years due to quantum suppression.

This sensitivity transforms vacuum stability from philosophical speculation to experimental physics. Precision measurements of the top mass will determine whether our vacuum is stable or metastable.

The discovery of the Higgs boson at CERN in 2012 added a measurable aspect to this abstract question. Combined with precision measurements of the top quark mass, physicists could finally calculate whether our universe sits in a truly stable vacuum or a metastable one. The result was unsettling: we appear to live on the edge. If our vacuum is indeed metastable, the primary concern becomes quantum tunneling — the mechanism by which the field could spontaneously transition to a lower minimum despite the energy barrier.

Classical physics forbids transitions between separated minima — the field cannot climb over the barrier. Quantum mechanics allows tunneling through barriers. In field theory, this occurs via instantons: field configurations that interpolate between vacua in imaginary time, where time becomes a spatial dimension in the calculation.

The process nucleates a bubble of true vacuum within the false vacuum sea. The probability depends on the Euclidean action $S_E$ (the action calculated in imaginary time) of the optimal tunneling path:
\[
\Gamma/V \sim A \exp(-S_E/\hbar)
\]
where $A$ is a dimensional prefactor containing field fluctuation modes. For small energy differences between vacua, $S_E$ becomes large, exponentially suppressing the decay rate.

Once a critical bubble forms, energy differences drive its expansion. The true vacuum has lower energy density, creating pressure that accelerates the bubble wall outward. The wall approaches the speed of light, converting false vacuum to true vacuum.

The bubble wall itself is a domain wall — a boundary layer where the field smoothly transitions between the two vacuum values. Its thickness is set by the inverse mass scale of the field, usually microscopic. The energy density in the wall is large, concentrated in this thin shell. As the bubble expands, this energy gets diluted over larger surface area, but the total energy grows as the bubble engulfs more false vacuum volume. The wall accelerates outward under constant pressure, asymptotically approaching the speed of light.

Inside the bubble, physics changes. The Higgs field takes its new value, altering all particle masses and couplings. Electrons might become too heavy to orbit nuclei, or too light to be localized. The balance enabling atomic structure disappears. Chemistry and biology cease to exist through redefinition of all physical laws.

Matter encountering the advancing wall undergoes complete transformation. Particles defined by their interactions with the old vacuum value cannot exist in the new vacuum. The process is not gradual — as the wall passes, particle masses and interaction strengths change discontinuously. Protons might become unstable, quarks might not confine (bind together to form protons and neutrons), electromagnetic and weak forces might merge or separate differently. No information about the previous state survives because the encoding mechanism no longer exists.

Standard Model parameters yield $S_E/\hbar \sim 10^{600}$, giving a vacuum lifetime of approximately $10^{10^{120}}$ years. This exceeds the universe's age ($10^{10}$ years) by more orders of magnitude than there are particles in the observable universe.

High-energy processes cannot trigger decay. Cosmic rays reach $10^{11}$ GeV, far above the $10^{10}$ GeV scale where the Higgs self-coupling runs negative, yet have bombarded Earth for billions of years without incident. The LHC's $1.4 \times 10^4$ GeV collisions are negligible by comparison. Vacuum decay requires coherent field excitations over macroscopic regions, not pointlike particle collisions — a single high-energy impact excites fields only locally, insufficient to nucleate the critical bubble geometry needed for tunneling.

A nuclear war destroys cities but leaves physics intact. Vacuum decay replaces physics itself. The universe continues, but under different rules that may not permit matter, let alone life.

\newpage

\begin{commentary}[The unbearable lightness of being]
Of all existential threats — asteroids, pandemics, nuclear war — vacuum decay offers the ultimate consolation: we'll never know it happened. No final moments of terror, no last goodbyes, no time for regret. The bubble wall travels at light speed, so the universe's rewriting arrives simultaneously with news of its approach.

\begin{multicols}{2}
\small\itshape
When you attend a funeral,\\
It is sad to think that sooner or\\
Later those you love will do the same for you.\\
And you may have thought it tragic,\\
Not to mention other adjec-\\
Tives, to think of all the weeping they will do.\\
But don't you worry.\\
No more ashes, no more sackcloth.\\
And an armband made of black cloth\\
Will some day never more adorn a sleeve.\\
For if the bomb that drops on you\\
Gets your friends and neighbors too,\\
There'll be nobody left behind to grieve.\\[0.5em]
And we will all go together when we go.\\
What a comforting fact that is to know.\\
Universal bereavement,\\
An inspiring achievement,\\
Yes, we all will go together when we go.\\[0.5em]
We will all go together when we go.\\
All suffuse with an incandescent glow.\\
No one will have the endurance\\
To collect on his insurance,\\
Lloyd's of London will be loaded when they go.\\[0.5em]
Oh we will all fry together when we fry.\\
We'll be french fried potatoes by and by.\\
There will be no more misery\\
When the world is our rotisserie,\\
Yes, we will all fry together when we fry.\\[0.5em]
Down by the old maelstrom,\\
There'll be a storm before the calm.\\[0.5em]
And we will all bake together when we bake.\\
There'll be nobody present at the wake.\\
With complete participation\\
In that grand incineration,\\
Nearly three billion hunks of well-done steak.\\[0.5em]
Oh we will all char together when we char.\\
And let there be no moaning of the bar.\\
Just sing out a te deum\\
When you see that I.C.B.M.,\\
And the party will be "come as you are."\\[0.5em]
Oh we will all burn together when we burn.\\
There'll be no need to stand and wait your turn.\\
When it's time for the fallout\\
And Saint Peter calls us all out,\\
We'll just drop our agendas and adjourn.\\[0.5em]
You will all go directly to your respective Valhallas.\\
Go directly, do not pass Go, do not collect two hundred dolla's.\\[0.5em]
And we will all go together when we go.\\
Ev'ry Hottentot and ev'ry Eskimo.\\
When the air becomes uranious,\\
And we will all go simultaneous.\\
Yes we all will go together\\
When we all go together,\\
Yes we all will go together when we go.
\end{multicols}

\normalsize\hfill\textit{— Tom Lehrer, from tomlehrersongs.com, allowed for any use by the author}
\end{commentary}
