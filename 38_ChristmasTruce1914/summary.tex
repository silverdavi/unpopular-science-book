On Christmas 1914, enemy soldiers climbed out of their trenches and shook hands. Along sectors of the Western Front, British and German troops spontaneously ceased fire, met in No Man's Land to exchange tobacco and souvenirs, sang carols together, and buried their dead side by side. Some kicked footballs around shell craters. This unofficial truce — neither ordered nor sanctioned — lasted hours to days depending on location. By Christmas 1915, high command used coordinated artillery barrages to prevent any recurrence. The truce demonstrated humanity's persistence within industrialized killing, yet changed nothing about the war that would claim 17 million lives. A momentary refusal to reduce enemies to targets, quickly suppressed by the machinery of total war.
